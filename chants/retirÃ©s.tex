\beginsong{Tu sens bon la terre}[by={Hugues Aufray}, cr={1964}]
\beginverse
\[Am]Tu sens bon la \[E7]terre, ma \[Am]terre,\[E7]
\[Am]Tu sens bon la \[G7]vie, ma \[C]mie.\[E7]
\[Am]Tu sens bon la \[G]pipe, mon \[Am]père,
\[Am]Tu sens bon l'\[E7]automne, ma \[Am]pluie.\[E7]
\endverse

\beginverse
Je n'ai pas d'argent dans la tête
Et pas d'argent dans les mains,
Mais pour moi, c'est toujours fête
Puisque j'ai du pain et du vin.
\endverse

\beginverse
Tu sens bon la terre, ma terre,
Tu sens bon le chien, mon chien.
Tu sens bon le linge, ma mère,
Tu sens bon l'été, mon jardin.
\endverse

\beginverse
Quand on a fini sa journée,
Qu'on n'a pas volé ses sous,
On se trouve à la veillée
Entre amis pour boire un bon coup.
\endverse

\beginverse
Tu sens bon la terre, ma terre,
Tu sens bon le pain, mon pain.
Tu sens bon l'école, p'tit frère,
Tu sens bon l'printemps, mon refrain.
\endverse
\endsong

\beginsong{Je reviens}[by={Hugues Aufray},cr=1966]
\beginverse
J'ai cou\[Bm]ru à travers le monde
De Sha\[G]nghaï ju\[A7]squ'à San Fer\[D]nando
Sous le vent et la tem\[F#7]pête
Pour toi j'ai mené mon \[Bm]bateau
Je reviens le cœur en f\[F#7]ête
Jusqu'aux portes de Saint-M\[Bm]alo
\endverse

\beginchorus
Je \[G]reviens, je \[A7]reviens, je reviens au \[D]pays
Sous le vent et la t\[F#7]empête
Pour toi j'ai mené mon ba\[Bm]teau
Je reviens le cœur en \[F#m]fête
Jusqu'aux portes de Saint-\[Bm]Malo
\endchorus

\beginverse
Droit devant, depuis vingt semaines
D'Amsterdam à l'île de Bornéo
J'ai souvent cru que l'orage
Finirait par avoir ma peau
Mais j'ai retrouvé courage
Et le chemin de Saint-Malo
\endverse

\beginverse
Sous les feux ou les vents de glace
D'Istanbul jusqu'à Valparaiso
J'ai fait le tour de la terre
Voile au vent sur le Santiago
Par les portes de l'enfer
Qui conduisaient à Saint-Malo
\endverse

\beginverse
Cheveux noirs ou bien cheveux d'ange
De Lisbonne au port de San Diego
Mes amours, mes demoiselles
S'envolaient comme des oiseaux
C'était toi vraiment la plus belle
De New York à Saint-Malo
\endverse
\endsong

\beginsong{Le bon Dieu s'énervait}[by={Hugues Aufray},cr=1966]
\beginverse
\[E]Le bon Dieu s'énervait dans son \[A]atelier.
\[E]"Ça fait trois ans que j'ai \[B7]planté cet arbre
\[E]Et j'ai beau l'ar\[E7]roser à lon\[A]gueur de journée,
Il pousse \[E]encore moins \[B7]vite que ma \[E]barbe."\[B7]
\endverse

\beginchorus
Pour faire un \[E]arbre, \[A]Dieu que c'est long.
Pour faire un \[E]arbre, \[B7]Dieu que c'est long.
Pour faire un \[E]arb\[E7]re, \[A]Dieu que c'est long.
Pour faire un \[E]arbre, mon \[B7]Dieu que c'est \[E]long.\[B7]
\endchorus

\beginverse
Le bon Dieu s'énervait dans son atelier.
"Sur ce maudit baudet, dix ans j'ai travaillé.
Je n'arrive pas à le faire avancer
Et encore moins à le faire reculer."
\endverse

\beginchorus
Pour faire un âne...
\endchorus

\beginverse
Le bon Dieu s'énervait dans son atelier
En regardant Adam marcher à quatre pattes.
"Et pourtant, nom d'une pipe, j'avais tout calculé
Pour qu'il marche sur ses deux pieds."
\endverse

\beginchorus
Pour faire un homme...
\endchorus

\beginverse
Le bon Dieu s'énervait dans son atelier
En regardant le monde qu'il avait fabriqué.
"Les gens se battent comme des chiffonniers
Et je ne peux plus dormir en paix."
\endverse

\beginchorus
Pour faire un monde...
\endchorus
\endsong


\beginsong{Le coeur gros}[by={Hugues Aufray},cr=1964]
\beginverse
\[D]Quand revient le vent de l'\[Bm]automne,
\[Bm]Je pense à tout ce temps per\[Em]du.
\[Em]Je n'ai fait de mal à \[A7]personne.
\[A7]Je n'ai pas fait de bien non \[D]plus
\[D]Et \[A7]j'ai le cœur \[D]gros.
\endverse

\beginverse
Pauvre chien perdu dans la ville,
Y a des abris pour toi, mon vieux.
On a la conscience tranquille
Et quand on regarde tes yeux,
On a le cœur gros.
\endverse

\beginverse
Après des mois de mauvais coups,
De filets pleins de goémons,
Quand le marin compte ses sous,
Sur la table de la maison,
Il a le cœur gros.
\endverse

\beginverse
Toi qui n'es pas mort à Madrid
Où tant de copains sont restés
Quand tu regardes tes mains vides
Et devant ton fusil rouillé,
Tu as le cœur gros.
\endverse

\beginverse
Quand tu l'as vu porté en terre,
Son cheval noir marchant devant,
Tu as soudain compris, mon frère,
Qu'il étaient plus qu'un président.
T'as eu le cœur gros.
\endverse

\beginverse
Assis au bord de la rivière,
Mes rêves suivent leur chemin,
Mais quand je pense qu'il y a sur terre
Deux enfants sur trois qui ont faim,
Moi, j'ai le cœur gros.
\endverse

\beginverse
Adieu fillette, adieu ma mie,
Adieu petite, le temps court.
Les cigognes sont reparties.
Elles reviendront sur'ment un jour.
N'aie plus le cœur gros.
\endverse
\endsong


\beginsong{Le joueur de pipeau}[by={Hugues Aufray},cr=1966]
\beginverse
\[Dm]Un é\[C]tran\[Dm]ger est \[C]arri\[Dm]vé un \[C]beau \[Dm]soir
\[Gm]De son\[F] pi\[Gm]peau il \[F]tirait \[Gm]des sons \[F]bizar\[Gm   ]res
\[Dm]Ses \[C]cheveux \[Dm]longs lui \[C]donnaient \[Dm]l'air d'un \[B&]vaga\[C]bond
\endverse

\beginverse
En ce temps-là la ville était envahie
Par tous les rats venant du fond du pays
Privés de pain les habitants mouraient de faim
\endverse

\beginverse
Le musicien leur dit "Si vous le voulez
Je peux sur l'heure du fléau vous délivrer"
Pour mille écus le marché fût bientôt conclu
\endverse

\beginverse
Devant l'église il joua de son pipeau
Comme le berger lui rassemble son troupeau
Et de partout les rats sortirent de leur trou
\endverse

\beginverse
Et tous ces rats qui le suivaient dans la rue,
Chemin faisant ils étaient cent mille et plus
Ils arrivèrent à la rivière et s'y noyèrent
\endverse

\beginverse
"C'est un sorcier!" S'écrièrent les bourgeois
Et déjà chacun le désignait du doigt
À coups de pierres et sans parjures ils le chassèrent
\endverse

\beginverse
Tout le village dormait paisiblement
Lorsque soudain on entendit dans le vent
Un doux refrain que les enfants connaissait bien
\endverse

\beginverse
Les p'tits enfants en chemise de nuit
Cherchaient le vent et le pipeau dans la nuit
Ils arrivèrent à la rivière et s'y noyèrent
\endverse
\endsong

\beginsong{Le port de Tacoma}[by={Hugues Aufray},cr=1968]
\beginverse
\[C]C'est dans la cale qu'on met les rats, houla \[G]la houla,
C'est dans la cale qu'on met les rats, houla houla.
\endverse

\beginchorus
P\[C]arés à virer,
Les \[F]gars, faudrait \[C]haler.
On s' repos'ra quand on arriv'ra
Dans le \[G]port de Ta\[C]coma.
\endchorus

\beginverse
C'est dans la mer qu'on met les mâts, houla la houla,
C'est dans la mer qu'on met les mâts, houla houlala.
\endverse

\beginverse
C'est dans la pipe qu'on met l'tabac, houla la houla,
C'est dans la pipe qu'on met l'tabac, houla houlala.
\endverse

\beginverse
C'est dans la gueule qu'on se met l'tafia, houla la houla,
C'est dans la gueule qu'on se met l'tafia, houla houlala.
\endverse

\beginverse
Mais les filles, ça s'met dans les bras, houla la houla,
Mais les filles, ça s'met dans les bras, houla houlala.
\endverse
\endsong

%%%%%%%%%%%%%%%%%%%%%%%%%%%%%%%%%%% Pauvre Martin
\beginsong{Pauvre Martin}[by={Georges Brassens}, cr={1953}]
\beginverse
Avec \[Am]une bêche à l'\[Am/E]épaule,
Avec, \[Am]à la lèvre, un \[Am/E]doux chant, \[Am] \[Am/E]
Avec, \[Dm]à l'âme, un grand co\[C]urage,
Il s'en \[E/B]allait trimer \[E]aux champs! \[E/B]\[E]
\endverse

\beginchorus
\[Am]Pauvre Martin, \[Am/E]pauvre misère,
\[Am]Creuse la terr', \[E]creuse le temps!\[Am]\[Am/E]
\endchorus

\beginverse
Pour gagner le pain de sa vie,
De l'aurore jusqu'au couchant,
De l'aurore jusqu'au couchant,
Il s'en allait bêcher la terre
En tous lieux, par tous les temps!
\endverse

\beginverse
Sans laisser voir, sur son visage,
Ni l'air jaloux ni l'air méchant,
Ni l'air jaloux ni l'air méchant,
Il retournait les champs des autres,
Toujours bêchant, toujours bêchant!
\endverse

\beginverse
Et quand la mort lui a fait signe
De labourer son dernier champ,
De labourer son dernier champ,
Il creusa lui-même sa tombe
En faisant vite, en se cachant...
\endverse

\beginverse
Il creusa lui-même sa tombe
En faisant vite, en se cachant,
En faisant vite, en se cachant,
Et s'y étendit sans rien dire
Pour ne pas déranger les gens...
\endverse

\beginchorus
Pauvre Martin, pauvre misère,
Dors sous la terr', dors sous le temps!
\endchorus
\endsong


%%%%%%%%%%%%%%%%%%%%%%%%%%%%%%%%%%%
\beginsong{Il n'y a pas d'amour heureux}[by={Georges Brassens}, cr={1965}]
\beginverse
\[Am]Rien n'est jamais acquis à l'homme, ni sa \[Dm ]force
Ni sa faiblesse ni\[Edim7] son cœur... et quand il \[Am]croit
Ouvrir ses bras son \[D]ombre est \[F]celle d'une \[E7]croix \[Am]
Et quand il croit \[Dm]serrer son bonheur il le \[G]broie
Sa vie est un é\[C]trange et douloureux di\[E7]vorce
Il n'y a pas d'amour he\[Am]ureux
\endverse

\beginverse
Sa vie elle ressemble à ces soldats sans armes
Qu'on avait habillés pour un autre destin
A quoi peut leur servir de se lever matin
Eux qu'on retrouve au soir désarmés, incertains
Dites ces mots : "ma vie" et retenez vos larmes
Il n'y a pas d'amour heureux
\endverse

\beginverse
Mon bel amour, mon cher amour, ma déchirure
Je te porte dans moi comme un oiseau blessé
Et ceux-là sans savoir nous regardent passer
Répétant après moi ces mots que j'ai tressés
Et qui pour tes grands yeux tout aussitôt moururent
Il n'y a pas d'amour heureux
\endverse

\beginverse
Le temps d'apprendre à vivre il est déjà trop tard
Que pleurent dans la nuit nos cœurs à l'unisson
Ce qu'il faut de regrets pour payer un frisson
Ce qu'il faut de malheur pour la moindre chanson
Ce qu'il faut de sanglots pour un air de guitare
Il n'y a pas d'amour heureux...
\endverse
\endsong


\beginsong{Quand il est mort le poète}[by={Gilbert Bécaud},cr=1963]
\beginverse
\[G]Quand il est mort, le po\[Am]è\[D7]te,
\[D7]Quand il est mort, le po\[G]ète,
\[G]Tous ses \[G7]amis,
\[C]Tous ses \[Cm]amis,
\[G]Tous ses \[A7]amis pleu\[D7]raient.\[D7]

Quand il est mort le poète,
Quand il est mort le poète,
Le monde entier,
Le monde entier,
Le monde entier \[D7]pleu\[G]rait.

On enterra son étoile,
On enterra son étoile,
Dans un grand champ,
Dans un grand champ,
Dans un grand champ de \[D7]blé.\[D7]

Et c'est pour ça que l'on trouve,
Et c'est pour ça que l'on trouve,
Dans ce grand champ,
Dans ce grand champ,
Dans ce grand champ, \[A]des \[D7]bleu\[G]ets.

\[G]La, la, la...\[Am]\[D7]
\[D7]La, la, la...\[G] La, la, la...\[G7]
\[C]La, la, la...\[Cm] \[G]La, la, la...\[D7]\[A]des \[D7]bleu\[G]ets
\endverse
\endsong


\beginsong{L'Amérique}[by={Joe Dassin},cr=1970]
\beginverse
\[A]Les amis, je dois \[C#m]m'en aller
\[F#m]Je n'ai plus qu'à \[C#m]jeter mes clés
Car \[F#m]elle m'attend depuis \[Bm7]que je suis né
\[E]L’Amérique\[E7]
J'abandonne sur mon chemin
Tant de choses que j'aimais bien
Cela commence par un peu de chagrin
L’Amérique
\endverse

\beginchorus
Mais \[A]L'Amérique, \[A]l'Amérique, je \[C#m]veux l'avoir et \[E7]je l'aurai
\[A]L'Amérique, \[A]l'Amérique, si \[C#m]c'est un rêve, je \[E]le saurai
Tous \[F#m]les sifflets de trains, toutes les sir\[E]ènes de bateaux
\[D]M'ont chanté cent fois la chanson \[E]de l'Eldorado
De l\[F#m]'Amérique\[F#m]\[D]\[E7]
\endchorus

\beginverse
Les amis, je vous dis adieu
Je devrais vous pleurer un peu
Pardonnez-moi si je n'ai dans les yeux
Que l'Amérique
Je reviendrai je ne sais pas quand
Cousu d'or et brodé d'argent
Ou sans un sou, mais plus riche qu'avant
De l'Amérique
\endverse

\beginchorus
L’Amérique, l'Amérique, je veux l'avoir et je l'aurai
L'Amérique, l'Amérique, si c'est un rêve, je le saurai
Tous les sifflets de trains, toutes les sirènes de bateaux
M'ont chanté cent fois la chanson de l'Eldorado
De l'Amérique
L’Amérique, l'Amérique, je veux l'avoir et je l'aurai
L'Amérique, l'Amérique, si c'est un rêve, je le saurai
L'Amérique, l'Amérique, si c'est un rêve, je rêverai
L'Amérique, l'Amérique, si c'est un rêve, je veux rêver
\endchorus
\endsong



\beginsong{Bonjour à toi l'artiste}[by={Nicole Rieu},cr=1975]
\beginverse
\[D]Et bonjour à \[A]toi, l'artiste\[G] de n'importe\[D] où
Qui fait le jour gais ou tristes, toi qui changes\[A] tout
\[G]Tu nous offres la \[D]musique \[A] comme un ca\[Bm]deau
\[Em7]Toi le magicien des temps nouv\[A4]eaux \[A]
Et bonjour à toi, le peintre de la lumière
Qui connaîs toutes les teintes de l'univers
Tu vas faire de l'an deux mille un millénaire
Le plus beau de l'histoire de la terre
\endverse

\beginverse
Il est\[D] temps d'acheter des cou\[A]leurs
Il est\[Bm] temps de te mettre au la\[F#m]beur
Il est\[G] temps Toi, le compo\[D]siteur
De te do\[Em]nner de tout ton c\[A]œur
Il est temps Et si tu commençais
Maintenant Demain tout serait prêt
Il est temps Tu fais ce qui te plaît
Prend tout ton \[Am]temps
Mais ne le prends pas trop, s'il t\[A4]e plaît\[A]
\endverse

\beginverse
Et bonjour à toi, l'artiste, le grand auteur
Le brillant illusioniste, le célèbre acteur
Tu vas nous changer le monde, tu vas chanter
Pour nous faire oublier le passé
\endverse
\endsong


\beginsong{Couleur Café}[by={Serge Gainsbourg},cr=1964]
\beginverse
J'\[G]aime \[C]ta couleur ca\[G7]fé, t\[C]es cheveux ca\[G7]fé
Ta\[C] gorge ca\[G7]fé j'\[C]aime quand pour \[G7]moi tu danses
Alors j'entends murmurer, tous tes bracelets
Jolis bracelets, à tes pieds ils se balancent
\endverse

\beginchorus
\[C]Couleur \[G/]café, Que j'aime ta couleur ca\[C]fé
\endchorus

\beginverse
C'est quand même fou l'effet, l'effet que ça fait
De te voir rouler, ainsi des yeux et des hanches
Si tu fais comme le café, rien qu'à m'énerver
Rien qu'à m'exciter, ce soir la nuit sera blanche
\endverse

\beginverse
L'amour sans philosopher, c'est comm' le café
Très vite passé, mais que veux tu que j'y fasse
On en a marr' de café, et c'est terminé
Pour tout oublier, on attend que ça se tasse
\endverse
\endsong


\beginsong{Qu'as-tu appris à l'école}[by={Graeme Allwright},cr=1968]
\beginchorus
\[C]Qu'as-tu appris à l'école, mon fils, à l'école aujourd'\[G7]hui ?
\[C]Qu'as-tu appris à l'école, mon fils à l'école \[G]aujourd'\[C]hui ?
\endchorus

\beginverse
J'ai \[F]appris qu'il n'faut ment\[C]ir jamais
Qu'il y a des bons et des mauvais
Que je suis libre comme tout le monde
Même \[F]si le maître par\[G]fois me gronde
C'est \[C]ça qu'on m'a dit à l'é\[F]cole, Papa\[C]
C'est ça qu'on m'a \[G7]dit à l'é\[C]cole
\endverse

\beginverse
Que les gendarmes sont mes amis
Et tous les juges très gentils
Que les criminels sont punis pourtant
Même si on s'trompe de temps en temps
C'est ça qu'on m'a dit à l'école, Papa
C'est ça qu'on m'a dit à l'école
\endverse

\beginverse
Que le gouvernement doit être fort
A toujours raison et jamais tort
Nos chefs sont tous très forts en thème
Et on élit toujours les mêmes
C'est ça qu'on m'a dit à l'école, Papa
C'est ça qu'on m'a dit à l'école
\endverse

\beginverse
J'ai appris que la guerre n'est pas si mal
Qu'il y a des grandes et des spéciales
Qu'on s'bat souvent pour son pays
Et p't'être j'aurais ma chance aussi
C'est ça qu'on m'a dit à l'école, Papa
C'est ça qu'on m'a dit à l'école.
\endverse
\endsong

\beginsong{Le jour de clarté}[by={Graeme Allwright},cr=1968]
\beginchorus
Quand tous les affam\[C]és \[Dm] et tous les opprim\[C]és \[Dm]
Entendront tous l'\[F]appel \[C] \[B&] le cri de li\[A]berté \[Em7] \[B&]
Toutes les chaînes bris\[Dm]ées \[C]tomberon\[B&]t pour l\[A]'éternité \[Dm] \[C]\[Dm]
\endchorus

\beginverse
On peut chant\[F]er tous les \[C]poèmes des\[Dm] sages
Et on peut parler de l'humilité
Mais il faut\[F] s'unir pour\[C] abolir \[B&]injustice et \[C]pauvreté
\[Dm]Les hommes\[F] sont tous \[G]pareils\[A], \[Dm]ils ont \[F]tous le même \[G]soleil\[A]
Il faut, mes frères, préparer le jour de clarté
\endverse

\beginchorus
LE jour de \[Dm]clarté\[C] \[Dm] quand tous \[Dm]les \[C]affamés\[Dm]
Entendront tous l'\[F]appel \[C] \[B&] le cri de li\[A]berté \[G] \[A]
Toutes les chaînes bris\[Dm]ées \[C]tomberon\[B&]t pour l\[A]'éternité \[Dm] \[C]\[Dm]
\endchorus

\beginverse
On peut discut\[F]er sur les \[C]droits de \[Dm]l'homme
Et on peut parler de fraternité
Mais qu'les \[F]hommes soient \[C]jaunes ou \[B&]blancs ou no\[C]irs
Ils ont\[F] la même destinée\[Am]
\[Dm]Laissez \[F]vos préjugés\[G] \[A]
Rejetez vos vieilles idées
Apprenez seulement l'amitié
\endverse

\beginverse
On ne veut plus parler de toutes vos guerres
Et on n'veut plus parler d'vos champs d'honneur
Et on n'veut plus rester les bras croisés
Comme de pauvres spectateurs
Dans ce monde divisé
Il faut des révoltés
Qui n'auront pas peur de crier
\endverse
\endsong

\beginsong{Emmène moi}[by={Graeme Allwright}, cr={1966}]
\beginverse
J'ai \[D]voyagé de Brest à Besançon
Depuis \[A]la Rochelle jusqu'en A\[D]vignon
De \[D]Nantes jusqu'à Monaco
En p\[A7]assant par Metz et Saint-\[D]Malo et \[D]Paris
Et j'ai \[D]vendu des marrons à la foire de Dijon
Et d'la \[A7]barbe à p\[D]apa
\endverse

\beginchorus
\[D]Emmène-mo\[G]i
Mo\[A7]n cœur est triste et j\[D]'ai mal \[D]aux pieds
Emmène-mo\[G]i
\[A7]Je ne veux plus voyag\[D]er
\endchorus

\beginverse
J'ai dormi toute une nuit dans un abreuvoir
J'ai attrapé la grippe et des idées noires
J'ai eu mal aux dents et la rougeole
J'ai attrapé des rhumes et des p'tites bestioles
Qui piquent
Sans parler de toutes les fois que j'ai coupé mes doigts
Sur une boîte à sardines
\endverse

\beginverse
Je les vois tous les deux comme si c'était hier
Au coucher du soleil, Maman mettant l'couvert
Et mon vieux Papa avec sa cuillère
Remplissant son assiette de pommes de terre
Bien cuites
Et les dimanches Maman coupant une tranche
De tarte aux pommes
\endverse
\beginchorus
\textnote{Refrain x2}
\endchorus
\endsong

\beginsong{Jusqu'à la ceinture}[by={Graeme Allwright}, cr={1968}]
\beginverse
En \[Am]mil-neuf-\[G]cent-qua\[F]rante-\[E]deux
Alors que \[Am]j'étais \[E]à l'a\[Am]rmée
On \[Am]était en manœuvre \[G]dans la \[F]Loui\[G]siane
Une \[Am]nuit au mois de \[E7]mai
Le \[Am]cap\[G]itaine nous \[F]montre un \[G]fleuve
Et c'est comme \[Dm]ça que tout a commen\[E7]cé
On a\[Am]vait d'la \[G]flotte \[F]jusqu'aux \[E7]g'noux
Et le vieux \[Am]con a \[E]dit   d'avan\[Am]cer
\endverse

\beginverse
Le sergent dit: " Mon capitaine,
Etes-vous sûr qu'c'est le chemin ? "
- " Sergent, j'ai traversé souvent
Et je connais bien le terrain
Allons, soldats, un peu de courage !
On n'est pas là pour s'amuser "
Y'en avait jusqu'à la ceinture
Et le vieux con a dit d'avancer
\endverse

\beginverse
Le sergent dit: " On est trop chargés
On ne pourra pas nager "
- " Sergent ne sois pas si nerveux
Il faut un peu de volonté
Suivez-moi: je marcherai devant
Je n'aime pas les dégonflés "
On avait d'la flotte jusqu'au cou
Et le vieux con a dit d'avancer
\endverse

\beginverse
Dans la nuit, soudain, un cri jaillit
Suivi d'un sinistre glou-glou
Et la casquette du capitaine
Flottait à côté de nous
Le sergent cria: " Retournez-vous
C'est moi qui commande à présent "
On s'en est sortis juste à temps
Le capitaine est mort là-dedans
\endverse

\beginverse
Le lendemain, on a trouvé son corps
Enfoncé dans les sables mouvants
Il s'était trompé de cinq cents mètres
Sur le chemin qui mène au camp
Un affluent se jetait dans le fleuve
Où il croyait la terre tout près
On a eu de la chance de s'en tirer
Quand ce vieux con a dit d'avancer
\endverse

\beginverse
La morale de cette triste histoire
Je vous la laisse deviner
Mais vous avez peut-être mieux à faire
Vous n'vous sentez pas concernés
Mais chaque fois que j'ouvre mon journal
Je pense à cette traversée
On avait de la flotte jusqu'aux genoux
Et le vieux con a dit d'avancer
Y'en avait jusqu'à la ceinture
Et le vieux con a dit d'avancer
Y'en avait de la flotte jusqu'au cou
Et le vieux con a dit d'avancer...
Y'en avait jusqu'à....
\endverse
\endsong

\beginsong{Jolie bouteille, sacrée bouteille}[by={Graeme Allwright}, cr={1966}]
\beginchorus
\[C]Jolie bouteille, sacrée bouteille,
Veux-tu me laissser tran\[G7]quil\[C]le?
Je veux te quitter, je veux m’en aller,
Je veux recommencer ma vie.
\endchorus

\beginverse
\[F]J’ai traîn\[G]é dans \[F]tous les caf\[C]és
J’ai fait la \[G7]manche bien des \[C]soirs
Les temps sont durs et j’suis même pas sûr
De me payer un coup à boire.
\endverse

\beginverse
J’ai mal à la tête
Et les punaises me guettent
Mais que faire dans un cas pareil?
Je demande souvent
Aux passants
De me payer une bouteille.
\endverse

\beginverse
Dans la nuit
J’écoute la pluie
Un journal autour des oreilles.
Mon vieux complet
Est tout mouillé
Mais j’ai toujours ma bouteille.
\endverse

\beginverse
Chacun fait
Ce qui lui plaît
Tout l’monde veut sa place au soleil.
Mais moi, j’m’en fous.
J’n’ai rien du tout,
Rien qu’une jolie bouteille.
\endverse
\endsong

\beginsong{Ca je ne l'ai jamais vu}[by={Graeme Allwright}, cr={1966}]
\beginverse
\[C]J’entre à la maison, l’autre nuit, j’avais \[F]bu un peu de vi\[C]n
J’ai \[C]vu un ch’val dans l’écurie où je mettais le \[G]mien
Al\[C]ors j’ai dit à ma p’tite fe\[F]mme: " Veux-tu bien m’expliquer\[C]
Y a\[C] un cheval dans l’écurie à la p\[G]lace de mon bidet\[C] ? "
" Mon \[F]pauvre ami, tu n’vois pas clair, le vin t’a trop saoulé\[C]
Ce n’est rien qu’une vache à \[D7]lait que ta mère m’a donnée\[G7] "
Dans la \[C]vie, j’ai vu pas mal de choses biz\[F]arres et saugrenues\[C],
Mais une selle sur une vache à \[C]lait, ça je n’ai j\[G]amais vu.\[C]
\endverse

\beginverse
La nuit suivante j’entre chez moi, j’avais bu un peu de vin
J’ai vu un chapeau accroché où j’accrochais le mien
Alors j’ai dit à ma p’tite femme: " Veux-tu bien m’expliquer
Qu’est-ce que c’est qu’ce chapeau-là à la place de mon béret? "
" Mon pauvre ami, tu n’vois pas clair, le vin t’a trop saoulé
Ce n’est rien qu’une vieille casserole que grand-mère m’a donnée "
Dans la vie j’ai vu pas mal de choses bizarres et saugrenues
Mais une vieille casserole en feutre, ça je n’ai jamais vu
\endverse

\beginverse
Une nuit plus tard j’entre chez moi, j’avais bu un peu de vin
Sur une chaise, j’ai vu un pantalon où je posais le mien
Alors j’ai dit à ma p’tite femme: " Je voudrais bien savoir
Pourquoi ce pantalon est gris, le mien est toujours noir "
" Mon pauvre ami, tu n’vois pas clair, le vin t’a trop saoulé
Ce n’est rien qu’un vieux chiffon que maman m’a donné "
Dans la vie j’ai vu pas mal de choses, mais ça c’est un mystère
Un chiffon avec deux tuyaux et une fermeture éclair
\endverse

\beginverse
En titubant, j’entre chez moi, je suis resté baba
J’ai vu une tête sur l’oreiller qui n’me ressemblait pas
Alors j’ai dit à ma p’tite femme: " Peux-tu m’expliquer ça
Qu’est-ce que c’est qu’cette tête-là, je n’pense pas qu’c’est moi! "
" Mon pauvre ami, tu n’vois pas clair, le vin t’a trop saoulé
Ce n’est rien qu’un vieux melon que grand-père m’a donné "
Des prix de concours agricoles, j’peux dire que j’en ai eus
Mais une moustache sur un melon, ça je n’ai jamais vu 
\endverse
\endsong

\beginsong{Travailler c'est trop dur}[by={Zachary Richard}]
\beginchorus
\[G]Travailler c’est trop dur, et \[C]voler c’est pas \[G]beau.
Demander la ch\[D]arité c’est quelque \[C]chose que j’peux pas fai\[G]re.
Chaque jour que moi je vis, on me demande de quoi moi je vis,
Je dis que je vis sur l’amour et j’espère de vivre vieux.
\endchorus

\beginverse
Moi je prends mon violon et j’attrape mon archet,
Et je joue ma veille valse pour faire mes amis danser.
Vous connaissez mes chers amis la vie est bien bien trop courte
Pour se faire des misères, allons danser ce soir.
\endverse

\beginverse
Moi je fais la musique
C’est presque tous les soirs,
Après trainer tout par tout
Et puis chanter dans les whiskie bars,
Et des fois, tu connais,
J’aimerais lacher, puis m’em aller,
Mais je suis venu ce soir
Pour le plaisir de chanter.
\endverse
\textnote{Refrain\rep{2}}
\endsong


\beginsong{V'là l'bon vent}[by={Édouard-Gabriel Midon}]
\beginverse
Derrière chez \[Em]nous y a un \[Am]étang,  
trois beaux \[G]canards s'y vont nageant.
Y en a deux noirs, y en a un blanc.
\endverse

\beginchorus
V'\[Am]là l'bon vent, v'là l'joli vent,
v'là l'bon vent, ma mie m'\[E]appelle.
V'\[Am]là l'bon vent, v'là l'joli vent,
v'là l'bon vent, ma \[E7]mie m'a\[Am]ttend.
\endchorus

\beginverse
Le fils du roi s'en va chassant
avec son grand fusil d'argent;
mire le noir et tue le blanc.
\endverse

\beginverse
O fils du roi, tu es méchant
d'avoir tué mon canard blanc.
Par-dessous l'aile il perd son sang.
\endverse

\beginverse
Par les yeux lui sort des diamants
et par le bec l'or et l'argent.
Toutes ses plumes s'envolent au vent.
\endverse

\beginverse
Trois dames s'en vont les ramassant.
C'est pour en faire un lit de camp
pour y coucher tous les passants.
\endverse
\endsong

\beginsong{Ex-fan des sixties}[by={Jane Birkin},cr={1978}]
\beginchorus
Ex\[Am7]-fan des six\[D7]ties petite Baby \[G]Doll
\[Am7]Comme tu dansais \[D]bien le Rock 'n '\[Bm]Roll
\[Am7]Ex-fan des six\[D]ties où sont tes années \[Em]folles
\[Am7]Que sont devenues \[D]toutes tes \[C]idoles ?\[G]\[G]\[Am7]\[G/B]

\endchorus

\beginverse
\[C] Où est l'ombre des \[F# m 7/5 b]Shadows\[Bm] des Byrds, des D\[Em7]oors
\[Am7]des Animals, des \[D7]Moo-dy \[G]Blues ?
\[G]\[G]\[Am7]\[G/B]
Séparés Mac Cartney,\[Bm7] Georges Harrison
et Ringo Starr, et John Lennon 
\endverse

\beginverse
\[Cm7]Disparus B\[A 7/5bb]rian Jones,\[Dm7] Jim Morri\[Gm7]son
\[Cm7]Eddy Cochran,\[Fm7]Buddy Holl\[B&]y\[B&]\[Cm7]\[B b/D]

\endverse

\beginverse
\[Em7]Idem Jimi \[Am 7/5b]Hendri,\[Dm7] Otis Redd\[Gm7]ing
\[Cm7]Janis Joplin, T.Rex,\[F7] El\[B&]vis\[B&]\[Cm7]\[B b/D]
\endverse
\endsong


\beginsong{Vive le Douanier Rousseau}[by={La\ Compagnie\ Créole},cr={1983}]
\beginverse
\[G]Bonjour, bonjour, je \[D]viens vous inviter.
Laissez tout tomber, \[G]on va embarquer
Pour un pays qui \[D]va vous enchanter,
Vous embéguiner. \[G]Laissez-vous tenter.
\endverse

\beginchorus
\[C]C'est une île perdue au m\[Bm]ilieu de l'océan,
Un \[A]jardin merveilleux, un sp\[D]ectacle permanent,
[D]Comme dans les, {comme dans les}\rep{2}
Comme dans \[Em]les tabl\[D]eaux du Douanier Rouss\[G]eau,
Y'a des perroquets \[D]bleus qui boivent du lait d'co\[G]co,
Comme dans les tableaux du Douanier Rousseau,
Y'a des poissons tropicaux pleins d' piquants sur le dos, 
Oh\[Am]o, oh\[Bm]o oh\[C]o...\[c]\[D]
\endchorus

\beginverse
La nuit tombée, si vous le voulez,
On ira canoter sous les palétuviers.
Aucun danger, on peut se baigner :
Là-bas, les crocodiles sont bien intentionnés.
\endverse

\beginchorus
Au clair de la lune, dans la forêt endormie,
Des ombres félines se dessinent par magie
Comme dans les, {comme dans les}
Comme dans les, {comme dans les}
Comme dans les tableaux du Douanier Rousseau,
Y'a des soleils de feu cachés dans les roseaux,
Comme dans les tableaux du Douanier Rousseau,
Y'a des p'tits singes amoureux
Qui jouent les Roméo, oh oh, oh oh, oh oh…
\endchorus

\beginverse
\[G]Tou, tou, tou, {Baoum}
\[D]Tou, tou, tou, {Baoum}
\[G]Tou, tou, tou, {Baoum}
\[D]C'est un vrai para\[G]dis !
\endverse

\beginchorus
C'est une île perdue au milieu de l'océan,
Un jardin merveilleux, un spectacle permanent,
Comme dans les, {comme dans les}
Comme dans les, {comme dans les}
Comme dans , {comme dans}, comme dans
Comme dans les tableaux du Douanier Rousseau,
Y'a des perroquets bleus qui boiv'nt du lait d' coco,
Comme dans les tableaux du Douanier Rousseau,
Y'a des poissons tropicaux pleins d' piquants sur le dos,
\endchorus

\beginverse
Comme dans les tabl\[D]eaux du Douanier Rouss\[G]eau,
Y'a des \[Em]soleils de \[D]feu cachés dans les ro\[G]seaux,
Comme dans \[Em]les tabl\[Am]eaux du Douanier \[D]Rouss\[G]eau,
Lalala, lalalala\[D] Lalala, lalalala
\[G]oho, \[Am]oho, \[Bm]oho, \[C]oh.
\[D]Vive le Douanier Rouss\[G]eau !
\endverse
\endsong


\beginsong{Et l'on y peut rien}[by={Jean-Jacques Goldman},cr={1982}]
\beginverse
\[G] Comme un \[C]fil en\[D]tre l’autre et l’\[G]un.\[C] Invi\[G]sible, il \[Em]pose \[D]ses liens
\[G] Dans les \[C]méandres \[D]des inconsc\[G]ients\[C] il se \[G]promène im\[D]puné\[G]ment
Et \[C]tout un peu \[G]tremble et le \[C]reste s’é\[G]teint
Jus\[C]te dans nos \[G]ventres un \[Em]nœud, une \[D]faim
Il \[C]fait roi l’es\[G]clave et peut \[C]damner les \[G]saints
L’honn\[C]ête ou le \[Em]sage et l’on \[C]n’y \[D]peut \[G]rien
\endverse

\beginverse
Et l’on résiste on bâtit des murs. Des bonheurs, photos bien rangées
Terroriste, il fend les armures, un instant tout est balayé
Tu rampes et tu guettes et tu mendies des mots
Tu lis ses poètes aime ses tableaux
Et tu cherches à la croiser t’as quinze ans soudain
Tout change de base et l’on n’y peut rien
\[G]\[C]\[D]\[G]\[C]\[G]\[D]
\endverse

\beginverse
\[A] Il s’in\[D]vite quand \[E]on l’attend \[A]pas\[D] quand on \[A]y croit, il \[F#m]s’enfuit dé\[E]jà
\[A] Frère \[D]qui un \[E]jour y goû\[A]ta\[D] jamais \[A]plus tu ne \[E]guéri\[A]ras
Il \[D]nous laisse \[A]vide et plus \[D]mort que vi\[A]vant
C’est \[D]lui qui dé\[]Acide on ne \[F#m]fait que \[E]semblant
Lui, \[D]choisit ses \[A]tours et ses \[D]va et ses \[A]vient
Ain\[D]si fait l’am\[F#m]our et l’on \[D]n’y \[E]peut \[A]rien
\endverse
\endsong

%%%%%%%%Au bout de mes reves

\beginsong{Au Bout de mes Rêves}[by={Jean-Jacques Goldman},cr={1982}]
\beginverse
\[E] Et même si le temps presse \[A]\[B7],\[E] même s'il est un peu court \[A]\[B7]
\[E] Si les années qu'on me laisse \[A]\[B7],\[E] ne sont que minutes et jours \[A]\[B7]
Et même si l'on m'arrête, ou s'il faut briser des murs
En soufflant dans les trompettes ou à force de murmures
\endverse

\beginchorus
\[A] J'irai au \[B7]bout de mes \[E]rêves,\[A] tout au \[B7]bout de mes \[E]rêves
\[A] J'irai au \[B7]bout de mes \[C#m]rêves, où la raison s'a\[C7]chè\[A]ve
Tout au \[B7]bout de mes rêves \[E]
\[A] J'irai au \[B7]bout de mes \[E]rêves,\[A] tout au \[B7]bout de \[C#m]mes rêves
Où la raison \[G#m]s'a\[A]chève, tout au \[B7]bout de mes rêves \[E]
\endchorus

\beginverse
Et même s'il faut partir, changer de terre ou de trace
S'il faut chercher dans l'exile, l'empreinte de mon espace

Et même si les tempêtes, les dieux mauvais les courants
Nous feront courber la tête, plier les genoux sous le vent
\endverse

\beginchorus
\[A] J'irai au \[B7]bout de mes \[C]rêves \[D] \[G] \[E7]
...
\endchorus

\beginverse
Et même si tu me laisses, au creux d'un mauvais détour
En ces moments où l'on teste, la force de nos amours

Je garderai la blessure au fond de moi tout au fond
Mais au dessus je te jure, que j'effacerai ton nom
\endverse

\beginchorus
\[A] J'irai au \[B7]bout de mes \[C]rêves \[D] \[G] \[E7]
... 
\endchorus

\endsong


\beginsong{Je marche seul}[by={Jean-Jacques Goldman},cr=1988]
\beginverse
\[E&] Comme \[Cm]un bateau dérive\[A&] sans \[Cm]but et sans mobile \[A&]
\[E&] Je \[Cm]marche dans la ville\[A&] tout \[Cm]seul et anonyme \[A&]
\[E&] La ville\[Cm] et \[E&]ses \[Cm]pièges ce \[A&]sont mes privi\[E&]lèges
\[Cm]Je suis ri\[E&]che de \[Cm]ça mais ça \[A&]ne s’achète \[E&]pas
Et j’m’en \[C]fous, j’m’en \[E&]fous de \[Am]tout
De ces chaî\[F]nes qui pendent à nos cous \[C]
J’m’enfuis,\[G] j’oublie \[Am]
Je m’offre \[F]une parenthèse, un sur\[C]sis \[G]
\endverse

\beginchorus
\[A7/9]Je marche \[D]seul\[A] \[Em] dans les \[G]rues qui \[A]se \[D]donnent \[A]
\[Dm] Et la \[G]nuit me \[A]par\[D]donne, je \[A]marche \[Em]seul \[A]
\[G] En oubliant les \[A4]heures,
Je \[A]marche \[D]seul \[A]
\[Em] Sans té\[G]moin, sans \[A]per\[D]sonne \[A]
\[Em] Que mes \[G]pas qui \[A]ré\[D]sonnent, je \[A]marche \[Em]seul \[A]
\[G] Acteur \[A]et voyeur \[D]
\endchorus

\beginverse
Se rencontrer, séduire, quand la nuit fait des siennes
Promettre sans le dire juste des yeux qui traînent
Oh, quand la vie s’obstine en ces heures assassines
Je suis riche de ça mais ça ne s’achète pas
Et j’m’en fous, j’m’en fous de tout
De ces chaînes qui pendent à nos cous
J’m’enfuis, j’oublie
Je m’offre une parenthèse, un sursis 
\endverse

\beginverse
Je marche seul
Quand ma vie déraisonne
Quand l’envie m’abandonne
Je marche seul
Pour me noyer d’ailleurs
...
\endverse
\endsong



\beginsong{Il changeait la vie}[by={Jean-Jacques Goldman},cr=1988]
\beginverse
\[A] C’é\[D]tait un cordonnier, sans \[F#m7]rien d’particulier
Dans \[Em7]un village dont le nom \[A7]m’a échappé
Qui \[D]faisait des souliers si \[F#m7]jolis, si légers
Que nos \[Em7]vies semblaient un peu moins lourdes \[A7]à porter
Il \[Em7]y mettait du temps, du ta\[F#m7]lent et du cœur
Ain\[D]si passait sa vie au mil\[G]ieu de nos \[Bm7]heures
Et \[Em7]loin des beaux discours, des gran\[A7]des théories
A sa \[B&]tâche, chaque jour, on pou\[C9]vait dire de lui
Il changeait la \[Dm]vie... \[E&]\[Gm7] \[A4] \[A7]
\endverse

\beginverse
C’était un professeur, un simple professeur
Qui pensait que savoir était un grand trésor
Que tous les moins que rien n’avaient pour s’en sortir
Que l’école et le droit qu’a chacun de s’instruire
Il y mettait du temps, du talent et du cœur
Ainsi passait sa vie au milieu de nos heures
Et loin des beaux discours, des grandes théories
A sa tâche chaque jour, on pouvait dire de lui
Il changeait la vie
\endverse

\beginverse
C’était un petit bonhomme, rien qu’un tout petit bonhomme
Malhabile et rêveur, un peu loupé en somme
Se croyait inutile, banni des autres hommes
Il pleurait sur son saxophone
Il y mit tant de temps, de larmes et de douleur
Les rêves de sa vie, les prisons de son cœur
Et loin des beaux discours, des grandes théories
Inspiré jour après jour de son souffle et de ses cris
Il changeait la vie
\endverse

\endsong

\beginsong{A nos actes manqués}[by={Jean-Jacques Goldman},cr={1991}]
\beginverse
\[E]A tous mes \[B]loupés, mes \[A]ratés, mes vrais sol\[E]eils
\[C#m]Tous les che\[G#m]mins qui me sont \[F#7]passés à cô\[B]té
\[S]A tous mes bateaux manqués, \[A]mes mauvais somm\[E]eils
\[F#m]A tous ceux \[E]que je n’ai pas \[A]été
Aux malentendus, aux mensonges, à nos silences
A tous ces moments que j’avais cru partager
Aux phrases qu’on dit trop vite sans qu’on les pense
A celles que je n’ai pas osées
\[A] A nos actes manqués
\endverse

\beginverse
\[E]Aux années p\[B]erdues à tent\[C#m]er de ressem\[E]bler
\[C#m]A tous les \[G#m]murs que je \[F#7]n’aurais pas su bri\[A]ser
\[E]A tout c’que \[B]j’ai pas vu, tout \[A]près, juste à cô\[E]té
T\[F#m]out c’que j’au\[E]rais mieux fait d’ignorer \[A]
\endverse

\beginverse
Au monde, à ses douleurs qui ne me touchent plus
Aux notes, aux solos que je n’ai pas inventés
Tous ces mots que d’autres ont fait rimer qui me tuent
Comme autant d’enfants jamais portés
A nos actes manqués
\endverse

\beginverse
Aux amours échoués de s’être trop aimé
Visages et dentelles croisés juste frôlés
Aux trahisons que j’ai pas vraiment regrettées
Aux vivants qu’il aurait fallu tuer\endverse

\beginverse
A tout ce qui nous arrive enfin, mais trop tard
A tous les masques qu’il aura fallu porter
A nos faiblesses, à nos oublis, nos désespoirs
Aux peurs impossibles à échanger
\endverse
\endsong


\beginsong{Le plat pays}[by={Jacques Brel}, cr={1966}]
\beginverse
\[C]Avec la mer du \[E]Nord pour dernier terrain \[C]vague,
Et des vagues de \[E]dunes pour arrêter les \[C]vagues,
Et de vagues \[Am]rochers que les marées \[Dm]dépassent,
Et qui ont à \[G7]jamais le coeur à marée \[C]basse.
Avec infin\[E]iment de brumes à \[Am]venir
Avec le vent d'\[Dm]ouest écoutez le ten\[G7]ir
Le plat pays qui est le \[C]mien.
\endverse

\beginverse
Avec des cathédrales pour uniques montagnes,
Et de noirs clochers comme mats de cocagne
Ou des diables en pierres décrochent les nuages,
Avec le fil des jours pour unique voyage,
Et des chemins de pluie pour unique bonsoir,
Avec le vent de l'est écoutez le vouloir,
Le plat pays qui est le mien.
\endverse

\beginverse
\[C]Avec un ciel\[E] si bas qu'un canal s'est per\[C]du,
Avec un ciel si \[E]bas qu'il fait l'humilit\[Am]é
Avec un ciel si \[Dm]gris qu'un canal s'est \[C]pendu,
Avec un ciel si \[Dm]gris qu'il faut lui pardon\[E7]ner.
Avec le vent du\[Am] nord qui vient s'éc\[Dm]arteler,
Avec le vent du \[G7]nord écoutez le c\[C]raquer,
Le \[G7]plat pays qui est le \[C]mien.
\endverse

\beginverse
Avec de l'Italie qui descendrait l'Escaut,
Avec Frida la Blonde quand elle devient Margot,
Quand les fils de Novembre nous reviennent en Mai,
Quand la plaine est fumante et tremble sous Juillet,
Quand le vent est au rire quand le vent est au blé,
Quand le vent est sud écoutez le chanter,
Le plat pays qui est le mien.
\endverse
\endsong


\beginsong{Sur la place}[by={Jacques Brel},cr=1954]
\beginverse
\[Am]Sur la place chauffée au soleil
Une fille s'est mise à danser
Elle \[Dm]tourne toujours pareille
Aux dan\[Am]seuses d'\[E]antiquités
Sur la \[Am]ville il fait trop chaud
Hommes et femmes sont assoupis
Et reg\[Dm]ardent par le carreau
Cett\[Am]e fille qui dan\[E]se à midi
\endverse

\beginchorus
\[Am]Ainsi certains \[E]jours paraît
\[Am]Une flamme \[E]à nos yeux
A\[Am] l'égli\[E]se où j'allais
On\[Am] l'appelait le\[E] Bon Dieu
L'amou\[Am]reux l'a\[Dm]ppelle l'a\[Am]mour
Le mendiant \[Dm]la chari\[Am]té
Le sole\[G]il l'app\[G7]elle le \[C]jour
Et \[E]le brave hom\[Am]me la bonté
\endchorus

\beginverse
Sur la place vibrante d'air chaud
Où pas même ne paraît un chien
Ondulante comme un roseau
La fille bondit s'en va s'en vient
Ni guitare ni tambourin
Pour accompagner sa danse
Elle frappe dans ses mains
Pour se donner la cadence
\endverse

\beginverse
Sur la place où tout est tranquille
Une fille s'est mise à chanter
Et son chant plane sur la ville
Hymne d'amour et de bonté
Mais sur la ville il fait trop chaud
Et pour ne point entendre son chant
Les hommes ferment leurs carreaux
Comme une porte entre morts et vivants
\endverse

\beginverse
Sur la place un chien hurle encore
Car la fille s'en est allée
Et comme le chien hurlant la mort
Pleurent les hommes leur destinée 
\endverse
\endsong


\beginsong{Les bonbons}[by={Jacques Brel},cr=1963]
\beginverse
\[C]Je vous ai \[G7]apporté des bon\[C]bons \[G7]
\[C]Parce que les \[G7]fleurs c'est péris\[C]sable \[Cmaj7]
\[Am]Puis les bon\[Em]bons c'est tellement \[Am]bon
Bien que les \[Em]fleurs soient plus présen\[Am]tables
\[Dm]Surtout quand \[G7]elles sont en bou\[C]tons \[G7]
\[C]Mais je vous ai a\[G7]pporté des bon\[C]bons \[G7]
\endverse

\beginverse
\[C]J'espère qu'on pou\[E/B]rra se prom\[Am]ener \[E]
\[Am]Que madame votre \[Em]mère ne dira \[Am]rien \[E]
\[Am]On ira \[Em]voir passer les tra\[Am]ins
A huit \[Em]heures je vous ramèner\[Am]aci
\[Dm]Quel beau di\[G7]manche pour la s\[C]aison \[G7]
\[C]Je vous ai ap\[G7]porté des bon\[C]bons \[G7]
\endverse

\beginverse
Si vous saviez ce que je suis fier
De vous voir pendue à mon bras
Les gens me regardent de travers
Y en a même qui rient derrière moi
Le monde est plein de polissons
Je vous ai apporté des bonbons
\endverse

\beginverse
Oh oui Germaine est moins bien que vous
Oh oui Germaine elle est moins belle
C'est vrai que Germaine a des cheveux roux
C'est vrai que Germaine elle est cruelle
Ça vous avez mille fois raison
Je vous ai apporté des bonbons
\endverse

\beginverse
Et nous voilà sur la Grand' Place
Sur le kiosque on joue Mozart
Mais dites-moi que c'est par hasard
Qu'il y a là votre ami Léon
Si vous voulez que je cède ma place
J'avais apporté des bonbons
\endverse

\beginverse
Mais bonjour mademoiselle Germaine
Je vous ai apporté des bonbons
Parce que les fleurs c'est périssable
Puis les bonbons c'est tellement bon
Bien que les fleurs soient plus présentables... 
\endverse
\endsong

\beginsong{Les vieux}[by={Jacques Brel},cr=1963]
\beginverse
\[C6]Les vieux ne parlent plus ou alors seulement parfois du bout des \[G9]yeux
Même riches ils sont pauvres ls n'ont plus d'illusions et n'ont qu'un cœur pour \[C6]deux
Chez eux ça sent le thym, le propre la lavande et le verbe d'an\[G9]tan
Que l'on vive à Paris, on vit tous en province quand on vit trop long\[C6]temps
Est-ce d'avoir trop ri que leur voix se lézarde quand ils parlent d'\[G9]hier
Et d'avoir trop pleuré que des larmes encore leur perlent aux paupière\[E7]s
Et s'ils tremblent un peu\[Am], est-ce de voir vieillir la pendule d'argent\[Dm]
Qui ronronne au\[Dm7] salon\[G] qui dit oui \[Dm7]qui dit \[G]non, qui dit "je\[Dm7] vous\[E7] attends"

Les vieux ne rêvent plus. Leurs livres s'ensommeillent, leurs pianos sont fermés
Le petit chat est mort, le muscat du dimanche ne les fait plus chanter
Les vieux ne bougent plus, leurs gestes ont trop de rides leur monde est trop petit
Du lit à la fenêtre, puis du lit au fauteuil, et puis du lit au lit
Et s'ils sortent encore bras dessus bras dessous tout habillés de raide
C'est pour suivre au soleil, l'enterrement d'un plus vieux, l'enterrement d'une plus laide
Et le temps d'un sanglot oublier toute une heure la pendule d'argent
Qui ronronne au salon qui dit oui qui dit non, et puis qui les attend

Les vieux ne meurent pas, ils s'endorment un jour et dorment trop longtemps
Ils se tiennent la main, ils ont peur de se perdre et se perdent pourtant
Et l'autre reste là, le meilleur ou le pire, le doux ou le sévère
Cela n'importe pas, celui des deux qui reste se retrouve en enfer
Vous le verrez peut-être, vous la verrez parfois en pluie et en chagrin
Traverser le présent, en s'excusant déjà de n'être pas plus loin
Et fuir devant vous une dernière fois la pendule d'argent
Qui ronronne au salon, qui dit oui qui dit non, qui leur dit "je t'attends"
Qui ronronne au salon, qui dit oui qui dit non et puis qui nous attend
\endverse
\endsong

\beginsong{Voir un ami pleurer}[by={Jacques Brel},cr=1977]
\beginverse
\[D]Bien sûr, il y a les guerres d´\[Bm]Irlande
\[D]Et les peuplades sans mu\[G]sique
\[Em]Bien sûr, tout ce manque de \[A7]tendre
Et il n´y a plus d´Am\[D]érique
Bien sûr, l´argent n´a pas d´odeur
Mais pas d´odeur vous monte au nez
Bien sûr, on marche sur les fleurs
Mais, mais voir un ami pleurer!
\endverse

\beginverse
Bien sûr, il y a nos défaites
Et puis la mort qui est tout au bout
Nos corps inclinent déjà la tête
Étonnés d´être encore debout
Bien sûr, les femmes infidèles
Et les oiseaux assassinés
Bien sûr, nos cœurs perdent leurs ailes
Mais, mais voir un ami pleurer!
\endverse

\beginverse
Bien sûr, ces villes épuisées
Par ces enfants de cinquante ans
Notre impuissance à les aider
Et nos amours qui ont mal aux dents
Bien sûr, le temps qui va trop vite
Ces métro remplis de noyés
La vérité qui nous évite
Mais, mais voir un ami pleurer!
\endverse

\beginverse
Bien sûr, nos miroirs sont intègres
Ni le courage d´être juif
Ni l´élégance d´être nègre
On se croit mèche, on n´est que suif
Et tous ces hommes qui sont nos frères
Tellement qu´on n´est plus étonné
Que, par amour, ils nous lacèrent
Mais, mais voir un ami pleurer!
\endverse
\endsong

\beginsong{Ces gens-là}[by={Jacques Brel},cr=1965]
\beginverse
D’abord, d’abord, y a l’\[Am]aîné, lui qui est comme\[Em] un \[Am]melon
Lui qui a un gros nez lui qui sait plus son nom
Monsieur tellement qu´y boit, tellement qu´il a bu
Qui fait rien de ses dix doigts, mais lui qui n´en peut plus
Lui qui est complètement cuit, et qui s´prend pour le roi
Qui se saoule toutes les nuits, avec du mauvais vin
Mais qu´on retrouve matin, dans l´église qui roupille
Raide comme une saillie, blanc comme un cierge de Pâques
Et puis qui balbutie, et qui a l´œil qui divague
Faut vous dire, Monsieur, que chez ces gens-là
On ne pense pas, Monsieur, on ne pense pas, on prie
\endverse

\beginverse
Et puis, y a l´autre, des carottes dans les cheveux
Qu´a jamais vu un peigne, qu´est méchant comme une teigne
Même qu´il donnerait sa chemise, à des pauvres gens heureux
Qui a marié la Denise, une fille de la ville
Enfin d´une autre ville, et que c´est pas fini
Qui fait ses p´tites affaires, avec son p´tit chapeau
Avec son p´tit manteau, avec sa p´tite auto
Qu´aimerait bien avoir l´air, mais qui a pas l´air du tout
Faut pas jouer les riches, quand on n´a pas le sou
Faut vous dire, Monsieur, que chez ces gens-là
On n´vit pas, Monsieur, on n´vit pas, on triche
\endverse

\beginverse
Et puis, il y a les autres: la mère qui ne dit rien
Ou bien n´importe quoi et du soir au matin
Sous sa belle gueule d´apôtre et dans son cadre en bois
Y a la moustache du père qui est mort d´une glissade
Et qui r´garde son troupeau bouffer la soupe froide
Et ça fait des grands slurp. Et ça fait des grands slurp
Et puis y a la toute vieille qu´en finit pas d´vibrer
Et qu´on attend qu´elle crève vu qu´c´est elle qu´a l´oseille
Et qu´on n´écoute même pas c´que ses pauvres mains racontent
Faut vous dire, Monsieur, que chez ces gens-là
On n´cause pas, Monsieur, on n´cause pas, on compte
\endverse

\beginverse
Et puis et puis, et puis il y a Frida
Qui est belle comme \[G7]un soleil\[C] et qui m´aime\[G7] par\[C]eil
Que moi j´aime \[G7]Frid\[C]a même qu´on se dit\[G7] sou\[Am]vent
Qu´on aura une \[Em]mai\[Am]son avec des tas de fenêtres
Avec presque pas de murs et qu´on vivra dedans
Et qu´il fera bon y être et que si c´est pas sûr
C´est quand même peut-être parce que les autres veulent pas
Parce que les autres veulent pas. Les autres ils disent comme ça 
Qu´elle est trop \[G7]belle pour \[C]moi que je suis tout \[G7]juste \[C]bon 
À égorger \[G7]les \[C]chats, j´ai jamais \[G7]tué de \[C]chats
Ou alors y a \[G7]longtem\[Am]ps ou bien j´ai oublié
Ou ils sentaient pas bon enfin ils ne veulent pas
\endverse

\beginverse
Parfois quand on se voit semblant que c´est pas exprès
Avec ses yeux mouillants, elle dit qu´elle partira
Elle dit qu´elle me suivra, alors pour un instant, 
Pour un instant seulement ,alors moi je la crois
Monsieur, pour un instant, pour un instant seulement
Parce que chez ces gens-là, Monsieur, on ne s´en va pas
On ne s´en va pas, Monsieur. On ne s´en va pas.
Mais il est tard, Monsieur. Il faut que je rentre chez moi
\endverse
\endsong


\beginsong{Quand on a que l'amour}[by={Jacques Brel},cr=1956]
\beginverse
\[Cm]\[Am]\[Dm]\[G7]
\[C]Quand on a\[Am] que l'amour
\[C]A s'offrir\[Am] en partage
\[Dm]Au jour du \[G7]grand voyage
Qu'est notre grand amour
\endverse

\beginverse
Quand on n'a que l'amour
Mon amour toi et moi
Pour qu'éclatent de joie
Chaque heure et \[C]chaque jo\[E]ur
\endverse

\beginverse
\[E]Quand on n'a\[Am] que l'amour
\[E]Pour vivre\[Am] nos promesses
\[Dm]Sans nulle au\[G7]tre richesse
Que d'y croire \[C]toujours\[E]
\endverse

\beginverse
Quand on n'a que l'amour
Pour meubler de merveilles
Et couvrir de soleil
\[Dm]La laideur \[E]des faubourgs
\endverse

\beginverse
\[Am]Quand on n'a que l'amour
\[Am]Pour unique raison
\[Dm]Pour unique \[G7]chanson
Et unique \[C]secou\[E]rs
\endverse

\beginverse
\[C]Quand on n'a\[Am] que l'amour
\[C]Pour habiller\[Am] matin
\[Dm]Pauvres et \[G7]malandrins
De manteaux de velours
\endverse

\beginverse
Quand on n'a que l'amour
A offrir en prière
Pour les maux de la terre
En simple \[C]troubad\[E]our
\endverse

\beginverse
\[E]Quand on n'a\[Am] que l'amour
\[E]A offrir \[Am]à ceux-là
\[Dm]Dont l'unique \[G7]combat
Est de chercher \[C]le jour\[E]
\endverse

\beginverse
Quand on n'a que l'amour
Pour tracer un chemin
Et forcer le destin
A chaque carrefour
\endverse

\beginverse
\[Am]Quand on n'a que l'amour
\[Am]Pour parler aux canons
\[Dm]Et rien qu'une \[G7]chanson
Pour convaincre \[C]un tamb\[E]our
\endverse

\beginverse
\[Am]Alors sans\[E] avoir rien
\[Am]Que la force\[E] d'aimer
\[Am]Nous \[Dm]aurons dans nos mains
\[G7]Amis le monde en\[C]tier 
\[Am]\[Dm]\[G7]
\endverse
\endsong

\beginsong{La colline aux Coraline}[by={Jean-Michel Caradec},cr=1974]
\beginverse
\[C]Deux petites flaques, \[C]un oiseau qui boite
\[G7]Sur le chemin don\[C]nons-nous la main
Sautons la barrière dans les fougères
Cherchons les fleurs de l’accroche-cœur
\endverse

\beginchorus
\[F]De Caroline à \[G7]Madeline, \[C]Christophe ou Li\[Am]son
\[F]Sur la colline \[G7]aux Corallines \[C]chantent \[G7]cette chan\[C]son\[Em]\[D#dim]\[G7/D]
\endchorus

\beginverse
Si les paroles sont un peu folles
C’est que les enfants inventent tout le temps
Chasse l’autruche à cache-cache elle truche
Quatre moutons fument sur le balcon
\endverse

\beginverse
Changeons le monde une seconde
Ça fera pas de mal au règne animal
C’est la baleine qui fera la laine
Et le chasseur qui aura peur

Le joueur de flûte a fait la culbute
Son pantalon n’a plus de fond
La nuit qui tombe fait grandir les ombres
Il faut rentrer maman va s’inquiéter
\endverse
\endsong


\beginsong{Place des grands hommes}[by={Patrick Bruel},cr={1989}]
\beginchorus
\[B&]On s'était dit rendez-vous\[Cm7] dans 10 ans
\[B&] Même jour, même heure, même \[D]pommes
\[Gm] On verra quand on au\[Dm]ra 30 ans
\[E&] Sur les marches de la place des grands \[F7]hommes
\endchorus

\beginverse
\[B&]Le jour est venu et moi\[Cm7] aussi
\[B&] Mais j' veux pas être le premier\[E&]
\[Gm ]Si on avait plus rien à se di\[Dm]re et si et si...
\[E&] Je fais des détours dans le quartier.\[F]
\[E&] C'est fou c'qu'un crépuscule\[F] de printemps
\[B&] Rappelle le même crépuscule\[E&] d'y a 10 ans
\[Cm] Trottoirs usés par les re\[B&]gards baissés.
\[F7] Qu'est-ce que j'ai fais de ces ann\[F]ées ?
\[E&] J'ai pas flotté tranquille\[F] sur l'eau,
\[B&] Je n'ai pas nagé le vent\[E&] dans le dos.
\[Cm] Dernière ligne droite, la rue\[B&] Souflot,
Combien seront \[Cm7]là 4, 3, 2, 1... 0 ?
\endverse

\beginchorus
On s'était dit rendez-vous dans 10 ans
Même jour, même heure, même pommes
On verra quand on aura 30 ans
Sur les marches de la place des grands hommes
\endchorus

\beginverse
J'avais eu si souvent envie d'elle.
La belle Séverine me regardera-t-elle ?
Eric voulait explorer le subconscient.
Remonte-t-il à la surface de temps en temps ?
J'ai un peu peur de traverser l' miroir.
Si j'y allais pas... J' me serais trompé d'un soir.
Devant une vitrine d'antiquités,
J'imagine les retrouvailles de l'amitié.
"T'as pas changé, qu'est-ce que tu deviens ?
Tu t'es mariée, t'as trois gamins.
T'as réussi, tu fais médecin ?
Et toi Pascale, tu t' marres toujours pour rien ?"
\endverse

\beginchorus
On s'était dit rendez-vous dans 10 ans
Même jour, même heure, même pommes
On verra quand on aura 30 ans
Sur les marches de la place des grands hommes
\endchorus

\beginverse
J'ai connu des marées hautes et des marées basses,
Comme vous, comme vous, comme vous.
J'ai rencontré des tempêtes et des bourrasques,
Comme vous, comme vous, comme vous.
Chaque amour mort à une nouvelle a fait place,
Et vous, et vous...et vous ?
Et toi Marco qui ambitionnait simplement d'être heureux dans la vie,
As-tu réussi ton pari ?
Et toi François, et toi Laurence, et toi Marion,
Et toi Gégé...et toi Bruno, et toi Evelyne ?
Et bah c'est formidable les copains !
On s'est tout dit, on s'serre la main !
On peut pas mettre 10 ans sur table
Comme on étale ses lettres au Scrabble.
\endverse

\beginverse
\[B&] Dans la vitrine je vois\[Cm7] le reflet
\[B&] D'une lycéenne derrière moi. \[E&]
\[Gm] Elle part à gauche, je la suivrai. \[Dm]
\[E&] Si c'est à droite... Attendez-moi ! \[B&]
\[Cm7] Attendez-moi ! Attendez-moi ! Attendez-moi !\[E&]
\endverse

\beginchorus
On s'était dit rendez-vous dans 10 ans,
Même jour, même heure, même pommes
On verra quand on aura 30 ans
Si on est d'venus des grands hommes...
Des grands hommes... des grands hommes...

\[B&] Tiens si on s' donnait rendez-vous dans 10 ans...
\endchorus
\endsong

\beginsong{Colchiques dans les prés}[by={Francine Cockenpot},cr=1945]
\beginverse
Col\[Dm]chiques dans les près, fleurissent, fleurissent,
Col\[Gm]chiques dans les près : c'est \[Dm]la fin de l'été.
\endverse

\beginchorus
Les feuilles d'automne em\[Am]portées par le vent
En \[F]ronde mono\[G]tone tombent \[Dm]en tourbillonnant.
\endchorus

\beginverse
Châtaignes dans les bois, se fendent, se fendent,
Châtaignes dans les bois, se fendent sous les pas.
\endverse

\beginverse
Nuages dans le ciel s'étirent, s'étirent,
Nuages dans le ciel s'étirent comme une aile.
\endverse

\beginverse
Et ce chant dans mon coeur murmure, murmure,
Et ce chant dans mon coeur appelle le bonheur.
\endverse
\endsong

\beginsong{Sarbacane}[by={Francis Cabrel},cr=1989]
\beginverse
\[C]On croyait savoir \[F]tout sur l'amour depuis \[C]toujours,
Nos corps par cœur et nos cœurs au chaud dans le \[G7]velours,
\[C]Et puis te voilà \[C7]bout de femme, comme \[F7]soufflée d'une sarbacane.
\[C]La ciel a même un \[G7]autre éclat \[C]depuis toi.
Les hommes poursuivent ce temps qui court depuis toujours,
Voilà que t’arrives et que tout s'éclaire sur mon parcours,
Pendue à mon cou comme une liane, comme le roseau de la sarbacane.
Le ciel s'est ouvert par endroits depuis toi.

\[F7]Pas besoin de phrases ni de longs\[C] discours,
\[F7]Ça change tout ded\[E&]ans, ça change tout \[G]autour.\[G7]\[C]
Finis les matins \[C7]paupières en panne,
\[F7]Lourdes comme les bouteilles de butane,
\[C]J'ai presque plus ma\[G7] tête à moi,
\[C]Depuis toi.
\[C]\[F7]\[C]\[G7]\[F7]\[C]\[G7]\[C]
\endverse

\beginverse
\[F7]Pas besoin de faire de trop longs disco\[C]urs,
\[F7]Ça change tout dedans\[E&], ça change tout autour,\[G7]
\[C]Pourvu que jamais \[C7]tu ne t'éloignes,
\[F7]Plus loin qu'un jet de sarbacane,
\[C]J'ai presque plus ma tête à moi,
\[C]Depuis toi.
\endverse

\beginverse
\[C]Alors te voilà \[C7]bout de femme,
Comme \[F]soufflée d'une \[F7]sarbacane.
\[C]Le ciel s'est ouvert \[G7]par endroits,
D\[C]epuis toi\[C7]. \[C] Oh depuis toi
\endverse
\endsong

\beginsong{L'encre de tes yeux}[by={Francis Cabrel},cr=1985]
\beginverse
\[G]Puisqu'on ne \[D]vivra jamais \[Em]tout les deux
\[C]Puisqu'on est fou, puisqu'on est seul\[G]
Puisqu'ils sont si \[D]nombreux
\[G]Même la \[D]morale parle \[Em]pour eux
\[C]J'aimerais quand même te dire\[G]
Tout ce que j'ai pu écrire\[D]
Je l'ai puisé à l'encre de tes\[C] yeux
\endverse

\beginverse
Je n'avais pas vu que tu portais des chaînes
A trop vouloir te regarder
J'en oubliais les miennes
On rêvait de Venise et de liberté
J'aimerais quand même te dire
Tout ce que j'ai pu écrire
C'est ton sourire qui me l'a dicté
\endverse

\beginverse
Tu viendras longtemps marcher dans mes rêves
Tu viendras toujours du côté
Où le soleil se lève
Et si malgré ça j'arrive à t'oublier
J'aimerais quand même te dire
Tout ce que j'ai pu écrire
Aura longtemps le parfum des regrets
\endverse

\beginverse
Mais puisqu'on ne vivra jamais tous les deux
Puisqu'on est fou, puisqu'on est seul
Puisqu'ils sont si nombreux
Même la morale parle pour eux
J'aimerais quand même te dire
Tout ce que j'ai pu écrire
Je l'ai puisé à l'encre de tes yeux
\endverse
\endsong

\beginsong{Encore et encore}[by={Francis Cabrel},cr=1985]
\beginverse
\[Em]D'abord vos corps qui se séparent
T'es seule dans la lumière des phares
\[G]T'entends à chaque fois que tu respires
Comme un bout de tissu qui se déchire
\[C]Et ça continue encore et encore
\[Am]C'est que le début d'accord, d'accord...\[Em]\[Em7]\[Em]\[Em7]
L'instant d'après le vent se déchaîne
Les heures s'allongent comme des semaines
Tu te retrouves seule assise par terre
À bondir à chaque bruit de portière
Et ça continue encore et encore
C'est que le début d'accord, d'accord...
\endverse

\beginchorus
\[G]Quelque chose vient de tomber
\[D]Sur les lames de ton plancher
\[Em]C'est toujours le même film qui \[C]passe
\[G]T'es toute seule au fond de l'espace
\[B]T'as personne devant\[Am]...
\endchorus

\beginverse
La même nuit que la nuit d'avant
Les mêmes endroits deux fois trop grands
T'avances comme dans des couloirs
Tu t'arranges pour éviter les miroirs
Mais ça continue encore et encore
C'est que le début d'accord, d'accord...
\endverse

\beginverse
\[G]Faudrait que t'arrives à en \[D]parler au passé
\[Em]Faudrait que t'arrives à ne \[C]plus penser à ça
\[G]Faudrait que tu l'oublies à \[D]longueur de journée\[Em]
\[G]Dis-toi qu'il est de l'a\[D]utre côté du pôle\[Em]
Dis-toi surtout qu'\[C]il ne reviendra pas
\[G]Et ça fait marrer les \[D]oiseaux qui s'envol\[C2]ent
Les oiseaux qui s'env\[D]olent
Les oiseaux qui s'envolent\[D4]\[D]\[D4]\[D]
\endverse

\beginverse
Tu comptes les chances qu'il te reste
Un peu de son parfum sur ta veste
Tu avais dû confondre les lumières
D'une étoile et d'un réverbère
Mais ça continue encore et encore
C'est que le début d'accord, d'accord...
\endverse

\beginchorus
Y a des couples qui se défont
Sur les lames de ton plafond
C'est toujours le même film qui passe
T'es toute seule au fond de l'espace
T'as personne devant...personne
\endchorus

\beginverse
Quelque chose vient de tomber
Sur les lames de ton plancher
C'est toujours le même film qui passe
T'es toute seule au fond de l'espace
T'as personne devant...personne...
Y a des couples qui se défont
C'est toujours le même film qui passe
Le même film qui passe
\endverse
\endsong

\beginsong{Je l'aime à mourir}[by={Francis Cabrel},cr=1979]
\beginverse
Moi \[C]je n'étais rien et voi\[C]là qu'aujourd'hui
Je \[C/B]suis le gardien du so\[C/B]mmeil de ses nuits
Je l'aime à \[Am]mourir vous \[Dm]pouvez détruire
Tout ce \[Dm]qu'il vous plaira elle \[F]n'a qu'à ouvrir
L'espace\[G] de ses bras pour tout\[C] reconstruire
Pour tout \[C/B]reconstruire je l'aime à \[Am]mourir
Elle a gommé les chiffres des horloges du quartier
Elle a fait de ma vie des cocottes en papier
Des éclats de rire elle a bâti des ponts
Entre nous et le ciel et nous les traversons
À chaque fois qu'elle ne veut pas dormir
Ne veut pas dormir je l'aime à mourir
\endverse

\beginchorus
Elle a dû \[E]faire toutes les \[Am]guerres
\[G]Pour être si forte \[C]aujourd'hui
Elle a dû \[E]faire toutes les \[Am]guerres
\[B&sus2]De la vie et l'\[C]amour aussi
\endchorus

\beginverse
Elle vit de son mieux son rêve d'opaline
Elle danse au milieu des forêts qu'elle dessine
Je l'aime à mourir elle porte des rubans
Qu'elle laisse s'envoler elle me chante souvent
Que j'ai tort d'essayer de les retenir
De les retenir je l'aime à mourir
Pour monter dans sa grotte cachée sous les toits
Je dois clouer des notes à mes sabots de bois
Je l'aime à mourir je dois juste m'asseoir
Je ne dois pas parler je ne dois rien vouloir
Je dois juste essayer de lui appartenir
De lui appartenir je l'aime à mourir
\endverse

\beginverse
Moi je n'étais rien et voilà qu'aujourd'hui
Je suis le gardien du sommeil de ses nuits
Je l'aime à mourir vous pouvez détruire
Tout ce qu'il vous plaira elle n'aura qu'à ouvrir
L'espace de ses bras pour tout reconstruire
Pour tout reconstruire je l'aime à mourir
\endverse
\endsong

\beginsong{La cabane du pêcheur}[by={Francis Cabrel},cr={1994}]
\beginverse
\[D]Le soir tombait \[G]de tout son poids au dessus de la \[D]rivi\[G]ère
\[Bm]Je rangeais mes cannes \[A]on ne voyait plus que du \[G]feu
\[D]Je l'ai vu s'approcher la\[G] tête ailleurs dans ses pri\[D]ères\[G]
\[Bm]Il m'a semblé voir \[A]trop briller ses y\[G]eux\[D]\[G]
\[G]Je lui ai \[D]dit \[G]
\[G]Si \[D]tu pleures pour un \[G]garçon tu seras pas la derni\[D]ère \[G]
\[Bm]Souvent, les poissons \[A]sont bien plus affect\[G]ueux
\[D]Va faire un petit \[G]tour, respire le grand air!\[D]\[G]
Ap\[Bm]rès, je te parlerai de l'am\[A]our si je me souviens un \[G]peu\[D]\[G]
\endverse
\beginchorus
\[G]Elle m'a dit
\[Em]Elle a dit justement c'est ce que je \[A]voudrais savoir
Et j'ai\[G] dit viens t'asseoir dans \[Bm]la cabane du pêch\[C]eur
\[C]C'est un mauvais \[G]rêve, oublie-\[D]le!
\[Em]Tes rêves sont toujours trop clairs \[A]ou trop noirs
Alors, \[G]viens faire toi-même le \[Em]mélange des couleurs
\[G]Sur les murs de la cabane du pêch\[Em]eur, \[G]viens t'asseoir
\endchorus

\beginverse
Je lui ai dit
Le monde est pourtant pas si loin, on voit les lumières
Et la terre peut faire tous les bruits qu'elle veut
Y a sûrement quelqu'un qui écoute là-haut dans l'univers
Peut-être qu'tu demandes plus qu'il ne peut?
\endverse

\beginchorus
Oh, elle m'a dit
Elle a dit justement c'est ce que je voudrais savoir
Et j'ai dit viens t'asseoir dans la cabane du pêcheur
C'est un mauvais rêve, oublie-le!
Tes rêves sont toujours trop clairs ou trop noirs
Alors, viens faire toi-même le mélange des couleurs
Sur les murs de la cabane du pêcheur viens t'asseoir
\endchorus

\beginverse
Elle m'a dit
Elle a dit finalement, je brûle de tout savoir
Et j'ai dit viens t'asseoir dans la cabane du pêcheur
Y a sûrement de la place pour deux!
Cette route ne mène nulle part
Alors, viens faire toi-même le mélange des couleurs
Sur les murs de la cabane du pêcheur
On va comparer nos malheurs là, dans la cabane du pêcheur
Partager un peu de chaleur là, dans la cabane du pêcheur
Moi, j'attends que le monde soit meilleur là, dans la cabane du pêcheur
\endverse
\endsong

\beginsong{J'ai la guitarre qui me démange}[by={Yves Duteil},cr={1979}]
\beginverse
J'\[B&]ai la guitare \[F7]qui me démange,\[B&]alors je gratte un p'tit \[F7]peu.
\[B&]Ça me soulage \[F7]et ça s'arrange\[B&]mais ça fait pas très sé\[F7]rieux.
\[E&]Pardonnez-moi, c'est \[Bb/D]très étrange,\[F7]ça me prend là où ça \[D7]veut.
\[E&]C'est la guitare \[Bb/F]qui me démange,\[F]alors je gratte un p'tit \[B&]peu.\[C7]\[F7]
J'aurais pu, \[B&]c'est hé\[F7]réditaire,\[Bb/D] être offic\[F7]ier d'Etat maj\[B&]or,
Archevê\[B&]que ou vé\[F7]térinaire,\[Bb/D] clerc de not\[F]aire ou \[C7]chercheur d'\[F7]or.
Le has\[Bb7]ard et la génétique\[E&] en ont voulu \[Cm7]tout autrement.\[Bb/F]
J'ai mis les d\[Eb6]oigts dans la musique,\[Bb/F] et \[C7]c'est ainsi qu'à présent\[F7]
J'ai la guitare qui me démange, alors je gratte un p'tit peu.
Ça me soulage et ça s'arrange mais ça fait pas très sérieux.
Dans l'industrie, l'électronique, le commerce et les assurances,
J'avais des dons pour la pratique, oui, mais côté références...
\endverse

\beginverse
J'ai la guitare qui me démange, alors je gratte un p'tit peu.
Ça fait du bien dans les phalanges, mais ça fait pas très sérieux.
Y a rien à faire pour que ça change, faut se faire une raison.
J'ai la guitare qui me démange, alors j'écris des chansons.
J'ai appris à lire, à écrire et je compte sur mes dix doigts
Pour composer de doux délires à partir de n'importe quoi.
Ne croyez pas que je m'amuse, que je cours après les honneurs.
Si je taquine un peu la muse, c'est pas pour les droits d'auteur.
C'est la guitare qui me démange, alors je gratte un p'tit peu.
Ça me soulage et ça s'arrange, mais au bout d'une heure ou deux,
Quand je me prends pour un artiste, ça donne un résultat miteux,
Ça me rend profondément triste, et quand je suis malheureux...
\endverse

\beginverse
J'ai la guitare qui me démange, alors je gratte un p'tit peu.
Ça me soulage et ça s'arrange, mais c'est un cercle vicieux.
Y a rien à faire pour que ça change, faut se faire à cette idée.
J'ai la guitare qui me démange, alors j'essaie de chanter.
J'ai consulté un spécialiste pour me guérir, mais sans succès.
Il m'a dit "Si le mal persiste, essayez de prendre un cachet."
Avant même que je le comprenne, j'étais déjà dev'nu chanteur,
Et c'est pour ça que sur la scène, entre les deux projecteurs...
J'ai la guitare qui me démange, alors je gratte un p'tit peu.
Ça me soulage et ça s'arrange et si c'est pas très sérieux,
C'est la plus belle leçon d'musique que j'ai reçue depuis toujours;
C'est la meilleure thérapeutique quand j'ai des chagrins d'amour.
\endverse

\beginverse
J'ai la guitare qui me démange, alors je gratte un p'tit peu.
Ça me soulage et ça s'arrange et quand je serai très vieux,
À ma mort, je veux qu'on m'installe avec ma guitare à la main.
Si vous voyez ma pierre tombale qui gigote à la Toussaint...
C'est la guitare qui me démange, alors je gratte un p'tit peu
Dans les nuages avec les anges et tout là-haut dans les cieux.
Pardonnez-moi si ça dérange : ça me prend là où ça veut.
C'est la guitare qui me démange, alors je gratte un p'tit peu.
Y a rien à faire pour que ça change, et si, dans un jour ou deux,
Y a la guitare qui vous démange, alors c'était contagieux.
\endverse
\endsong


%%%%%%%%%%%%%%%%%%%%%%%%%%%% Ça plane pour moi
\beginsong{Ca plane pour moi}[by={Plastic Bertrand},cr=1977]
\beginverse
\[B&]Wham! Bam!
Mon chat Splash gît sur mon lit
A bouffé sa langue en buvant dans mon whisky
Quant à moi, peu dor\[E&]mi, vidé, brimé
J'ai dû dormir dans la gou\[B&]ttière où j'ai eu un flash
Ouh-ouh-ouh-ouh\[F], en quatre cou\[B&]leurs
\[B&]Allez hop, un matin
Une louloute est venue chez moi
Poupée de cellophane, cheveux chinois
Un sparadrap, une gueule de bois\[E&]
A bu ma bière dans un grand \[B&]verre en caoutchouc
Ouh-ouh-ouh-ouh\[F]
Comme un Indien dans son igloo\[B&]
\endverse

\beginchorus
Ça plane pour moi, ça plane pour \[B&]moi
Ça plane pour \[E&]moi, moi, moi, moi, moi
Ça plane pour \[B&]moi
Ouh-ouh-ouh-ouh\[F]
Ça plane pour \[B&]moi
\endchorus

\beginverse
Allez hop, la nana, quel panard quelle vibration
De s'envoyer sur le paillasson
Limée, ruinée, vidée, comblée
"You are the King of the divan"
Qu'elle me dit en passant
Ouh-ouh-ouh-ouh, I am the King of the divan
\endverse

\beginverse
Allez hop t'occupe, t'inquiète
Touche pas ma planète
It's not today quel le ciel me tombera sur la tête\[E&]
Et que la colle me manquera\[B&]
Ouh-ouh-ouh-ouh\[F]
Ça plane pour \[B&]moi
\endverse

\beginverse
Allez hop
Ma nana s'est tirée, s'est barrée
Enfin c'est marre, a tout cassé
L'évier, le bar, me laissant seul
Comme un grand connard
Ouh-ouh-ouh-ouh
Le pied dans le plat
\endverse
\endsong


\beginsong{Plus près des étoiles}[by={Gold},cr=1984]
\beginverse
\[D4]Ils ont quit\[D]té leurs ter\[D2]res
\[D4]Leurs champs de \[D]fleurs et leurs li\[Bm]vres \[B2]sacrés
\[B4]Traversés les\[Bm] rizières\[B2]
\[B4]Jusqu'au grand \[Bm]fleuve\[D] salé\[D2]
\[D4]Sans amour,\[D] sans un cri
Ils ont fer\[D4]mé leurs visa\[D]ges de miel\[Bm]\[B2]
\[B4]Les yeux mou\[B4]illés de pl\[B2]uie
\[B4]Les mains t\[Bm]endues vers \[G]le ciel
\endverse

\beginchorus
\[G]Un peu plus près des é\[A]toiles
Au jar\[Bm]din de lumière et \[Em]d'argent
\[G]Pour oublier\[A] les rivages brû\[Bm]lants
\[G]Un peu plus près des é\[A]toiles
A l'a\[Bm]bri des col\[Em]ères du vent
\[G]A peine un peu \[F#m]plus libres \[A]qu'avant
\endchorus

\beginverse
Au pied des murs de pierres
Ils ont brûlé leurs dragons de papier
Refermés leurs paupières
Sur les chenilles d'acier
Eux qui croyaient vieillir
En regardant grandir leurs enfants
A l'ombre du sourire
Des Bouddhas de marbre blanc
\endverse

\beginverse
Ils parlent à demi-mots
A mi-chemin entre la vie et la mort
Et dans leurs yeux mi-clos
Le soleil, le soleil brille encore
Une île de lumière
Un cerf volant s'est posé sur la mer
Un vent de liberté
Trop loin, trop loin pour les emporter
\endverse
\endsong

\beginsong{Vous les copains je ne vous oublierai jamais}[by={Sheila},cr=1964]
\beginverse
Vous les copains, je n'vous oublierai jamais
\[G]Di doua di di doua di \[C]dam di di \[G]dou
Toute la vie, nous serons toujours des amis
Di doua di di doua di dam di di dou
\[G]Tous ensemble (tous ensemble)
On est bien (on est bien)
Tous ensemble, on est bien
Car on suit le même chemin
\endverse

\beginverse
Quand on se voit, on se tutoie gentiment
Di doua di di doua di dam di di dou
On est sincère, on chante et on danse tout le temps
Di doua di di doua di dam di di dou
Aujourd'hui (aujourd'hui)
Et demain (et demain)
Aujourd'hui et demain
On se tiendra par la main
\endverse

\beginverse
\[G]Woho, wo-\[G7]o
Si \[Em]un jour nous sommes sép\[Em]arés
\[C]Nous, on sait que notre\[D] coeur ne changera \[D7]jamais
\endverse

\beginverse
Si l'un de nous quelquefois a des ennuis
Di doua di di doua di dam di di dou
On est tous là, on se met tous avec lui
Di doua di di doua di dam di di dou
Et nos peines (et nos peines)
Et nos joies (et nos joies)
Et nos peines et nos joies
Sont partagées chaque fois
\endverse

\beginverse
Pas de problèmes, car si quelqu'un nous plaît bien
Di doua di di doua di dam di di dou
Tout simplement on l'adopte on lui dit : "Viens !"
Di doua di di doua di dam di di dou
Car il faut (car il faut)
Des amis (des amis)
Car il faut des amis
Pour être heureux dans la vie
\endverse

\beginverse
Vous les copains, je n'vous oublierai jamais
Di doua di di doua di dam di di dou
Vous les copains, je n'vous oublierai jamais
Di doua di di doua di dam di di dou 
\endverse
\endsong

\beginsong{C'est ma terre}[by={Christophe Maé},cr=2007]
\beginverse
\[F]On oublie un peu \[C]facilement \[G]d’où l’on vient, \[Am]d’où l’on part
Ça nous arrange de perdre de temps en temps la mémoire
Quand il y a danger on regarde son voisin c’est bizarre
Sans voir qu’on l'a peut-être laissé trop longtemps à l’écart
\endverse

\beginchorus
Y a-t-il un cœur qui s’élève pour qu'tout le monde soit d’accord
Un cœur qui prenne la relève quelqu’un qui vienne en renfort
\endchorus

\beginchorus
C’est ma terre où je m’assois
Ma rivière l’eau que je bois
Qu’on y touche pas
C’est mes frères autour de moi
Mes repères et ma seule voie
Qu’on n'y touche pas
\endchorus

\beginverse
On n'alimente nos peurs qu’en détournant
Nos regards
De nos belles valeurs qui ne seraient pourtant
Qu’un devoir
Et Si l’on apprenait à se prendre la main
A se voir
Autrement que des inconnus qui n’font rien
Qu’des histoires
\endverse

\beginchorus
Y a-t-il un un cœur qui s’élève pour qu’tout le monde soit d’accord
Un cœur qui prenne la relève quelqu’un qui vienne en renfort
\textnote{Refrain\rep{3}}
\endchorus

\beginchorus
Y a-t-il un un cœur qui s’élève pour qu’tout le monde soit d’accord
Un cœur qui prenne la relève quelqu’un qui vienne en renfort
\textnote{Refrain\rep{2}}
\endchorus
\endsong


\beginsong{Quand t'es dans le désert}[by={Jean-Patrick Capdevielle},cr=1980]
\beginverse
\[Am]Moi je traîne dans le désert depuis plus de 28 jours
Et d\[Em]éjà quelque mirages me disent de faire demi-tour
\[G]La fée des neiges me suit tapant sur son\[Am] tambour
\[Am]Les fantômes du syndicat des marchands d'incertitudes
Se sont \[Em]glissés jusqu'à ma lune, reprochant mon attitude
\[G]C'est pas très populaire le goût d'\[Am]la solitude.
\endverse

\beginchorus
\[C] Quand t'es dans le \[G]désert\[Am] depuis trop longtemps
\[C] Tu t'demandes à \[G]qui ça sert \[D]toutes les règles un peu truquées
\[D]Du jeu qu'on veut te faire jouer \[D]Les yeux \[Am]bandés.
\endchorus

\beginverse
Tous les rapaces du pouvoir menés par un gros clown sinistre
Foncent vers moi sur la musique d'un piètre accordéoniste,
Je crois pas qu'ils viennent me parler des joies d'la vie d'artiste
D' l'autre coté ,voilà Caïn toujours aussi lunatique
Son œil est rempli de sable et sa bouche pleine de verdicts
Il trône dans un cimetière de vielles pelles mécaniques.
\endverse

\beginverse
Les gens disent que les poètes finissent tous trafiquants d'armes
On est 50 millions de poètes, c'est ça qui doit faire notre charme
Sur une lune de Saturne mon perroquet sonne l'alarme
C'est drôle mais tout le monde s'en fout
Vendredi tombant nulle part, y a Robinson solitaire
Qui m'a dit : " j'trouve plus mon île, vous n'auriez pas vu la mer "
Va falloir que j'lui parle du thermonucléaire.
\endverse

\beginverse
Hier un homme est venu vers moi d'une démarche un peu traînante
Il m'a dit t'as tenu combien de jours, j' lui est répondu : bientôt 30
J' me souviens qu'il espérait tenir jusqu'à 40
Quand j'lui ai demandé son message, il m'a dit d'un air tranquille :
Les politiciens finiront un jour au font d'un asile
J'ai compris que je pourrais bientôt regagner la ville
\endverse
\endsong


%%% la bête est revenue
\beginsong{La bête est revenue}[by={Pierre Perret}]

\beginverse
S\[C]ait-on pourquoi, un matin,
Cette bête s´est réveill\[Am]ée
Au milieu de pantins
Qu´elle a tous émerveillé\[F]s
En proclamant partou\[G]t, haut et for\[C]t :
"N\[B7]ous mettrons l´étranger dehor\[E]s"
\endverse

\beginverse
Puis cette ogresse aguicheuse
Fit des clones imitatifs.
Leurs tirades insidieuses
Convainquirent les naïfs
Qu´en suivant leurs dictats xénophobes,
On chasserait tous les microbes.
\endverse

\beginchorus
Attentio\[Am]n mon ami, je l´ai vue.
Méfie-to\[C]i : la bê\[D]te est reven\[Esus4]ue!  \[E]
C´est une hy\[Am]dre au disco\[G]urs enjôle\[Dm/F]ur
Q\[Am/E]ui forge une nouve\[F]lle r\[Bsus4]ace d´op\[B7]presseur\[Esus4]s. 
\[Am]Y a nos liberté\[Am/E]s sous sa botte.
Am\[F]i, ne lu\[Bm7b5]i ouvre pas ta por\[Esus4]te. \[E]
\endchorus

\beginverse
D´où cette bête a surgi,
Le ventre est encore fécond.
Bertold Brecht nous l´a dit.
Il connaissait la chanson.
Celle-là même qu´Hitler a tant aimée,
C´est la valse des croix gammées
\endverse

\beginverse
Car, pour gagner quelques voix
Des nostalgiques de Pétain,
C´est les juifs, encore une fois,
Que ces dangereux aryens
Brandiront comme un épouvantail
Dans tous leurs sinistres éventails.
\endverse

\beginverse
N´écoutez plus, braves gens,
Ce fléau du genre humain,
L´aboiement écœurant
De cette bête à chagrin
Instillant par ces chants de sirène
La xénophobie et la haine.
\endverse

\beginverse
Laissons le soin aux lessives
De laver plus blanc que blanc.
Les couleurs enjolivent
L´univers si différent.
Refusons d´entrer dans cette ronde
Qui promet le meilleur des mondes.
\endverse

\endsong



\beginsong{La complainte du phoque en Alaska}[by={Beau\ Dommage},cr=1974]
\beginverse
\[C]Cré-moé, cré-moé pas, quéqu' part \[G]en Ala\[E]ska
Y a un \[Dm]phoque qui s'e\[G]nnuie en \[Em]maudit\[A7]
Sa \[Dm]blonde est \[G]partie \[B7]gagner sa \[Em]vie
Dans un \[Am]cirque aux\[D]     Etats-\[G]Unis\[G7]
Le phoque est tout seul, il r'garde le soleil
Qui descend doucement sur le glacier
Il pense aux Etats en pleurant tout bas
C'est comme ça quand ta blonde t'a lâché
\endverse

\beginchorus
\[C]Ça \[G]vaut pas la \[Am]peine de \[Em]laisser ceux qu'\[Em]on aime
Pour al\[Em]ler faire tour\[Dm]ner des ba\[G]llons sur s\[C]on \[G]nez
Ça fait rire les enfants ça dure jamais longtemps
Ça fait plus rire personne quand les enfants sont grands
\[B7]Ou, ou \[Em]ou,\[F#dim]ou, ou ou\[G]\[G7]
\endchorus

\beginverse
Quand le phoque s'ennuie, il r'garde son poil qui brille
Comme les rues de New York après la pluie
Il rêve à Chicago, à Marilyn Monroe
Il voudrait voir sa blonde faire un show
C'est rien qu'une histoire, j'peux pas m'en faire accroire
Mais des fois j'ai l'impression qu'c'est moi
Qui est assis sur la glace les deux mains dans la face
Mon amour est partie puis j'm'ennuie
\endverse
\endsong


%%%%%%%%Chante
\beginsong{Chante}[by={Les\ Forbans}, cr={1982}]
\beginchorus
Chante, chante, danse et mets tes baskets
Chouette, c'est sympa tu verras
Viens, surtout n'oublie pas
Vas-y ramène-toi, oui, tout le monde chez moi
\endchorus

\beginverse
Ce soir c'est la boum dans le living-room
Les parents sont partis faut que tu téléphones
Wouap dou bap pah pah
Surtout fait bien gaffe de n'oublier personne
\endverse

\beginverse
Je voudrais vous voir danser tous comme des fous
Que la musique réveille tout ce qui est en nous
Sylvia, Patricia et Barbara
J'aimerais que ce soir vous soyez toutes là, ouais!
\endverse

\beginverse
Il n'y aura pas d'excuse valable
Tous les absents seront coupables
Vous connaissez la route qui mène chez moi
Surtout n'oubliez pas, non n'oubliez pas
\endverse

\beginverse
Nous ferons ce soir tout ce qui nous plaît
Peut-être boirons-nous autre chose que du lait
Marions-nous ce soir, oui, jusqu'au lendemain
Venez faut qu'on se marre jusqu'au petit matin, ouais!
\endverse
\endsong



\beginsong{Aimer est plus fort qu'être aimé}[by={Daniel Balavoine},cr=1985]
\beginverse
Toi qui \[Am7]sais ce qu'est un rempart
Tu avances sous les regards \[G]courr\[F]oucés
\[G]Tu écris mais sur le buvard
Tous les mots se sont inver\[F]sés
Si tu parles, il te faut savoir
Que ceux qui lancent des regards courroucés
Ne voudrons voir dans leur miroir
Que ce qui peut les arranger
\endverse

\beginverse
Toi qui as \[B&7]brisé la glace
Sais que rien ne remplace la \[F]vérité
\[D] Et qu'il n'y a que deux races
Ou les faux ou les \[G]vrais
\endverse

\beginverse
L'amo\[C]ur te porte dans tes ef\[Am]forts
L'amour \[F]de tout\[Dm] délie les sec\[G]rets
Et \[C]face à tout ce qui te dé\[Am]vore
\[F]Aimer est \[Dm]plus fort que d'être \[G]aimé
\endverse

\beginverse
Toi qui sais ce qu'est le blasphème
On ne récolte pas toujours ce qu'on sème
Tu connais l'ambition suprême
De ceux qui te vouent de la haine

Ils voudraient sous la menace
Te fondre dans la masse pour t'étouffer
Mais pour couler le brise-glace
Il faudrait un rocher

L'amour te porte dans tes efforts
L'amour de tout délie les secrets
Et face à tout ce qui te dévore
Aimer est plus fort que d’être aimé
\endverse
\endsong



\beginsong{Le funambule}[by={Jean-Roger Caussimon},cr=1970]
\beginverse
\[Am]]De tous ses copains du cirque forain
Pas un n'avait dit au vieux funam\[G]bule
Qu'il était aussi parfois som\[F]nambule
Ça n'aurait servi strictement à \[E]rien,
Le public par\[G]ti, la lune deh\[E]ors
À travers les \[Dm]trous de la vieille toile
Allumait un \[Am]ciel tout rempli d'étoiles
Le vieux funam\[E]bule, arrivait alors.
\endverse

\beginverse
Lui qui n'était pas tellement sûr de lui
Qu'avait mal des reins, qu'avait des vertiges
Était tout changé c'était un prodige
Oui c'était vraiment le jour et la nuit.
Plus besoin d'ombrelle ni de balancier
Les sauts périlleux devenaient faciles
Il était gracieux, il était agile
Comme un demi-dieu, sur son fil d'acier.
\endverse

\beginverse
Et ce fut ainsi qu'un enfant le vit
Un enfant puni ou un fils de pauvre
Qui s'était glissé dans l'odeur des fauves
Et le regardait d'un regard ravi.
Spectateur fortuit de ce numéro
L'enfant applaudit à tant de merveilles
Mais un somnambule quand on le réveille
Comme un funambule ça tombe de haut.
\endverse

\beginverse
De tous ses copains du cirque forain
Pas un n'avait dit au vieux funambule
Qu'il était aussi parfois somnambule
Les gens du voyage sont des gens très bien
\endverse
\endsong

\beginsong{Ma préférence}[by={Julien Clerc},cr=1985]
\beginverse
\[C]Je le sais sa \[Gm7]façon d’être à \[A7]moi parfois
\[Dm]Vous déplait. \[Am]Autour d’elle et moi \[G7]le silen\[Cm]ce se fait
\[Cm]Mais elle est \[A&]ma préférence à \[B&]moi…\[B7]
\endverse

\beginchorus
\[G&]Oui, je sais cet \[A&m]air d’indiffé\[B&7]ence qui est
\[E&m]Sa défen\[G&]e vous \[A&m]fait souvent of\[B&7]fense…
Mais \[G&]quand elle est par\[A&m]mi mes amis \[B&7]de faïence
\[E&m]De faïen\[G&]ce je \[A&m]sais sa défail\[B&7]lance…
\endchorus

\beginverse
Je le sais on ne me croit pas fidèle à
Ce qu’elle est et déjà vous parlez d’elle à
L’imparfait mais elle est
Ma préférence à moi…
\endverse

\beginchorus
Il faut le croire moi seul je sais quand elle a froid
Ses regards ne regardent que moi
Par hasard elle aime mon incertitude
Par hasard j’aime sa solitude…
\endchorus

\beginverse*
\[C]\[Gm7]\[A7]
\[Dm]
\[Fm]\[G7]\[Cm]
\[Cm]
\[A&]\[B&]\[B&7]
\endverse

\beginchorus
Il faut le croire moi seul je sais quand elle a froid
Ses regards me regardent que moi
Par hasard elle aime mon incertitude
Par hasard j’aime sa solitude…
\endchorus

\beginverse
Je le sais sa façon d’être à moi, parfois
Vous déplaît autour d'elle et moi le silence se fait
Mais elle est elle est ma chance à moi
Ma préférence à moi. Ma préférence à moi…\rep{3}
\endverse
\endsong


\beginsong{Lily voulait aller danser}[by={Julien Clerc},cr=1982]
\beginverse
Quand To\[C]ny est entré dans le snack-bar
Il devait être au moins minuit moins \[G7]l'quart
Lili la serveuse Semblait très nerveuse
Un dernier client sucrait son \[C]café
À la télé\[C] y avait un match de boxe
Pendant qu'on entendait dans le juke\[G7] box
Gronder le piano de Fats Domino
De quoi vous mettr' des fourmis dans les \[C]pieds
Comme\[F] tous les \[C]soirs
\[D7]Après l'snack \[G7]bar...
\endverse

\beginchorus
Lili \[C]voulait aller danser
Lili voulait aller dan\[G7]ser
Lili voulait aller danser
Aller danser le rock'n \[C]roll
Lili \[C]voulait aller danser
Lili voulait aller dan\[G7]ser
Lili voulait aller danser
Mais Tony trouvait ça moins \[C]drôle
\[F]Lui qui n'aimait pas du\[G7] tout le rock'n\[C] roll
\endchorus

\beginverse
Il a dit : " Lili fais-moi un hot-dog "
Elle a fermé son numéro de Vogue
Elle a obéi et puis elle a dit :
" Sais-tu qu'j't'attends depuis plus d'une heure ? "
Dans son hot-dog il a mis du ketchup
Et quand Lili eut refait son make-up
Elle a dit : " Salut tu n'me r'verras plus "
Il lui a mis son couteau sur le cœur
Et puis il l'a
Prise dans ses bras...
\endverse

\beginchorus
" Lili j'voudrais bien t'épouser
Si tu voulais ne plus jamais
Ne plus jamais aller danser
Aller danser le rock'n roll "
Elle n'a pas su se désister
Elle n'a pas su lui résister
Elle a dit oui sans hésiter
En pleurnichant sur son épaule
\endchorus

\beginverse
Car ell'l'aimait encor' plus que l'rock'n roll
Il lui a promis d'l'aimer tout' sa vie
D'ailleurs on peut lire encore aujourd'hui
Au-d'sus de leur lit :" Tony loves Lili "
Dans un cœur en bois gravé au couteau
Oh oh oh oh
Oh oh oh oh
\endverse

\beginchorus
Lili voulait aller danser
Lili voulait aller danser
Lili voulait aller danser
Aller danser le rock'n roll
Mais Tony trouvait ça moins drôle
Lui qui n'aimait pas du tout le rock'n roll
\endchorus
\endsong

\beginsong{Ca fait pleurer le bon Dieu}[by={Julien Clerc},cr=1973]
\beginverse
\[F]Ils ont au fond \[C7]d’ leurs mouchoirs
\[C7]Un tout p’tit peu \[Dm]de brouillard
\[G7]Pour dissimuler \[C]leur chagrin
\[B&]Quand ils pleurent \[C7]dans leurs mains...
\endverse

\beginchorus
\[F]On n’ saura \[C7]jamais très bien
\[C7]Pourquoi pleurent \[Dm]les enfants
\[G7]Faudrait leur dire \[C]plus souvent
\[B&]Ce que disent les \[C]paysans

\[Dm]Pleure donc, pleure donc \[B&maj9]pas comme \[C]ça
\[B&]Ça fait ple\[A7]urer l’ bon \[Dm]Dieu La La
\[B&]Ça fait ple\[A7]urer l’ bon \[Dm]Dieu
Pleure donc, pleure donc pas comme ça
Ça fait pleurer l’ bon Dieu La La
Ça fait pleurer l’ bon Dieu - Bon\[F] Dieu...
\endchorus

\beginverse
C’est ainsi qu’ les paysans
Bercent leurs enfants malheureux
C’est ce que croient les enfants
En essuyant leurs grands yeux
On n’ saura jamais très bien
Pourquoi pleurent les enfants
Faudrait leur dire plus souvent
Ce que disent les paysans
\endverse

\beginverse
Depuis j’ai appris bien sûr
Que l’ bon Dieu ne pleurait pas
Du moins pas aussi souvent
Pas aussi souvent qu’ l’on croit
Mais chaque fois que je vois
Quelqu’un pleurer près de moi
Je ne peux pas m’empêcher
De doucement lui chanter
\endverse
\endsong

%%%Legalité

\beginsong{Légalité}[by={Bradaframanadamada}]
\beginverse
Bradaframandada booyaka big smoka
Swiss high grade mi growa
From Zion to Sainte-Croix
\endverse
\beginverse
J'aimerais bien comprendre, pourquoi la ganja peut effrayer certains comme le virus Ebola
Alors qu'en fait pas du tout, et même bien au contraire
Elle peut neutraliser vos douleurs en un éclair
Oh Marie-Jeanne, Marie-Jeanne, on te fait des misères
Alors que tu es délicieuse, en exté' ou sous serre
\endverse

\beginverse
Le mouvement est amorcé pour qu'enfin en Suisse
On n'soit plus les meilleurs seulement que en tennis
Je veux fumer de l'herbe de qualité, boucher l'trou d'la sécu en fumant mon tarpé mi sing
Faya
Entre un pétard ou un Ricard, oui si t'as vraiment le cafard, moi je choisis le marocco
Ghetto
OCB fais tourner tourner, OCB le carton blindé, même Zapata il en fumait des calumets
Tryo, Matmatah, Billy ze kick et Manu Chao, l'avaient déjà très bien compris
Marijuana illegal
\endverse

\beginverse
Pour l'instant mais ça va changer, grâce à l'intelligence de nos autorités
Qui étonnamment ont compris que 200 000 massives
Fument tous les jours à des fins récréatives
Oui, c'est comme au Canada ou au Colorado
On va enfin arrêter d'embêter le bédot
On croise les doigts et pas les bras
Pour ce grand projet du ministre suisse de la santé
\endverse

\beginverse
Haïlé Sélassié Berset the first, nous implorons ta clémence fédérale
Pour tout ce qui n'a pas été encore Jah fait sur le chemin de la légalisation
\endverse

\beginverse
Mais sache que la route sera glissante man, sinueuse accidenté
Il faudra contrôler tes freins, le niveau d'huile, les pneus d'été
Sam Fradaga et Jacka D te le dise man, boucle ta ceinture et mets le champignon
Tiens en parlant de champi' on pourrait faire une fricassée afin d'bien s'fracasser l'bidon
\endverse

\beginverse
Laissons-nous rouler, s'il vous plaît, en toute légalité
Laissais-nous planter, la sinse' pour ensuite la fumer
Laissons-nous rouler, s'il vous plaît, en toute légalité (la sinse')
L'égalité, égal ouais, à la légalité
\endverse

\beginverse
Yeah man
When Daddy smokes some weed
Is neva speed
When Daddy drinks some whisky
It is gonna be more risky
Yeah man
Rastafari marijuana cannabis
\endverse
\endsong


\beginsong{Blackbird}[by={The\ Beatles}] 
\beginchorus
B\[G]lackbird s\[Am7]inging in the d\[G/B]ead of ni\[G]ght
T\[C]ake these \[C#dim]broken w\[D]ings and l\[D#dim]earn to fl\[Em]y \[Eb]
A\[D]ll y\[C#dim]our li\[C]fe \[Cm]
Y\[G/B]ou were only w\[A7]aiting for this m\[D7]oment to ari\[G]se
\endchorus

\beginverse
Blackbird singing in the dead of night
Take these sunken eyes and learn to see
All your life
You were only waiting for this moment to be free
\endverse

\beginverse
B\[F]lack\[C/E]bir\[Dm]d \[C] fly\[Bb7] \[C]\rep{2}
Into the l\[D7sus4]ight of a dark black night
\endverse

\beginverse
B\[F]lack\[C/E]bir\[Dm]d \[C] fly\[Bb7] \[C],\rep{2}
Into the l\[D7sus4]ight of a dark black night
\endverse

\beginverse
Blackbird singing in the dead of night
Take these broken wings and learn to fly
All your life
You were only waiting for this moment to arise
You were only waiting for this moment to arise
You were only waiting for this moment to arise
\endverse 

\endsong


%%%%%%%%%%%%%%%%%% Leave Her Johnny
\beginsong{Leave Her, Johnny}[by={Traditionnel}, cr={19e siècle}]

\beginverse
\[D]Oh the times was hard and the \[G]wages \[D]low
\[A]Leave her, Johnny, \[D]leave her
But \[G]now once \[D]more as\[G]hore we'll \[D]go
And it's \[D]time for \[A]us to \[D]leave her
\endverse

\beginchorus
\[A]Leave her, Johnny, \[D]leave her
Oh \[G]leave her, Johnny, \[D]leave her
For the \[G]voyage is \[D]done and the \[G]winds do \[Bm]blow
And it's \[G]time for \[A]us to \[D]leave her
\endchorus

\beginverse
Oh, I thought I heard the old man say
Hey! Leave her, Johnny, leave her
Oh, tomorrow you will get your pay
And it's time for us to leave her
\endverse
\beginverse
Oh, the work was hard and the voyage long
Leave her, Johnny, leave her
Oh, the sea was high and the gales was strong
And it's time for us to leave her
\endverse
\beginverse
Oh, sing that we boys will never be
Leave her, Johnny, leave her
In a hungry bitch, the likes of she
And it's time for us to leave her
\endverse
\beginverse
Well the rats have left, and we the crew
Leave her, Johnny, leave her
Oh, it's time by damn that we went too
And it's time for us to leave her
\endverse

\endsong


%%%%%%%%%%%%%%% Old Maui
\beginsong{Old Maui}[by={The\ Dreadnoughts}, cr={2016}]

\beginverse
It's a \[Am]damn tough \[E]life full of \[F]toil and \[E]strife
We \[F]whalermen \[G]under\[Am]go
And we \[Am]won't give a \[E]damn when the \[F]gale is \[E]done
How \[F]hard the \[G]winds do \[Am]blow
For We're \[C]homeward bound from the \[G]Arctic Gound
With a \[Am]good ship taut and \[E]free
And we \[Am]don't give a \[E]damn when we \[F]drink our \[E]rum
With the \[F]girls of \[G]Old \[Am]Maui
\endverse

\beginchorus
Rolling \[C]down to Old \[G]Maui, me boys
Rolling \[Am]down to Old \[E]Maui
We're \[Am]homeward \[E]bound from the \[F]Arctic \[E]Ground
Rolling \[F]down to \[G]Old \[Am]Maui
\endchorus

\beginverse
Once more we sail with a Northerly gale
Through the ice, and wind, and rain
Them coconut fronds, them tropical shores
We soon shall see again
For Six hellish months we passed away
On the cold Kamchatka sea
But now we're bound from the Arctic ground
Rolling down to Old Maui
\endverse

\beginverse
Once more we sail the Northerly gale
Towards our Island home
Our whaling done, out mainmast sprung
And we ain't got far to roam
Our stans'l booms is carried away
What care we for that sound
A living gale is after us
Thank God we're homeward bound
\endverse

\beginverse
How soft the breeze through the island trees
Now the ice is far astern
Them native maids, them tropical glades
Is awaiting our return
Even now their big, brown eyes look out
Hoping some fine day to see
Our baggy sails running 'fore the gales
Rolling down to Old Maui
\endverse

\endsong



\beginsong{The sound of Silence}[by={Simon and Garfunkel},cr=1964]
\beginverse
\[Bm]Hello darkness, my old \[A]friend
I've come to talk with you \[Bm]again
\[D] Because a vision sof\[G]tly cree\[D]ping
Left its seeds while I \[G]was slee\[D]ping
And the \[G]vision that was planted in my \[D]brain
\[G]Still re\[D]mains
\[Bm] Within the \[A7]sound of \[Bm]silence
\endverse

\beginverse
In restless dreams I walked alone
Narrow streets of cobblestone
'Neath the halo of a street lamp
I turned my collar to the cold and damp
When my eyes were stabbed by the flash of a neon light
That split the night
And touched the sound of silence
\endverse

\beginverse
And in the naked light I saw
Ten thousand people, maybe more
People talking without speaking
People hearing without listening
People writing songs that voices never share
No one dared
Disturb the sound of silence
\endverse

\beginverse
"Fools" said I, "You do not know
Silence like a cancer grows
Hear my words that I might teach you
Take my arms that I might reach you"
But my words like silent raindrops fell
And echoed in the wells of silence
\endverse

\beginverse
And the people bowed and prayed
To the neon god they made
And the sign flashed out its warning
In the words that it was forming
And the sign said, "The words of the prophets
Are written on the subway walls
And tenement halls
And whispered in the sounds of silence"
\endverse
\endsong


\beginsong{Dancing Queen}[by={ABBA},cr=1976]
\beginchorus
\[G]You can dance, \[E]you can jive
\[Am]Having the \[Am/G]time of your \[D/F#]life
Ooh, \[F]see that girl, \[Dm]watch that scene
\[G]Digging the \[C]dancing queen\[F]\[C]\[F]
\endchorus

\beginverse
\[C]Friday night and the lights are \[F]low
\[C]Looking out for a place to \[Am]go
\[C]Where they play the right \[Am]music
\[C]Getting in the \[Am]swing
You come to look \[E]for a \[Am]king \[Em-Am]
\[C]Anybody could be that \[F]guy
\[C]Night is young and the music's \[Am]high
\[F]With a bit of rock \[Dm]music
\[F]Everything is \[Dm]fine
You're in the \[Em]mood \[Am]for a dance\[Em-Am]
And when you \[F]get the chance\[G]
\endverse

\beginchorus
You are the \[C]dancing queen, \[F]young and sweet, only \[F]seventeen
\[C]Dancing queen, \[F]feel the beat from the \[C]tambourine\[F], oh yeah
You can dance, you can jive
Having the time of your life
Ooh, see that girl
Watch that scene
Digging the dancing queen
\endchorus

\beginverse
You're a teaser, you turn 'em on
Leave 'em burning and then you're gone
Looking out for another
Anyone will do
You're in the mood for a dance
And when you get the chance
\endverse
\endsong


%%%%%%%Take a chance on me
\beginsong{Take a chance on me}[by={ABBA}]
\beginverse
If you change your m\[G]ind
I'm the first in line
Honey, I'm still free
Take a chance on m\[D]e
If you need me, let me know
Gonna be around
If you've got no place to go
When you're f\[G]eeling down
\endverse

\beginverse
If you're all alone
When the pretty birds have flown
Honey, I'm still free
Take a chance on me
Gonna do my very best
And it ain't no lie
If you put me to the test
If you let me try
\endverse 

\beginverse
Take a c\[Am]hance on m\[D]e
Take a c\[Am]hance on m\[D]e
\endverse

\beginverse
W\[Am]e can go dancing (oh)
We can go walking (yeah)
As l\[G]ong as we're together (long as we're together)
L\[Am]isten to some music (oh)
Maybe just talking (yeah)
G\[G]et to know you better (get to know you better)
\endverse

\beginverse
'Cause you know I've got
S\[Em]o much that I wanna do
W\[C]hen I dream I'm alone with you, it's m\[Em]agi\[C]c \[D]
Y\[Em]ou want me to leave it there
Af\[C]raid of a love affair, but I think y\[Am]ou know \[D]
That I c\[Am]an't let go\[D]
\endverse

\beginverse
If you change your m\[G]ind
I'm the first in line
Honey, I'm still free
Take a chance on m\[D]e
If you need me, let me know
Gonna be around
If you've got no place to go
When you're f\[G]eeling down
\endverse

\beginverse
If you're all alone
When the pretty birds have flown
Honey, I'm still free
Take a chance on me
Gonna do my very best
And it ain't no lie
If you put me to the test
If you let me try
\endverse 

\beginverse
Take a c\[Am]hance on m\[D]e
Take a c\[Am]hance on m\[D]e
\endverse

\beginverse
Oh, you can t\[Am]ake your time, baby (oh)
I'm in no hurry (yeah)
K\[G]now I'm gonna get you (know I'm gonna get you)
Y\[Am]ou don't wanna hurt me (oh)
Baby, don't worry (yeah)
I\[G] ain't gonna let you (I ain't gonna let you)
\endverse

\beginverse
Let me tell you now
M\[Em]y love is strong enough
T\[C]o last when things are rough, it's m\[Em]agic\[C] \[D]
Y\[Em]ou say that I waste my time
B\[C]ut I can't get you off my mind, no, I c\[Am]an't let go \[D]
'Cause I l\[Am]ove you so \[D]
\endverse
\beginverse
If you change your m\[G]ind
I'm the first in line
Honey, I'm still free
Take a chance on m\[D]e
If you need me, let me know
Gonna be around
If you've got no place to go
When you're f\[G]eeling down
\endverse

\beginverse
If you're all alone
When the pretty birds have flown
Honey, I'm still free
Take a chance on me
Gonna do my very best
And it ain't no lie
If you put me to the test
If you let me try
\endverse 

\beginverse
Take a c\[Am]hance on m\[D]e
Take a c\[Am]hance on m\[D]e
\endverse

\beginverse
Ba-ba-ba, b\[G]a, ba
Ba-ba-ba, ba, ba
Honey, I'm still free
Take a chance on m\[D]e
Gonna do my very best
Baby, can't you see
Gotta put me to the test
Take a c\[G]hance on me
Take a chance, take a chance, take a chance on me
\endverse
\endsong

\beginsong{I will survive}[by={Gloria Gaynor},cr=1978]
\beginverse
At \[Am]first, I was afraid, I was \[Dm]petrified
Kept thinking \[G]I could never live without you \[Cmaj7]by my side
But then I \[Fmaj7]spent so many nights thinking \[Bm7&5]how you did me wrong
And I grew \[Esus4]strong, and I learned \[E]how to get along
And so you're back from outer space
I just walked in to find you here with that sad look upon your face
I should have changed that stupid lock
I should have made you leave your key
If I'd have known for just one second you'd be back to bother me
\endverse

\beginchorus
Go on now, go. Walk out the door
Just turn around now 'cause you're not welcome anymore
Weren't you the one who tried to hurt me with goodbye?
Did you think I'd crumble?
Did you think I'd lay down and die?
Oh, no, not I
I will survive
Oh, as long as I know how to love I know I'll stay alive
I've got all my life to live
I've got all my love to give
And I'll survive
I will survive, hey, hey
\[Am]\[Dm]\[G]\[Cmaj7]\[Fmaj7]\[Bm7&5]\[Esus4]\[E]
\endchorus

\beginverse
It took all the strength I had not to fall apart
Kept trying hard to mend the pieces of my broken heart
And I spent, oh, so many nights just feeling sorry for myself
I used to cry, but now I hold my head up high
And you see me somebody new
I'm not that chained-up little person still in love with you
And so you felt like dropping in
And just expect me to be free
And now I'm saving all my loving for someone who's loving me
\endverse

\beginverse
Go on now, go. Walk out the door
Just turn around now 'cause you're not welcome anymore
Weren't you the one who tried to break me with goodbye?
Did you think I'd crumble?
Did you think I'd lay down and die?
Oh, no, not I
I will survive
Oh, as long as I know how to love I know I'll stay alive
I've got all my life to live
I've got all my love to give
And I'll survive
I will survive
\endverse

\endsong


\beginsong{Bon voyage Monsieur Dumollet}[by={Marc-Antoine-Madeleine Désaugiers},cr={1809}]
\beginverse
Bon voy\[D]age Monsieur Dumollet
A Saint-Ma\[A7]lo débarquez sans nau\[D]frage
Bon voyage, Monsieur Dumollet
Et revenez si le pays vous \[D]plaît
\endverse

\beginverse
Mais si vous allez voir la capitale
Méfiez-\[G]vous des voleurs, des a\[D]mis,
Des billets doux, des coups, de la cabale
\[G]Des pisto\[D]lets et des \[E]tortico\[A7]lis
Bon voyage Monsieur Dumollet
\endverse

\beginverse

Là vous verrez les deux mains dans les poches
Aller, venir des sages et des fous
Des gens bien faits, des tordus, des bancroches,
Nul ne sera jambé si bien que vous
Bon voyage Monsieur Dumollet
\endverse

\beginverse

Des polissons vous feront bien des niches
A votre nez riront bien des valets
Craignez surtout les barbets, les caniches,
Car ils voudront caresser vos mollets
Bon voyage Monsieur Dumollet
\endverse

\beginverse

L'air de la mer peut vous être contraire
Pour vos bas bleus, les flots sont un écueil
Si ce séjour venait à vous déplaire
Revenez nous avec bon pied bon oeil
Bon voyage Monsieur Dumollet
\endverse

\beginverse

A Saint Malo débarquez sans naufrage
Bon voyage Monsieur Dumollet
Et revenez si le pays vous plaît.
\endverse
\endsong
