\beginsong{Ne sens-tu pas claquer tes doigts}[by={Traditionnel\ français}]

\beginverse
N’entends-tu pas claquer tes \[C]doigts, claquer tes doigts
Et la musique monter en \[Am]toi, monter en toi?
N'attends pas que le feu soit \[F]mort, le feu soit mort,
Chante tant que tu peux \[G]encore, tu peux encore.
Encore, encore, encore, encore, en\[C]core…
\endverse

\beginverse
Tu peux donner tout c’que tu as, tout c’que tu as
Mais oui, la vie c’est fait pour ça, c’est fait pour ça
Tu peux sourire autour de toi, autour de toi
Tendre la main à qui voudra, à qui voudra.
La main, la main, la main, la main, la main…
\endverse

\beginverse
Ne sens-tu pas chanter la vie, chanter la vie
Et toute la joie réunie, joie réunie
Qui nous rassemble tous ici, oui tous ici
Jusqu’à la fin de notre vie, de notre vie.
La vie, la vie, la vie, la vie, la vie…
\endverse

\beginverse
Viens avec nous rire et chanter, rire et chanter
Ce soir tu dois participer, participer
Ensemble, vivons cette veillée, cette veillée
Ici tu n’es plus étranger, plus étranger.
Rire et, chanter, chanter, chanter, chanter…
\endverse

\beginverse
Et pour tous ceux qui ont faim, ceux qui ont faim
Donne-leur un peu de ton pain, peu de ton pain
Pour qu’ils partagent tous ensemble, oui tous ensemble
Cet amour qui nous rassemble, oui nous rassemble.
L’amour, l’amour, l’amour, l’amour, l’amour.
\endverse

\beginverse
Ne sens-tu pas battre ton cœur, battre ton cœur
Qui s’éparpille en mille fleurs, en mille fleurs?
Et prends la main de ton ami, de ton ami
Et garde-la toute ta vie, toute ta vie.
La vie, la vie, la vie, la vie, la vie.
\endverse
\endsong

%%%%%%%%%%%%%%%%%%%%%%%%%%%%%%%%%%%%%%%%%%
\beginsong{Au gré du vent (canon)}[by={Traditionnel breton}]
\beginverse*
\[Dm]Dans le port de Lorient, de grands voiliers tout \[C]blancs,
Attendent en rê\[(A7b5)]vant, au gré du vent,
Dans le port de Lorient, de grands voiliers tout blancs,
Attendent en rêvant, au gré du vent,
Puis ils s’en vont pour l’Orient, au gré du vent.
\endverse
\endsong

% %%%%%%%%%%%%%%%%%%%%%%%%%%%%%%%%%
% \beginsong{Nous n'irons plus au bois}[by={Traditionnel\ français}]

% \beginverse
% Nous n'irons \[F]plus au bois, les lauriers \[C7]sont cou\[F]pés,
% La belle que voilà ira les ramas\[F]ser
% \endverse

% \beginverse
% Entrez dans la dan\[C]se, \[F]voyez, comme on dan\[C]se,
% \[F]Sautez, dansez, e\[C7]mbrassez qui vous vou\[F]drez. 
% \endverse

% \beginverse
% La belle que voilà ira les ramasser,
% Mais les lauriers du bois, les lairons-nous couper ? 
% \endverse

% \beginverse
% Non chacune à son tour ira les ramasser.
% Si la cigale y dort il n'faut pas la blesser.
% \endverse

% \beginverse
% Le chant du rossignol viendra la réveiller.
% Et aussi la fauvette avec son doux gosier.
% \endverse

% \beginverse
% Et Jeanne la bergère avec son blanc panier.
% Allant cueillir la fraise et la fleur d'églantier.
% \endverse

% \beginverse
% Cigale, ma cigale, allons, il faut chanter.
% Car les lauriers du bois sont déjà repoussés.
% \endverse
% \endsong


%%%%%%%%%%%%%%%% Vent frais

\beginsong{Vent frais, vent du matin (canon)}[by={Traditionnel\ français}]
\beginverse
\[Cm]Vent \[B&]frais, \[Cm]vent du \[G7]matin
\[Cm]Vent qui \[B&]souffle aux \[Cm]sommets des grands \[G7]pins
\[Cm]Joie du \[G7]vent qui \[Cm]souffle \[G7]allons dans le grand
\[Cm]Vent \[B&]frais, \[Cm]vent du \[G7]matin...
\endverse
\endsong

%%%%%%%%%%%%%%%%%%%%%% Ani couni
\beginsong{Ani couni}[by={Traditionnel\ iroquois}]

\beginchorus
\[Am]Ani \[Dm]couni \[Am]chaoua\[Dm]ni \[Am] \rep{2}
\[Dm]Awawa \[Am]bikana \[Dm]caïna\[Am] \rep{2}
\[Am]E aoun\[Em]i biss\[Am]ini \rep{2}
\endchorus

\endsong

% \beginsong{Cadet Roussel}[by={Traditionnel\ français},cr=1792]
% \beginverse
% Cadet Rous\[A]sel a trois maisons (bis)
% \[Bm]Qui n'ont ni \[C#m7]poutres ni che\[E7]vrons: (bis)
% C'est pour loger les hirondelles;
% Que direz-vous d'Cadet Roussel
% \[A]Ah! \[F#m]Ah! Ah! mais vrai\[E]ment,
% \[A]Cadet Rous\[B7]sel est bon en\[A]fant.
% \endverse

% \beginverse
% Cadet Roussel a trois habits : (bis)
% Deux jaunes, l'autre en papier gris; (bis)
% Il met celui-là quand il gèle,
% Ou quand il pleut ou quand il grêle;
% Ah! Ah! Ah! mais vraiment,
% Cadet Roussel est bon enfant.
% \endverse

% \beginverse
% Cadet Roussel a trois chapeaux : (bis)
% Les deux ronds ne sont pas très beaux, (bis)
% Et le troisième est à deux cornes,
% De sa tête il a pris la forme
% Ah! Ah! Ah! mais vraiment,
% Cadet Roussel est bon enfant.
% \endverse

% \beginverse
% Cadet Roussel a trois beaux yeux (bis)
% L'un r'garde à Caen, l'autre à Bayeux (bis)
% Comme il n'a pas la vue bien nette,
% Le troisième c'est sa lorgnette
% Ah! Ah! Ah! mais vraiment,
% Cadet Roussel est bon enfant.
% \endverse

% \beginverse
% Cadet Roussel a une épée (bis)
% Très longue mais toute rouillée (bis)
% Aussi chacun de dire d'elle
% Qu'ell'ne fait peur qu'aux hirondelles
% Ah! Ah! Ah! mais vraiment,
% Cadet Roussel est bon enfant.
% \endverse

% \beginverse
% Cadet Roussel a trois souliers (bis)
% Il en met deux dans ses deux pieds ; (bis)
% Le troisième étant pour bancroche,
% Quand il le met c'est dans sa poche :
% Ah! Ah! Ah! mais vraiment,
% Cadet Roussel est bon enfant.
% \endverse

% \beginverse
% Cadet Roussel a trois cheveux : (bis)
% Un pour chaqu'face, un pour la queue, (bis)
% Pourtant parfois avec adresse
% Il les met tous les trois en tresse
% Ah! Ah! Ah! mais vraiment,
% Cadet Roussel est bon enfant.
% \endverse

% \beginverse
% Cadet Roussel a trois garçons (bis)
% L'un est voleur, l'autre est fripon, (bis)
% Le troisième est un peu ficelle
% Il ressemble à Cadet Roussel
% Ah! Ah! Ah! mais vraiment,
% Cadet Roussel est bon enfant.
% \endverse

% \beginverse
% Cadet Roussel a trois gros chiens: (bis)
% L'un court aux lièvres, l'autre aux lapins, (bis)
% L'troisième s'enfuit quand on l'appelle
% Comm' le chien de Jean de Nivelle
% Ah! Ah! Ah! mais vraiment,
% Cadet Roussel est bon enfant.
% \endverse

% \beginverse
% Cadet Roussel a trois beaux chats, (bis)
% Qui n'attrapent jamais les rats, (bis)
% Le troisièm' n'a pas de prunelles,
% Il monte au grenier sans chandelle
% Ah! Ah! Ah! mais vraiment,
% Cadet Roussel est bon enfant.
% \endverse

% \beginverse
% Cadet Roussel a marié (bis)
% Ses trois filles dans trois quartiers ; (bis)
% Les deux premières ne sont pas belles
% La troisième n'a pas de cervelle
% Ah! Ah! Ah! mais vraiment,
% Cadet Roussel est bon enfant.
% \endverse

% \beginverse
% Cadet Roussel a trois deniers, (bis)
% C'est pour payer ses créanciers ; (bis)
% Quand il a montré ses ressources,
% Il les remet dedans sa bourse
% Ah! Ah! Ah! mais vraiment,
% Cadet Roussel est bon enfant.
% \endverse

% \beginverse
% Cadet Roussel s'est fait acteur, (bis)
% Comme CHENIER s'est fait auteur ; (bis)
% Au café quand il joue son rôle,
% Les aveugles le trouvent drôle
% Ah! Ah! Ah! mais vraiment,
% Cadet Roussel est bon enfant.
% \endverse

% \beginverse
% Cadet Roussel ne mourra pas
% Car avant de sauter le pas (bis)
% On dit qu'il apprend l'orthographe
% Pour faire lui-même son épitaphe :
% Ah! Ah! Ah ! mais vraiment Cadet Roussel est bon enfant. 
% \endverse
% \endsong

% \beginsong{Nous aimons vivre au fond des bois}[by={Traditionnel\ russe}]
% \beginverse
% \[Em] Nous aimons vivre au fond des bois,
% Aller \[B7]coucher sur la \[Em]du-re.
% \[A]La forêt nous dit \[Em]de ses mille mots :
% Lance-\[B7]toi dans la grande aven\[Em]tu\[E]re.
% La forêt nous dit de ses mille mots :
% Lance-toi dans la grande aventu - u - re.
% \endverse

% \beginchorus
% Laï, Laï, Laï, Laï, Laï, Laï, Laï, Laï!
% Laï, Laï, Laï, Laï, Laï, Laï, Laï, Laï, Laï! Hey!
% Laï, Laï, Laï, Laï, Laï, Laï, Laï, Laï!
% Laï, Laï, Laï, Laï, Laï, Laï, Laï, Laï, Laï!
% \endchorus

% \beginverse
% Nous aimons vivre auprès du feu
% Et danser sous les étoi - oi - les.
% La nuit claire nous dit de ses mille voix :
% Sois gai lorsque le ciel est sans voi - oi - le.
% La nuit claire nous dit de ses mille voix :
% Sois gai lorsque le ciel est sans voi - oi - le.
% \endverse

% \beginverse
% Nous aimons vivre sur nos chevaux
% Dans les plaines du Cauca - a - se.
% Emportés par de rapides galops
% Nous allons plus vite que Péga - a - se.
% Emportés par de rapides galops
% Nous allons plus vite que Péga - a - se.
% \endverse
% \endsong


% \beginsong{Ode an die Freude}[by={Ludwig van\ Beethoven}, cr=1824]
% \beginverse
% \[F]Freude, schöner \[C]Götterfunken,
% \[Dm]Toch\[C]ter \[F]aus Elys\[C]ium,
% \[F]Wir betreten \[C]feuertrunken.
% \[Dm]Himmli\[C]sche, dein Heil\[F]igtum!
% \[C]Deine \[F]Zauber \[C]binden \[F]wieder
% \[C]Was die \[A7]Mode \[Dm]streng \[G7]get\[C]eilt;
% \[F]Alle Menschen \[C]werden Brüder
% \[Dm]Wo dein \[C]sanfter Flügel \[F]weilt.
% \endverse

% \beginverse
% Wem der grosse Wurf gelungen
% Eines Freundes Freund zu sein,
% Wer ein holdes Weib errungen,
% Mische seinen Jubel ein!
% Ja, wer auch nur eine Seele
% Sein nennt auf dem Erdenrund!
% Und wer's nie gekonnt, der stehle
% Weinend sich aus diesem Bund.
% \endverse

% \beginverse
% Freude trinken alle Wesen
% An den Brüsten der Natur;
% Alle Guten, alle Bösen,
% Folgen ihrer Rosenspur.
% Küsse gab sie uns und Reben,
% Einen Freund, geprüft I'm Tod;
% Wollust ward dem Wurm gegeben,
% Und der Cherub steht vor Gott!
% \endverse

% \beginverse
% Froh, wie seine Sonnen fliegen
% Durch des Himmels prächt'gen Plan,
% Laufet, Brüder, eure Bahn,
% Freudig, wie ein Held zum Siegen.
% \endverse

% \beginverse
% Seid umschlungen, Millionen.
% Dieser Kuss der ganzen Welt!
% Brüder! Über'm Sternenzelt
% Muss ein lieber Vater wohnen.
% Ihr stürzt nieder, Millionen?
% Ahnest du den Schöpfer, Welt?
% Such ihn über'm Sternenzelt!
% Über Sternen muss er wohnen. 
% \endverse
% \endsong

% \beginsong{Hymne à la Joie}[by={Ludwig van\ Beethoven}, cr={1824}]
% \beginverse
% \[F]Joie discrète, hum\[C]ble et fidèle
% \[Dm]Qui mur\[C]mure \[F]dans les \[C]eaux
% \[F]Dans le froisse\[C]ment des ailes
% \[Dm]Et les \[C]hymnes des oi\[F]seaux.
% \[C]Joie qui \[F]vibre \[C]dans les \[F]feuilles
% \[C]Dans les \[A7]prés et \[Dm]les \[G7]mois\[C]sons
% \[F]Nos âmes blan\[C]ches t’accueillent
% \[Dm]Par de \[C]naïves chan\[F]sons.
% \endverse

% \beginverse
% Tous les hommes de la terre
% Veulent se donner la main
% Vivre et s’entraider en frères
% Pour un plus beau lendemain,
% Plus de haine, plus de frontière,
% Plus de charniers sur nos chemins
% Nous voulons d’une âme fière
% Nous forger un grand destin
% \endverse

% \beginverse
% Que les peuples se rassemblent
% Dans une éternelle foi
% Que les hommes se rassemblent
% Dans l’égalité des droits.
% Nous pourrons tous vivre ensemble
% La charité nous unira
% Que pas un de nous ne tremble
% La fraternité viendra.
% \endverse

% \beginverse
% Joie immense, joie profonde,
% Ombre vivante de Dieu
% Abats-toi sur notre monde
% Comme un aigle vient des cieux.
% Enserre dans ton étreinte
% La tremblante humanité
% Que s’évapore la crainte
% Que naisse la liberté
% \endverse

% \beginverse
% Joie énorme, joie terrible
% Du sacrifice total
% Toi qui domptes l’impossible,
% Et maîtrises le fatal;
% Joie sauvage, âpre et farouche,
% Cavalière de la mort,
% Nous soufflons à pleine bouche
% Dans l’ivoire de ton cor.
% \endverse

% \beginverse
% Joie qui monte et déborde,
% Tu veux nos cœurs? les voilà.
% Et nos âmes sont les cordes,
% Où ton archet passera
% Que ton rythme nous emporte
% Aux splendeurs de l’Eternel
% Comme un vol de feuilles mortes,
% Que l’orage entraîne au ciel. 
% \endverse
% \endsong

% \beginsong{La Piémontaise}[by={Traditionnel\ français}]
% \beginverse
% Mon \[G]Dieu ! que \[D7]je suis à mon \[G]aise
% Quand j'ai ma \[D]mie auprès de \[D7]moi
% \[G]Tout doucement je la re\[Am]garde, Et je lui \[D7]dis embrasse-\[G]moi.\rep{2}
% \endverse

% \beginverse
% Comment veux-tu que je t'embrasse ?
% Quand on me dit du mal de toi
% On dit que tu pars pour l'armée
% Dans le Piémont servir le Roi
% \endverse

% \beginverse
% Quand tu seras dans ces campagnes
% Tu ne penseras plus à moi
% Tu penseras aux Italiennes
% Qui sont cent fois plus belles que moi.
% \endverse

% \beginverse
% Si fait, si fait, si fait, ma belle
% J'y penserai toujours à toi
% Je m'en ferai faire une image,
% Tout à la semblance de toi
% \endverse

% \beginverse
% Quand je serai à table à boire
% À tous mes amis je dirai
% Chers camarades venez voir,
% Celle que mon coeur à tant aimé
% \endverse

% \beginverse
% Je l'ai aimée, je l'aime encore
% Je l'aimerai tant qu'je vivrai
% Je l'aimerai quand je serai mort.
% Si c'est donné aux trépassés.
% \endverse

% \beginverse
% Alors, j'ai tant versé de larmes
% Que trois moulins en ont tourné.
% Petits ruisseaux, grandes rivières
% Pendant trois jours ont débordé. 
% \endverse
% \endsong


% \beginsong{Le chant des Marais}[by={Rudi Gogue}]
% \beginverse
% \[Gm]Loin dans l’infini s’étendent
% \[Cm]De grands \[Gm]prés ma\[Am7]ré\[D#7]ca\[Gm]geux.
% \[B&]Et là-bas nul oi\[Gm]seau ne \[B&]chante
% \[Cm]Dans les \[Gm]arbres \[D7]secs et \[Gm]creux.
% \endverse

% \beginchorus
% Ô, \[B&]terre de dét\[F7]resse
% Où \[Gm]nous devons sans \[D7]cesse
% Pio\[Gm]cher, Piocher, \[D7]Pio\[Gm]cher
% \endchorus

% \beginverse
% Dans ce camp morne et sauvage
% Entouré de murs de fer
% Il nous semble vivre en cage
% Au milieu d’un grand désert
% \endverse

% \beginverse
% Bruit des pas et bruit des armes,
% Sentinelles jours et nuits,
% Et du sang, des cris, des larmes,
% La mort pour celui qui fuit.
% \endverse

% \beginverse
% Mais un jour dans notre vie,
% Le printemps refleurira.
% Liberté, liberté chérie
% Je dirai: Tu es à moi 
% \endverse

% \beginverse
% Ô, terre d'allégresse
% Où nous pourrons sans cesse
% Aimer (bis) 
% \endverse
% \endsong

\beginsong{Chevaliers de la table ronde}[by={Traditionnel\ français}]
\beginverse
 Cheva\[C]liers de la Table Ronde goûtons \[G7]voir si le vin est \[C]bon\rep{2}
\endverse

\beginchorus
Goûtons \[F]voir, oui, oui, oui, goûtons v\[C]oir, non, non, non
Goûtons \[G]voir \[G7]si le vin est \[C]bon\[C7] \rep{2}
\endchorus

\beginverse
S’il est bon, s’il est agréable, j’en boirai jusqu’à mon plaisir
\endverse
\beginverse
J’en boirai cinq a six bouteilles, et encore ce n’est pas beaucoup
\endverse
\beginverse
Si je meurs, je veux qu’on m’enterre dans une cave où il y a du bon vin
\endverse
\beginverse
Les deux pieds contre la muraille et la tête sous le robinet
\endverse
\beginverse
Et les quatre plus grands ivrognes porteront les quat’coins du drap
\endverse
\beginverse
Pour donner le discours d’usage on prendra le bistrot du coin
\endverse
\beginverse
Et si le tonneau se débouche j’en boirai jusqu’à mon plaisir
\endverse
\beginverse
Et s’il en reste quelques gouttes ce sera pour nous rafraîchir
\endverse
\beginverse
Sur ma tombe je veux qu’on inscrive ici gît le Roi des buveurs
\endverse
\endsong

\beginsong{La Marseillaise}[by={Clause-Joseph Rouget\ de\ Lisle}, cr=1792]
\beginverse
Allons enf\[G]ants de \[D]la Pat\[G]rie
Le jour de \[C]gloire est \[D]arri\[G]vé
Contre nous de la tyranni\[D]e
L'étend\[D7]ard sanglant est lev\[G]é
L'étendard sanglant est lev\[D]é
Entendez vous dans les campa\[Em]gnes
\[G7]Mug\[C]ir ces féroces sol\[D]dats
Ils \[Gm]viennent jusque dans vos \[F]bras
Égor\[E&]ger vos fils\[Cm], vos compa\[D]gnes
\endverse

\beginchorus
Aux \[G]armes citoy\[D]ens! For\[G]mez vos batail\[D]lons!
Mar\[G]chons, \[D]mar\[G7]chons, \[C]qu'un sang im\[D]pur \[C]ab\[G]reuve \[D7]nos sil\[G]lons
\endchorus

\beginverse
Que veut cette horde d'esclaves
De traîtres, de Rois conjurés ?
Pour qui ces ignobles entraves,
Ces fers dès longtemps préparés ?
Ces fers dès longtemps préparés ?
Français ! pour nous, ah ! quel outrage !
Quels transports il doit exciter !
C'est nous qu'on ose méditer
De rendre à l'antique esclavage !
\endverse

\beginverse
Quoi ! des cohortes étrangères
Feraient la loi dans nos foyers ?
Quoi ! ces phalanges mercenaires
Terrasseraient nos fiers guerriers
Terrasseraient nos fiers guerriers
Grand Dieu ! par des mains enchaînées
Nos fronts sous le joug se ploieraient,
De vils despotes deviendraient
Les maîtres de nos destinées ?
\endverse

\beginverse
Tremblez, tyrans ! et vous, perfides,
L'opprobre de tous les partis,
Tremblez ! vos projets parricides
Vont enfin recevoir leur prix
Vont enfin recevoir leur prix
Tout est soldat pour vous combattre,
S'ils tombent, nos jeunes héros,
La terre en produit de nouveaux
Contre vous tous prêts à se battre
\endverse

\beginverse
Français ! en guerriers magnanimes
Portez ou retenez vos coups.
Épargnez ces tristes victimes
A regret s'armant contre nous
A regret s'armant contre nous
Mais le despote sanguinaire,
Mais les complices de Bouillé,
Tous ces tigres qui sans pitié
Déchirent le sein de leur mère
\endverse

\beginverse
Amour sacré de la Patrie
Conduis, soutiens nos bras vengeurs !
Liberté, Liberté chérie !
Combats avec tes défenseurs
Combats avec tes défenseurs
Sous nos drapeaux, que la victoire
Accoure à tes mâles accents,
Que tes ennemis expirant
Voient ton triomphe et notre gloire !
\endverse

\beginverse
Nous entrerons dans la carrière,
Quand nos aînés n'y seront plus
Nous y trouverons leur poussière
Et les traces de leurs vertus.
Et les traces de leurs vertus.
Bien moins jaloux de leur survivre
Que de partager leur cercueil,
Nous aurons le sublime orgueil
De les venger ou de les suivre !
\endverse

\beginverse
Enfants, que l'Honneur, la Patrie
Fassent l'objet de tous nos vœux !
Ayons toujours l'âme nourrie
Des feux qu'ils inspirent tous deux. (bis)
Soyons unis ! Tout est possible ;
Nos vils ennemis tomberont,
Alors les Français cesseront
De chanter ce refrain terrible :
\endverse

\endsong



%%%%%%%%%%%%%%%%%%%%% L'Homme de Cro
\beginsong{L'Homme de Cromagnon}[by={Les Vagabonds}, cr={1946}]

\beginverse
C'était au \[F]temps d'la \[C]préhist\[F]oire
Il y a deux ou trois \[C]cent mille ans
Vint au monde \[Gm7]un être bizarre
Proche par\[C7]ent d'l'orang-ou\[F]tang
Debout sur ses \[F]pattes \[C]de derr\[F]ière
Vêtu d'un slip en peau \[C]d'bison
Il allait \[Gm7]conquérir la Terre
C'était l'Hom\[C]me de \[C7]Cro-Ma\[F]gnon
\endverse

\beginchorus
\[F]L'Homme de cro, l'Homme de ma, l'Homme de gnon
\[C7]L'Homme de Cro-Mag\[F]non, pom pom
\[Gm7]L'Homme de Cro de Ma\[F]gnon ce n'est pas du bi\[C7]don
L'Homme de Cro-Mag\[F]non, pom \[F7]pom
\[B&]L'Homme de Cro de Ma\[F]gnon ce n'est pas du bi\[C]don
L'Homme de \[C7]Cro-Magn\[F]on
\endchorus

\beginverse
Armé de sa hache de pierre
De son couteau de pierre itou
Il chassait l'ours et la panthère
En serrant les fesses malgré tout
Devant l'diplodocus en rage
Il se faisait tout d'même un peu petit
En disant dans son langage
"Vivement qu'on invente le fusil"
\endverse

\beginverse
Il était poète à ses heures
Disant à sa femme en émoi
"Tu es belle comme un dinosaure
Tu ressembles à Lolo Brigitta
Si tu veux voir des cartes postales
Viens dans ma caverne, tout là-haut
J'te ferai voir mes peintures murales
On dirait du vrai Picasso"
\endverse

\beginverse
Deux cent mille ans après, sur Terre
Comme nos ancêtres, nous admirons
Les bois, les champs et les rivières
Mais s'ils revenaient, quelle déception
Nous voir suer six jours sur sept
Ils diraient sans faire de détail
"Faut-y qu'nos héritiers soient bêtes
Pour avoir inventé l'travail"
\endverse


\endsong



%%%%%%%%%%%%%%%%%%%%%%%%%%%%% Ram sam sam
\beginsong{Ram Sam Sam}[by={Traditionnel marocain}]

\beginverse*
A \[C]ram sam sam, a ram sam sam
\[G]Guli guli guli guli \[C]ram sam sam
A ram sam sam, a ram sam sam
\[G]Guli guli guli guli \[C]ram sam sam
\[C]A rafiq, a rafiq
\[G]Guli guli guli guli \[C]ram sam sam
\[Am]A rafiq, a rafiq
\[G7]Guli guli guli guli \[C]ram sam sam
\endverse


\endsong


%%%%%%%%%%%%%%%%%%%%%%%%%%%%%%%%%%
%%%%%%%%%%%%%%%% Le chant des partisons

\beginsong{Le chant des partisans}[by={Joseph Kessel}, cr={1944}]

\beginverse
Am\[F]i, entends-tu le vol \[B&]noir des corbeaux sur nos \[C]plaines?
Am\[F]i, entends-tu les cris \[B&]sourds du pays qu'on \[C]enchaîne?
Oh\[Am]é, partisans, ouv\[Dm]riers et paysans, c'est l'al\[Gm7]arme
Ce \[Am]soir l'ennemi connaît\[Dm]ra le prix du sang et les \[Gm7]larmes
\endverse

\beginverse
Montez de la mine, descendez des collines, camarades
Sortez de la paille les fusils, la mitraille, les grenades
Ohé, les tueurs à la balle et au couteau, tuez vite
Ohé, saboteur, attention à ton fardeau, dynamite
\endverse

\beginverse
C'est nous qui brisons les barreaux des prisons pour nos frères
La haine à nos trousses et la faim qui nous pousse, la misère
Il y a des pays où les gens au creux des lits font des rêves
Ici, nous, vois-tu, nous on marche et nous on tue, nous on crève
\endverse

\beginverse
Ici chacun sait ce qu'il veut, ce qu'il fait quand il passe
Ami, si tu tombes un ami sort de l'ombre à ta place
Demain du sang noir séchera au grand soleil sur les routes
Sifflez, compagnons, dans la nuit la Liberté nous écoute
\endverse
\endsong

%%%%%%%%%%%%%%%%%%%%%%%%%%%%%%%%%%%%
\beginsong{La Strasbourgeoise}[by=Henri Natif, cr=1870]

\beginverse
Petit pa\[Am]pa, voici la mi-car\[Am]ême,  
Car te voi\[Dm]ci déguisé en sold\[Am]at.      
Petit pa\[Dm]pa, dis-moi si c'est pour \[Am]rire    
Ou pour faire \[E]peur aux tout-petits enf\[Am]ants \rep{2}
\endverse

\beginverse
Non, mon enfant, je pars pour la Patrie :
C'est un devoir où tous les papas s'en vont.
Embrasse-moi, petite fille chérie,
Je rentrerai bien vite à la maison (x2)
\endverse

\beginverse
Dis-moi, maman, quelle est cette médaille,
Et cette lettre qu'apporte le facteur?
Dis-moi maman, tu pleures et tu défailles
Ils ont tué petit père adoré (x2)
\endverse

\beginverse
Oui, mon enfant, ils ont tué ton père;
Pleurons ensemble, car nous les haïssons.
Quelle guerre atroce qui fait pleurer les mères
Et tue les pères des petits anges blonds (x2)
\endverse

\beginverse
La neige tombe aux portes de la ville.
Là est assise une enfant de Strasbourg.
Elle reste là malgré le froid, la bise,
Elle reste là malgré le froid du jour (x2)
\endverse

\beginverse
Un homme passe, à la fillette donne.
Elle reconnaît l'uniforme allemand.
Elle refuse l'aumône qu'on lui donne.
A l'ennemi, elle dit bien fièrement (x2)
\endverse

\beginverse
Gardez votre or, je garde ma puissance;
Soldat prussien, passez votre chemin.
Moi, je ne suis qu'une enfant de la France.
A l'ennemi, je ne tends pas la main (x2)
\endverse

\beginverse
Tout en priant sous cette cathédrale,
Ma mère est morte sous ce porche écroulé,
Frappée à mort par l'une de vos balles,
Frappée à mort par l'un de vos boulets (x2)
\endverse

\beginverse
Mon père est mort sur vos champs de bataille,
Je n'ai pas vu l'ombre de son cercueil,
Frappé à mort par l'une de vos balles.
C'est la raison de ma robe de deuil (x2)
\endverse

\beginverse
Vous avez eu l'Alsace et la Lorraine,
Vous avez eu des millions d'étrangers,
Vous avez eu Germanie et Bohême,
Mais mon p'tit coeur, vous ne l'aurez jamais,
Mais mon p'tit coeur, lui restera français !
\endverse

\endsong


%%%%%%%%%%%%%%%%%%%%%%%%%%%%%%%%%%%%%%%%%% La bamba
\beginsong{La bamba}[by={Ritchie Valens}, cr={1955, (c'est pas français je sais)}]

\beginverse
Para bailar la b\[C]amba
Para bail\[F]ar la b\[C]amba se n\[F]eces\[G]ita una p\[F]oca de g\[C]racia
Una p\[F]oca de gr\[C]acia pa' m\[F]í, pa' t\[G]i, arri\[F]ba y arri\[C]ba
Y arr\[F]iba, y arr\[C]iba, por t\[F]i se\[G]ré
Por t\[F]i se\[C]ré, por t\[F]i se\[G]ré
Yo no soy marinero
Yo no soy marinero, soy capitán
Soy capitán, soy capitán
\endverse

\beginchorus
Bamba, bamba, bamba, bamba
Bamba, bamba, va
\endchorus

\beginverse
Para bailar la bamba \rep{2}
Se necesita una poca de gracia
Una poca de gracia pa' mí, pa' ti, arriba y arriba
Para bailar la bamba \rep{2}
Se necesita una poca de gracia
Una poca de gracia pa' mí, pa' ti, arriba y arriba
Y arriba, y arriba, por ti seré
Por ti seré, por ti seré
\endverse

\endsong

%%%%%%%%%%%%%%%%%%%%%%%%%%%%%% La vache
\beginsong{La Vache}[by={Ricoune}]

\beginchorus
L\[Am]a vache, la vache
Quelle pute cette vache
La vache elle est barjot \textit{(ouais!)}
M\[Dm]oi je connais une v\[Am]ache qui dra\[E7]gue tous les taur\[Am]eaux
\endchorus

\beginverse
A\[Am]u milieu du troupeau, avec son petit veau
El\[E]le fait la belle
d\[Am]ès qu'il a tourné le dos
Elle s'enfuit dans les roseaux et la s\[E]aladelle
Elle va vers un monde n\[Dm]ouveau
D\[Am]ans l'espoir de rencontrer un t\[E]aureau
\endverse

\beginverse
Quand elle a fait sa course, elle revient au pré
Elle est très fatiguée
Mais ça ne l'empêche pas de garder la tête haute
Et toute sa fierté
Elle va vers un nouveau monde
Dans l'espoir de rencontrer un taureau
\endverse
\beginverse
Aux premières chaleurs
Quand les mâles sont ailleurs
Elle saute la clôture
Elle laisse ses copines
Parler mal de leurs voisines
Elle part à l'aventure
Elle va vers un monde nouveau
Dans l'espoir de rencontrer un taureau
\endverse

\endsong

%%%%%%%%%%%%%%%%%%%% L'Internationale

\beginsong{L'Internationale}[by=Eugène Potier, cr={1871}]

\beginverse
De\[G]bout, les damnés de la t\[G]erre
Deb\[D7]out, les forçats de la f\[G]aim
La r\[G]aison tonne en son c\[C]ratère,
C'est l'é\[D7]ruption de la f\[G]aim.
D\[D]u passé f\[A7]aisons table r\[D]ase,
F\[A7]oule esclave, debout, d\[D]ebout
Le m\[D]onde va\[D7] changer de b\[G]ase,
Nou\[D]s ne sommes rien, s\[A7]oyons t\[G]out.
\endverse

\beginchorus
C'est la l\[G]utte fi\[C]nale; Groupons \[D7]nous et de\[G]main
L'In\[G]ternati\[Em7]onale se\[C]ra le genre hum\[G]ain.
\endchorus

\beginverse
Il n'est pas de sauveurs suprêmes
Ni Dieu, ni César, ni Tribun,
Producteurs, sauvons-nous nous-mêmes
Décrétons le salut commun.
Pour que le voleur rende gorge,
Pour tirer l'esprit du cachot,
Soufflons nous-mêmes notre forge,
Battons le fer tant qu'il est chaud .
\endverse

\beginverse
L'État comprime et la Loi triche,
L'impôt saigne le malheureux;
Nul devoir ne s'impose au riche;
Le droit du pauvre est un mot creux
C'est assez languir en tutelle,
L'Égalité veut d'autres lois;
"Pas de droits sans devoirs, dit-elle
Égaux pas de devoirs sans droits."
\endverse

\beginverse
Hideux dans leur apothéose,
Les rois de la mine et du rail
Ont-ils jamais fait autre chose
Que dévaliser le travail?
Dans les coffres-forts de la banque
Ce qu'il a crée s'est fondu,
En décrétant qu'on le lui rende,
Le peuple ne veut que son dû.
\endverse

\beginverse
Les rois nous saoulaient de fumée,
Paix entre nous, guerre aux Tyrans
Appliquons la grève aux armées,
Crosse en l'air et rompons les rangs!
S'ils s'obstinent ces cannibales
A faire de nous des héros,
Ils sauront bientôt que nos balles
Sont pour nos propres généraux.
\endverse

\beginverse
Ouvriers, paysans, nous sommes
Le grand parti des travailleurs,
La terre n'appartient qu'aux hommes,
L'oisif ira loger ailleurs.
Combien de nos chairs se repaissent!
Mais si les corbeaux, les vautours,
Un de ces matins disparaissent,
Le soleil brillera toujours.
\endverse

\endsong



%%%%%%%%%%%%%%%%%%%%%%%% Fanchon
\beginsong{Fanchon}[by={Traditionnel\ français}, cr={1757}]

\beginverse
\[G]Am\[C]is, il \[G]faut faire une \[C]p\[G]au\[C]se, j'aperçois \[G]l'ombre d'un bouchon,
Buv\[G]ons à \[F]l'aimable Fan\[C]chon, chantons pour \[F]elle quelque ch\[D7]o\[G7]se.
\endverse

\beginchorus
\[C]Ah! Que son entretien est doux, qu'elle a de \[G]mérite et de \[C]gloire.
Elle \[G]aime à rire, elle aime à \[C]boire, elle \[G]aime à chanter comme \[C]nous
Elle aime à rire, elle aime à boire, elle aime à chanter comme nous
Elle \[C]aime à \[G]rire, elle aime à \[C]boire,
Elle \[C]aime à \[G]chanter comme \[C]nous, oui comme nous
Oui \[F]com\[G]me \[C]nous, \[G]oui comme \[G]nous !
\endchorus

\beginverse
Fanchon quoique bonne chrétienne, fut baptisée avec du vin,
Un bourguignon fut son parrain, une bretonne sa marraine.
\endverse

\beginverse
Fanchon préfère la grillade à d'autres mets plus délicats,
Son teint prend un nouvel éclat quand on lui verse une rasade.
\endverse

\beginverse
Fanchon ne se montre cruelle que lorsqu'on lui parle d'amour,
Mais moi je ne lui fais la cour que pour m'enivrer avec elle.
\endverse

\beginverse
Un jour le voisin La Grenade lui mit la main dans le corset,
Elle répondit par un soufflet sur le museau du camarade.
\endverse
\endsong


%%%%%%%%%%%%%%%%%%%%%%%%%%%%%% Cucurbitacé
\beginsong{Cucurbitacée}[by={Michel l'ingénieur\ informaticien}, cr={2009w}]

\transpose{0}

\beginverse
\[D]As-tu déjà vu mon cu-, mon cu-, mon \[A]cucurbita\[D]cée ?
\[D]Non, je n'ai jamais vu ton cu-, ton cu-, \[A]ton cucurbita\[D]cée
\[D]T'as déjà goûté mon cu-, mon cu-, mon \[A]cucurbita\[D]cée ?
\[D]J'ai jamais mangé ton cu-, ton cu-, ton \[A7]cucurbita\[D]cée.
\endverse

\beginchorus
\[D]Mon cu- cucurbita\[A7/G]cée, il est \[A7/C#]beau
\[A7]Mon cu- cucurbita\[D]cée, il est bien
\[D]Mon cu- cucurbita\[A7/G]cée, il sent \[A7/C#]bon
\[A7]Mon cu- cucurbita\[D]cée, \[A7]-ta\[D]cée
\endchorus

\beginverse
As-tu déjà vu mon cu-, mon cu-, mon cucurbitacée ?
Et il a quel goût ton cu-, ton cu-, ton cucurbitacée ?
Il est succulent, mon cu-, mon cu-, mon cucurbitacée
C'est vrai qu'il est beau ton cu-, ton cu-, ton cucurbitacée
\endverse

% \beginchorus
% Mon cu- cucurbitacée, il est beau
% Mon cu- cucurbitacée, il est bien
% Mon cu- cucurbitacée, il sent bon
% Mon cu- cucurbitacée, -tacée
% \endchorus

\beginverse
As-tu déjà vu mon cu-, mon cu-, mon cucurbitacée ?
Ah ça y est, je l'ai vu ton cu-, ton cu-, ton cucurbitacée
Eh bah, t'en penses quoi de mon cu-, mon cu-, mon cucurbitacée
J'suis un peu déçu de ton cu-, ton cu-, ton cucurbitacée
\endverse

\beginchorus
Mon cu- cucurbitacée, il est bof
Mon cu- cucurbitacée, bah il est bof
Mon cu- cucurbitacée, bah il est pas terrible
Mon cu- cucurbitacée, -tacée
\endchorus

\endsong


%%%%%%%%%%%%%%%%%%%%%%%%%%%%%%%%%%%%%%%%%%%%% France
\beginsong{France}[by={Kemar}, cr={2021}]
\beginchorus
\[G]J'aimerais que tout le \[D]monde se mette \[G]à danser
Car c'est la \[C]fête a\[D]près \[G]tout
\[G]J'aimerais que tout le \[D]monde se mette \[G]à danser
Car c'est la \[C]fête \[D]jusqu'au \[G]bout
\endchorus

\beginverse
Ce soir \[G]c'est la party de \[D]l'année
Tout le monde a envie de s'amu\[G]ser
Tout le monde a envie de s'éclat\[D]er
Dans les limites du raison\[Em]nable
\[G7]Il ne \[C]faut pas décon\[D]ner
Il y'au\[Gm]ra un petit peu \[F]d'alcool
Mais pas \[E&]trop, il ne faut pas e\[Cm]xagér\[D]er
\[G]Tout le monde veut faire la \[D]fête
\[G]Tout le monde veut \[D]s'ambiancer
\[G]Posé, tran\[G7]quille
\[C]On peut faire la \[D]fête
\[C]De mani\[G]ère totale\[D]ment raison\[G]nable
\endverse

\beginverse
Et c'est reparti pour un tour
Ça peut partir à tout moment sur une chenille
Et vas-y que je lève mon verre
Je lève mon verre à cette France qui fait plaisir à voir
Tout le monde apprécie le moment
Il y a des biscuits et aussi des enfants
C'est une soirée conviviale
Et si on faisait un Trivial Pursuit
J'adore ce jeu
Il permet d'évaluer
Les connaissances de tout à chacun
\endverse

\endsong



%%%%%%%%%%%%%%%%%%%%%%%%%%%%%%%%%%%%%%%%%
\beginsong{Crom}[by=Naheulband]
\beginverse
\[Em]Crom, c’est le dieu barbare de \[G]la baston,
\[Em]Crom, des mandales, des chtars, des \[G]gnons (oh oh oh)
\[Am]Crom, assis en haut de sa \[C]montagne
Quand \[F]les guerriers \[G]meurent, il ri\[Am]cane
\endverse

\beginverse
Crom, devenu dieu de ses propres mains
Crom, avec des techniques de bourrin (oh oh oh)
Crom, il a soloté tous les donjons
Sans armure et sans pantalon
\[Am]Stratégie, diplomatie, \[G]Crom n’a jamais rien com\[Am]pris
\[F]C’est la ba\[G]ston, et voi\[E]là !
\endverse
\beginverse
Crom, on en parle en mangeant du poulet (piou (×7))
Crom, dans les campements qui sentent les pieds (oh oh oh)
Crom, on le prie mais il n’écoute pas
Car il s’empiffre au Valhalla
S’équiper d’une grosse épée
Massacrer à tour de bras
Écraser ses ennemis
Les voir mourir devant soi
Crom, c’est ce qu’il veut, et voilà !
\endverse
\beginverse
Crom, a les bras comme des cuisses de taureau (oh oh oh)
Crom, n’utilise pas trop son cerveau (oh oh oh)
Crom, a traversé la mer en nageant
Avec son glaive entre les dents
\endverse
\beginverse
Crom, a coupé la tête au Léviathan (oh oh oh)
Crom, a fait pleurer Lara Fabian (oh oh oh)
Crom, a toujours épaté Chuck Norris
Il va bien plus loin quand il pisse !
\endverse
\beginverse*
\[Am]Crom, a le torse huilé
Crom, ne s’est jamais peigné
\[C]Crom, d’un seul regard, il peut \[F]tuer (oh oh oh)
Crom, a des filles à ses pieds
Crom, la main sur son épée
Crom, boit sa bière dans un crâne éclaté (oh oh oh)
\[Am]Il t’attend \[G6]au Valhal\[E]la
\endverse
\beginverse
Crom, a gardé le secret de l’acier
Crom, et si tu viens lui demander (oh oh oh)
Crom, mais où est donc le secret perdu ?
Il te répondra : dans ton cul !
\endverse
\endsong

%%%%%%%%%%%%%%%%%%%%%%%%%%%%%%%
\beginsong{La vie d'aventurier}[by=Naheulband]
\beginverse
Quand j’au\[G]rai mon niveau 2 j’a\[D]chèterai un du\[G]vet
Pour dormir dans les donjons, sans \[D]jamais m’enrhu\[G]mer
\endverse
\beginchorus
C’est \[G]un \[D]peu \[G]ça, la \[Am]vie d’aventu\[G]rier
Et \[G]on \[D]y \[G]va, y’a des \[D]gobelins à sai\[G]gner
\endchorus

\beginverse
Quand j’aurai mon niveau 3 j’achèterai un carquois
J’y mettrai des flèches magiques du genre qui t’arrache un bras
\endverse
\beginverse*
. . . y’a des squelettes à poutrer
\endverse

\beginverse
Quand j’aurai mon niveau 4 j’achèterai un cheval
Je sais pas vraiment monter, tant pis ça m’est égal !
\endverse
\beginverse*
. . . y’a des orques à décimer 
\endverse
\beginverse
Quand j’aurai mon niveau 5 j’achèterai des parchemins
Pour lancer des sortilèges même si j’suis pas magicien
\endverse
\beginverse*
. . . y’a des momies à cramer
\endverse
\beginverse
Quand j’aurai mon niveau 6 j’achèterai une armure
C’est moins facile de courir mais on évite les fractures
\endverse
\beginverse*
. . . y’a des trolls à bousiller
\endverse
\beginverse
Quand j’aurai mon niveau 7 j’achèterai un anneau
Qui donne +2 en charisme et la puissance d’un taureau
\endverse
\beginverse*
. . . y’a des minotaures à tuer
\endverse
\beginverse
Quand j’aurai mon niveau 8 j’achèterai une baliste
Pour assiéger les donjons des nécromancultistes
\endverse
\beginverse*
. . . y’a des liches à éventrer
\endverse
\beginverse
Quand j’au\[G]rai mon niveau 9 j’a\[Em]chèterai une vor\[G]pale
Pour ar\[Em]racher au dragon ses \[D]parties géni\[G]tales
\endverse
\beginverse*
. . . y’a un dragon pour tester
\endverse
\beginverse
Quand j’aurai mon niveau 10, j’achèterai des pansements
Pour corriger les erreurs du niveau précédent
\endverse
\beginchorus
C’est surtout ça, la vie d’aventurier
Je rentre chez moi, ma carrière est terminée
\endchorus
\endsong


%%%%%%%%%%%%%%%%%%%%%%%%%%%%% Pena baoina
\beginsong{Peña baiona}[cr={2019}]

\beginverse
Dans notre \[Dm]cher petit Bayonne, il est une Peña, la \[Am7]Peña Baion\[Dm]a
Ils portent \[Dm]fièrement partout leurs foulards bleus et blancs
À Dax ou \[F]à Narbonne, on ne voit plus que ces gars-\[C]là
Qui ont dans \[Dm]l'cœur leurs chers \[Am7]joueurs du rugby \[Dm]roi
Chez nous à \[Dm]Jean Dauger ou bien partout à l'extérieur
Sur tous les \[F]stades enfiévrés, elle nous met tant d'ar\[C]deur
C'est la Pe\[Dm]ña qui crie sa \[Am7]joie sur cet air-\[Dm]là
\endverse

\beginchorus
\[B&]Allez, allez! Les bleus et blancs de l'Avir\[F]on Bayonnais
C'est la Peña, c'est la Pe\[C7]ña Baiona
On est tous là, allez les \[F]gars, encore une \[F7]fois
\[B&]Allez, allez! Les bleus et blancs de l'Avir\[F]on Bayonnais
Jouez au ras puis écar\[C]tez, c'est l'essai
On applaudit à vos expl\[Dm]oits, \[Am7]c'est gag\[Dm]né
\endchorus

\beginverse
Dans notre cher petit Bayonne, il est une Peña Baiona
Ils portent fièrement partout leurs foulards bleus et blancs
À Dax ou à Narbonne, on ne voit plus que ces gars-là
Qui ont dans l'cœur leurs chers joueurs du rugby roi
Partout nos joueurs brillent souvent, la victoire est au bout
Vêtus de bleu et blanc s'habillent et la fête est partout
C'est la Peña qui crie sa joie sur cet air-là
\endverse

\beginverse*
Dans notre \[Dm]cher petit Bayonne, il est une peña, la pe\[Am7]ña baion\[Dm]a
\endverse

\endsong


%%%%%%%%%%%%  Dans les prisons de Nantes
\beginsong{Dans les prisons de Nantes}
\beginverse
\[Em]Dans les prisons de Nan\[D]tes
L'ann didou didou d'ann
Didou di l'ann di
L'ann didou didou d'ann
\[Em]Dans les prisons de Nantes
Y avait un \[G]pri\[D]son\[Dm]nier
Y avait un prisonnier
\endverse

\beginverse
Personne ne le vint voir ... Que la fille du geôlier.\rep{2}
Un jour il lui demande ... Et que dit-on de moi ?\rep{2}
On dit de vous en ville ... Que vous serez pendu.\rep{2}
Mais s'il faut qu'on me pende ... Déliez-moi les pieds.\rep{2}
La fille était jeunette ... Les pieds lui a déliés.\rep{2}
Le prisonnier alerte ... Dans la Loire s'est jeté.\rep{2}
Dès qu'il fût sur les rives ... Il se mit à chanter.\rep{2}
Je chante pour les filles ... Surtout celles des geôliers.\rep{2}
Si je reviens à Nantes ... Oui je l'épouserai.\rep{2}
Dans les prisons de Nantes ... Y avait un prisonnier.\rep{2}
\endverse
\endsong



%%%%%%%%%%%%%%%%%%%%%%%%%%%% Le rigodon de Bresse

\beginsong{Le rigodon}[by={Courant d'Eire}]
\beginchorus
\[Dm]Nous on aime bien \[F]mieux dan\[C]ser le rigo\[Am]don de Bresse
\[Dm]Les jeunes et les \[F]vieux au \[C]sortir de la \[Am]messe
\[Dm]C'est notre \[F]façon à \[C]nous de rem\[Am]uer les fesses
\[Dm]On n'a pas d'le\[F]çon à \[C]prendre du show bus\[Am]iness
\endchorus

\beginverse 
\[F]On sait qu'il y a en Orient \[C]des danseuses du \[F]ventre
\[F]Qui sur des sons discordants \[C]le sortent et le \[Dm]rentrent
\[F]Mais dans le bourg de Viriat \[C]quand ils font la \[F]vougue
\[F]Toutes les filles qui sont là \[C]affirment avec \[Am]fougue
\endverse
 
\beginverse
Paraît qu'aux Etats Unis où l'on craint personne
A force de rouler la caisse on trouve qu'ils déconnent
Nous on n'a pas de revolver quand on mène les vaches
La Metro Goldwin Mayer faudrait qu'ils nous lâchent
\endverse
 
\beginverse
Quand ils ont bu d'la vodka les russes ont la pêche
ça les aide à oublier qu'ils sont dans la dèche
Mais le vin de baragnon celui qui nous saoule
Est le seul autorisé près des cages à poules
\endverse

\beginverse
Autrefois les Africains nous fichaient la frousse
Dès qu'on entendait au loin le tam-tam de brousse
Mais les gars des Mepillats sans qu'on les y pousse
Sifflent bien plus fort que ça les brunes et les rousses
\endverse
 
\beginverse
Avec son béret vissé au sommet d'la tête
Tu t'rappelles de ton pépé quand il faisait la fête
Et tu n'es pas loin d'penser tout comme ton ancêtre
Que ce s'rait bien mieux d'rester où l'on t'a vu naître
\endverse
\endsong


%%%%%%%%%%%%%%% Entendez-vous dans le feu
\beginsong{Entendez-vous dans le feu (canon)}[by={Traditionnel}]
\beginverse*
\[C]Enten\[G]dez-vous \[C]dans le feu,
Tous ces bruits mystérieux?
Ce sont les tisons qui chantent.
Que nos cœurs soient joyeux.
\endverse
\endsong

%%%%%%%%%%%%%% Terre rouge
\beginsong{Terre rouge (canon)}[]
\beginverse*
\[Am]Terre rouge, \[G]terre de \[Am]feu
^Terre, terre, ^terre de ^lumière
^Terre rouge ^sous le ciel ^bleu
\endverse
\endsong





%%%%%%%%%%%%%%%%%%%%%%%%%%%%%%%%%%%
\beginsong{Bombom Stand}[by={Bredeler},cr=2006]
\beginverse
\[C]Mini mini Mama het a Bombomstand,
\[G]A Bombomstand,\[C] a Bombomstand.
Mine mine Pabbe het a kochleffel in der Hand
Un schlaujt min’ra Mama auf der Bombomstand.\rep{2}
\endverse

\beginverse*
Un I wett, un I wett, un I wett met dir,
In Strossburi gett’s ke Jung frau meht (bis).
\[C]Bawela \[F]Bawela \[G]jetzt gett’s \[C]los.
\endverse

\beginverse
\[B]S’Marigel kommt ge fahre
Mit sine krumme Wade
Zim zam zum, in a Latte Zün a nie
Un \[F#]direkt mit d’r Fratz in a \[B]Mischthüfa nie.
\endverse

\beginverse
\[B]Mini mini Mama het a Bombomstand,
\[F#]A Bombomstand, \[B]a Bombomstand.
\[B]Mine mine Pabbe het a kochleffel in der Hand
\[F#]Un schlaujt min’ra Mama uff der\[B] Bombomstand.
\endverse

\beginverse*
Holiép, holiép, holiép met dir
In Strossburi gett’s ke Jung frau meht \rep{2}
\endverse

\endsong


%%%%%%%%%%%%%%%%%%%%%%%%%%%%%%%%%
\beginsong{Hans in Schnòckeloch}[by={Bredelers}]

\transpose{-5}
\beginverse
D'r \[G]Hans im Schnòckeloch hät \[D]àlles, wàs er \[G]will !
D'r Hans im Schnòckeloch hät àlles, wàs er will !
Un \[D7]wàs er hät, dess \[G]will er nit,
Un wàs er will, dess hät er nit.
D'r \[G]Hans im Schnòckeloch hät \[D]àlles, wàs er \[G]will !
\endverse
\beginverse
D'r Hans im Schnòckeloch sajt àlles, wàs er will !
D'r Hans im Schnòckeloch sajt àlles, wàs er will !
Wàs er sajt, dess dankt er nit,
Un wàs er dankt, dess sajt er nit,
D'r Hans im Schnòckeloch sajt àlles, wàs er will !
\endverse
\beginverse
D'r Hans im Schnòckeloch màcht àlles, wàs er will !
D'r Hans im Schnòckeloch màcht àlles, wàs er will !
Wàs er màcht, dess soll er nit,
Un wàs er soll, dess màcht er nit.
D'r Hans im Schnòckeloch màcht àlles, wàs er will !
\endverse
\beginverse
D'r Hans im Schnòckeloch geht ànne, wo er will !
D'r Hans im Schnòckeloch geht ànne, wo er will !
Wo er isch, dò bliebt er nit,
Un wo er bliebt, dò gfàllts im nit.
D'r Hans im Schnòckeloch geht ànne, wo er will !
\endverse
\beginverse
Jetzt het d'r Hans sò sàtt
Un isch vom Eland màtt.
Lawe, majnt er, kànn er nit,
Un sterwe, sajt er, will er nit.
Er springt züem Fenschter nüss,
Un kommt ins Nàrrehüss. 
\endverse
\endsong

%%%%%%%%%%%%%%%%%%%%%%%%%%%%%%%%%
\beginsong{Albert}[by={Patrick Breitel}, cr={2010}]
\beginchorus
Je m'ap\[A]pelle Albert, j'habite \[E]là derrière dans la \[B7]vallée de Muns\[E]ter \rep{2}
Ya, la, \[A]la, pa, pa,  ya, la, \[E]la, pa, pa,  ya, la, \[B7]la, la, la, la, la
\endchorus

\beginverse
chaque sam'\[E]di après-mi\[B7]di, avec ma femme Ger\[E]maine
on va \[E]remplir le ca\[B7]ddy; on achète pour la se\[E]maine
on a \[A]mis notre jogging; on va \[E]faire notre shopping
on ren\[B7]contre les copains; on se \[E]raconte les potins
les \[A]nouvelles du village et les \[E]histoires de ménage
après \[F#7]on s'en va, SALUT ! Que c'est \[B7]chouette au SUPER U !
\endverse

\beginverse
Avec mes copains pompiers, quand on fait des exercices
on n'arrête pas d'arroser, à l'eau, à la bière, au pastis
mais moi, ça que je préfère, c'est notre kilt en plein air
quand on fait valser les filles, faut voir leurs yeux comme ils brillent
c'est pas tous les jours qu'un homme dans un joli uniforme
les fait tourner, quel régal, surtout qu'je suis caporal
\endverse
\beginverse
Tous les dimanche matin, pendant que ma femme prend son bain
je prépare mon tiercé en avalant mon café
puis je l'attends pour descendre, elle met sa robe de chambre
un fichu sur les bigoudis, elle emmène le chien faire pipi
moi, je prend la mobylette et je m'en vais au bistro
ma femme rouspète et ça pète quand j'ai bu un coup de trop
\endverse
\beginverse
En décembre sur Noël, j'mets partout des p'tites lumières
la maison, elle est si belle, de devant, de côté, de derrière
pére Noël qui grimpe au mur, des guirlandes à la clôture
aux gouttières, le long du toit, au balcon en-haut, en bas
y a Bambi dans le jardin, et blanche neige et les sept nains
comme pour les maisons fleuries, on reçoit le premier prix !
\endverse
\beginverse
Au printemps quand tout est vert, j'emmène toute la famille
on va jusqu'à Gerardmer à la fête des jonquilles
ma femme fait pour le quartier des bouquets plein le panier
et moi, j'enfile des fleurs autour des enjoliveurs
aux portières et au capot, à l'antenne et au rétro
quelle est jolie la voiture dans sa jaune garniture !
\endverse
\beginverse
En juillet pour les congés, on aime bien voyager
on accroche la caravane; on s'arrête juste à la douane
on roule jusqu'en Italie, terminus à Rimini
au camping de la plage, c'est presque comme au village
on installe le Butagaz, on sort notre MUNSTERKÄSS
on retrouve les voisins, c'est le quartier alsacien
\endverse

\endsong


% \beginsong{Ah tu sortiras biquette}[by=Traditionnel\ français]
% \beginchorus
% \[G]Ah ! Tu sortir\[D]as, Biquette, Bi\[G]quette,
% Ah ! Tu sortiras de ce chou-là
% Ah ! Tu sortiras, Biquette, Biquette,
% Ah ! Tu sortiras de ce chou-là
% \endchorus

% \beginverse
% \[D]On envoie chercher le chien, \[G](bis)
% \[D]Afin de mordre Biquette. \[G](bis)
% \[D7]Le chien ne veut pas mordre Biquette.
% Biquette ne veut pas sortir du chou.
% \endverse

% \beginverse
% On envoie chercher le loup, (bis)
% Afin de manger le chien. (bis)
% Le loup ne veut pas manger le chien.
% Le chien ne veut pas mordre Biquette.
% Biquette ne veut pas sortir du chou.
% \endverse

% \beginverse
% On envoie chercher l’bâton, (bis)
% Afin d’assommer le loup. (bis)
% Le bâton n’veut pas assommer le loup.
% Le loup ne veut pas manger le chien.
% Le chien ne veut pas mordre Biquette.
% Biquette ne veut pas sortir du chou.
% \endverse

% \beginverse
% On envoie chercher le feu, (bis)
% Afin de brûler l’bâton. (bis)
% Le feu ne veut pas brûler le bâton.
% Le bâton n’veut pas assommer le loup.
% Le loup ne veut pas manger le chien.
% Le chien ne veut pas mordre Biquette.
% Biquette ne veut pas sortir du chou.
% \endverse

% \beginverse
% On envoie chercher de l’eau, (bis)
% Afin d’éteindre le feu. (bis)
% L’eau ne veut pas éteindre le feu.
% Le feu ne veut pas brûler le bâton.
% Le bâton n’veut pas assommer le loup.
% Le loup ne veut pas manger le chien.
% Le chien ne veut pas mordre Biquette.
% Biquette ne veut pas sortir du chou.
% \endverse

% \beginverse
% On envoie chercher le veau, (bis)
% Pour lui faire boire l’eau. (bis)
% Le veau ne veut pas boire de l’eau.
% L’eau ne veut pas éteindre le feu.
% Le feu ne veut pas brûler le bâton.
% Le bâton n’veut pas assommer le loup.
% Le loup ne veut pas manger le chien.
% Le chien ne veut pas mordre Biquette.
% Biquette ne veut pas sortir du chou.
% \endverse

% \beginverse
% On envoie chercher l’boucher, (bis)
% Afin de tuer le veau. (bis)
% Le boucher n’veut pas tuer le veau.
% Le veau ne veut pas boire de l’eau.
% L’eau ne veut pas éteindre le feu.
% Le feu ne veut pas brûler le bâton.
% Le bâton n’veut pas assommer le loup.
% Le loup ne veut pas manger le chien.
% Le chien ne veut pas mordre Biquette.
% Biquette ne veut pas sortir du chou.
% \endverse

% \beginverse
% On envoie chercher le diable, (bis)
% Pour qu’il emporte le boucher. (bis)
% Le diable veut bien prendre l’boucher.
% Le boucher veut bien tuer le veau.
% Le veau veut bien boire l’eau.
% L’eau veut bien éteindre le feu.
% Le feu veut bien brûler le bâton.
% Le bâton veut bien assommer le loup.
% Le loup veut bien manger le chien.
% Le chien veut bien mordre Biquette.
% Biquette veut bien sortir du chou !
% \endverse

%\endsong



%%%%%%%%%%%%%%%%%%%%%%%%%% Ća m'vénère
\beginsong{Ca m'vénère}[by={Grims}]
\beginverse*
\nolyrics Accords : \[Am] \[F] \[C] \[G] 
\endverse

\beginverse
(parlé) y a trop d'trucs qui m'énervent sur terre
les chômeurs sans travail, les orphelins sans père
plus rien ne tourne rond et personne veut l'admettre
et ça m'vénère... dis-leur, Maître ! 
\hspace{.1cm}
Ça m'vénère!
quand je marche en chaussettes dans la salle de bain
et qu'y a de l'eau partout, j'fais ça tous les matins
je tire mal mon rideau qui a des motifs indiens
oh ça m'vénère, oui ça m'vénère
\endverse

\beginverse
(parlé) ok Maître, j'te comprends mais j'pensais pas à ça
plus aux problèmes de la jeunesse, tu vois les trucs comme ça
tous les échecs scolaires, les familles déchirées
ça m'vénère, dis-leur Maître, j'peux plus respirer
\hspace{.1cm}
Ça m'vénère
quand je me cogne le doigt d'pied sur le coin du lit
quand je commande un cheese cake et qu'y a pas d'coulis
quand j'repasse mes T-shirts et que je fais des plis
oh ça m'vénère, oui ça m'vénère
\endverse


\beginverse
(parlé) t'as raison mais une fois de plus, c'est pas c'que j'voulais dire
genre les s.d.f. sans toit qui n'savent pas où dormir
ces jeunes qui vendent de la drogue et qui deviennent dealers
allez frère, j'compte sur toi, cette fois-ci, dis-leur !
\hspace{.1cm}
Ça m'vénère
quand j'conduis et qu'un voyant que j'connais pas s'allume
quand Évelyne Dhéliat dit qu'demain y a d'la brume
quand je souffle les bougies magiques et qu'elles s'rallument
oh ça m'vénère, oui ça m'vénère
quand mes écouteurs sont tout emmêlés - oooh
quand j'ai un Kinder, qu'j'sais pas construire le jouet
quand il fait chaud et qu'mes lunettes sont pleines de buée
oh ça m'vénère oui ça m'vénère
\endverse

\beginverse
(parlé) écoute, j'suis d'accord, mais j'vais r'centrer un peu
j'vais parler des galères des grands et des ti-peu
\hspace{.1cm}
Ça m'vé-nère
quand j'ai du Scotch, qu'j'trouve pas l'bout du rouleau
quand quelqu'un range une bouteille vide dans le Frigo
quand je vais à Disney et qu'je vois pas Dingo
oh ça m'vénère, oui ça m'vénère
quand j'fais des mots fléchés et qu'un mot rentre pas
quand j'entends le refrain de la chanson "Bella"
ah oui mais non c'est vrai que cette chanson, c'est moi...
oh j'me vénère, j'crois qu'j'me vénère
\endverse

\endsong


%%%%%%%%%%%%%%%%%%%% Demouler un cake
\beginsong{Démouler un cake}[by={Jérome Niel}, cr={2024}]
\beginverse
Je \[F]viens de mettre au point une re\[B&]cette
Que \[C]beaucoup d'entre vous vont ado\[F]rer
On \[F]oublie les légumes, y a pas d'cour\[B&]gettes
Par \[C]contre y a d'la pur\[F]ée
\endverse

\beginchorus
\[A7]L'objectif il est \[Dm]super clair, \[G]il suffit d'ava\[C]ler 
\[A7]Tout c'qu'il y a dans ton \[Dm]frigidaire \[G]et y a \[G7]plus qu'à y aller
\[C]Démouler un cake, \[Dm]démouler un cake
\[G]Démouler un cake à la \[C]merde
J'vais \[C]démouler un cake, \[Dm7]démouler un cake
\[G]Et si ça t'fait chier, bah j't'em\[C]merde, grosse merde
\endchorus

\beginverse
J'adore quand tout se mélange dans mon bide
Pizza, osso bucco et rhum Coca 
Mon trou, il est tout chaud, il est acide
Et il attend que ça !
\endverse

\beginchorus
Démouler un cake, un cake à la merde (bis)
\endchorus

\endsong


%%%%%%%%%%%%%%%%%%% J'ai mangé un croissant
\beginsong{J'ai mangé un croissant}[by={David Fils\ de\ Momone},cr=2023]

\beginverse
\[G]J'ai mangé un croissant, et maintenant il est absent
\[Em]J'ai mangé un croissant, \[D]et maintenant il est absent
\endverse

\beginverse
\[G]J'ai mangé un kiwi, et maintenant il est parti
\[Em]J'ai mangé un kiwi, \[D]et maintenant il est parti
\endverse

\endsong

%%%%%%%%%%%%%%%%%%%%%%%%%%%%%%%%%%%%%%%%%%%%%%%%%%%%%%
\beginsong{J'ai mangé une pomme et j'ai plein d'énergie}[by={David Fils\ de\ Momone},cr=2023]

\beginverse*
\[Am]Ce matin, \[G]je me lève et j'ai du chagrin
\[Am]J'ai très peu \[G]d'énergie, j'ai envie d'manger une pomme
\endverse

\beginchorus
\[C]J'ai mangé une pomme et j'ai \[G]plein d'énergie
J'ai mangé une pomme et j'ai \[C]plein d'énergie 
\endchorus

\endsong

%%%%%%%%%%%%%%%% Le saucisson
\beginsong{Je veux du saucisson}[by={Elias Campredon}, cr=2021]

\beginverse
Oui, j'veux du bon \[F]saucis\[C]son, non pas \[G7]d'la contre-fa\[Am7]çon
De la très \[F]bonne fac\[C]ture: Oui, de \[G7]la matière \[C]sûre.
\endverse

\beginchorus
J'veux du \[F]sauci\[C]sson, j'veux du \[G7]saucis\[Am]son.
J'veux du \[F]saucis\[C]son, je veux \[G7]du sauci\[C]sson!
\endchorus

\beginverse
Emballage en plastique, ou bien en papier carton.
Pas b'soin d'faire l'arithmétique: oui, je veux du saucisson.
\endverse

\beginverse
Avec des herbes ou du poivre, ou plus classique de la peau,
J'adore en manger le soir, c'est inné, c'est dans ma peau!
\endverse
\beginverse*
Avec plein ou peu de gras, peu importe tout me va,
Tant qu'il y a du saucisson: Je suis complètement à fond!
\endverse

\beginverse
Le seul et unique saucisson, c'est vraiment beaucoup trop bon,
A chaque bout j'me sens mieux parce que c'est trop délicieux!
\endverse
\beginverse*
Alors je vous remercie, même si vous êtes aigri
Vous êtes vraiment trop sympa d'aider un gars comme moi!
\endverse
\endsong


%%%%%%%%%%%%%%%%%%%%%%%%%%%%%%%%%%% Bon anniversaire les ptits indiens
\beginsong{Bon anniversaire les p'tits Indiens}[by={Paul Glaeser}, cr=2001]
\transpose{-3}
\beginchorus
\[C]Bon anniversaire les p'tits In\[G]diens ("Chez Buffalo")
\[C]Bon anniversaire les p'tits cow-\[C7]boys ("Chez Buffalo")
Venez \[F]chanter et dan\[C]ser a\[G]vec tous vos co\[Am]pains
Ici \[F]tout est per\[G7]mis, on s'amuse \[C]bien
\endchorus

\beginchorus
[...] Allumons les bougies sur le grand gâteau
Laissons faire la magie, c'est l'heure des cadeaux
\endchorus

\beginverse
Les gar\[C]çons, les filles ont les \[G]yeux qui brillent
Devant \[F]les desserts choco\[C]lat-vanille
Voilà les cookies made in \[G]Buffalo
Plein de \[Dm7]gourmandises pour ma \[G7]jolie squaw
\endverse

\beginchorus
[...] Sur un air de banjo et d'harmonica
Chantons Happy Birthday encore une fois
\endchorus

\beginverse
Comme tous les ans, en ce jour de fête
Souhaitons du bonheur à tous les enfants
Qu'on habite un ranch ou dans un tipi
Sous le grand totem on se réunit
\endverse

\beginchorus
[...] Tapez des pieds et des mains au prochain refrain
Et faites "Bouloulou" comme de vrais Indiens
La, la, la ...
\endchorus

\endsong




