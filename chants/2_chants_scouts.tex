
%%%%%%%%%%%%%%%%%%%%% Le chant de la promesse
\beginsong{Chant de la promesse}[by={Jacques Sevin}, cr={1921}]
\beginverse
\[D]Devant tous je m'engage, \[Bm]sur \[E7]mon hon\[A4]neur,\[A]
Et \[D]je te fais hommage, \[G]de \[A]moi Seig\[D]neur!
\endverse

\beginchorus
Je \[G]veux t'aimer sans \[D]cesse, \[Em]de \[A]plus en \[D4]plus, \[D]
Pro\[G]tège ma pro\[D ]messe, \[G]Seig\[A]neur Jé\[D]sus!
\endchorus

\beginverse
Je jure de te suivre, en fier chrétien,
Et tout entier je livre, mon cœur au Tien.
\endverse

\beginverse
Fidèle à ma Patrie, je le serai;
Tous les jours de ma vie, je servirai.
\endverse

\beginverse
Je suis de tes apôtres, et chaque jour,
Je veux aider les autres, pour ton amour.
\endverse

\beginverse
Ta Règle a sur nous-mêmes, un droit sacré;
Je suis faible, tu m'aimes: je maintiendrai!
\endverse
\endsong

%%%%%%%%%%%%%%%%%%%%% Chant de promesse (5ème)
\beginsong{Chant de promesse (5ème)}[by={Clément Baumann}, cr={2009}]
\beginverse
\[D]Ce soir nous sommes \[Bm]rassemblés pour \[G]célébrer et \[A]s’engager
\[D]Notre loi à tous nous fait \[Bm]avancer, c’est \[G]notre fier\[A]té!
Laisse \[D]toi habiter par la \[Bm]sérénité et \[G]tends une main de fraterni\[A]té
La pro\[D]messe que tu \[Bm]souhaites prononcer, elle \[G]permettra de te gui\[A]der!
\endverse

\beginchorus
Aujourd’hui je veux chanter, ne pas rester muet,
Je veux parler pour dire la vérité, je veux changer et toujours progresser
Pour la vie je veux rester fidèle et engagé,
Je veux marcher, ne pas rester courbé et sillonner un chemin d’amitié!
\endchorus

\beginverse
Ce soir nous sommes rassemblés pour célébrer et s’engager
Rien n’est différent mais tout a changé, un mystère nous a transformés
Maintenant va semer la graine qu’on t’a donnée pour en récolter des milliers
Et avec tous les scouts du monde entier nous marcherons à tes côtés.
\endverse
\endsong


%%%%%%%%%%%%%%%%%%% Le monsieur en chemise

\beginsong{Le monsieur en chemise}[by={Marc Durand}, cr={2013}]

\beginverse
\[Am]Je suis jaune, j’ai 8 ans et j’ai peur sans ma ma\[C]man,
j’sais rien faire, j’pleure tout le temps et j’me d’mande ce qui m’att\[F]end,
y a un m’sieur qui m’regarde du haut de ses 19 \[C]ans
qui me dit  "t’inquiète pas tu vas kiffer pour long\[G]temps"
\endverse

\beginchorus
Et il av\[F]ait rai\[C]son le mon\[G]sieur, le monsieur en che\[C]mise
il a\[F]vait rai\[C]son de me \[G]faire un p’tit peu la le\[C]çon.
\endchorus

\beginverse
Je suis bleu, j’ai 12 ans, et j’ai plus peur sans maman
entouré de copains, ils ne sont jamais bien loin
j’ai appris j’ai grandi, j’ai découvert la vraie vie
et l’monsieur qui me dit, "tu vas voir c’est pas fini"
\endverse

\beginverse
Je suis rouge, j’ai 15 ans et j’kiffe quand y a plus maman
entouré des mêmes gens qui avait quand j’avais 8 ans
y a un m’sieur qui m’regarde du haut de ses 25 ans
qui me dit "mon p’tit gars t’es en train de dev’nir grand"
\endverse

\beginverse
Je suis vert, j’suis majeur et j’pense même plus à ma mère
à bâtir des projets, à partir à l’étranger
y a un m’sieur qui me dit "après ça c’est terminé"
je lui dis "sur’ment pas à mon tour de faire rêver"
\endverse
\endsong




%%%%%%%%%%%%%%%%%%%%%%%%%%%%%%%%%%%%%%% Des couleurs sur mon chemin
\beginsong{Des couleurs sur mon chemin}[by={Amplitude}, cr={2010}]

\beginverse
\[F#m]Tant de moments parta\[A]gés
\[D]Les beaux jours passés ens\[A]embl\[B]e
\[F#m]Des sourires par mill\[A]iers
\[D]Des délires qui nous ress\[A]emb\[E]lent
\[F#m A B F#m F#m A B F#m]Des couleurs sur mon chemin ! \rep{8}
\endverse

\beginchorus
\[E] Une \[F#m]chemise un \[D]foulard et mon \[A]sac sur le \[C#m]dos
\[D]La pro\[A]messe, d'une \[Bm]main qui t'aide à passer la rivière
Une \[F#m]chemise un \[D]foulard et mon \[A]sac sur ton \[C#m]dos
\[D]Le scou\[A]tisme dans la \[C#m]peau !
\endchorus

\beginverse
On échange nos trésors, des rencontres sur la route
Une gourde au bout de l'effort, les étoiles en clef de voûte
Des couleurs sur mon chemin ! \rep{8}
\endverse

\beginverse
Lorsque les épaules souffrent on s’assoit près du chemin
L'amour donne un deuxième souffle quand on partage le pain
Des couleurs sur mon chemin ! \rep{8}
\endverse

\beginverse
Les yeux plongés dans le ciel, une tente et de l'air pur
Une cour sans frontières généreuse est la nature
Des couleurs sur mon chemin ! \rep{8}
\endverse
\endsong

% %%%%%%%%%%%%%%%%%%%%%%%%%% la scoutance
% \beginsong{La scoutance}[by={Le Cinquième},cr=2022]
% \beginverse
% \[Em] \[G] \[D] \[C]
% Chez les scouts
% J'y passe le plus clair de mon temps
% En réunion entre potes
% En week-end ou au camp
% Nous on kiffe, on s'en fout du regard des gens
% Moi je suis scout un jour, et heureux tout le temps
% Sapé comme jamais
% Ma chemise me rend cool, même s'il manque un insigne
% Et qu'elle est toujours en boule
% Le foulard autour du cou, toujours bien noué
% Car j'ai maté un tuto, pour faire le nœud carré
% Avec mon équipe, monter la tente me rend fier
% Même avec un bout de bois, pour caler la faîtière
% Je construit des tables, je construit des bancs
% Mais une fois sur deux
% Il ne tiennent pas jusqu'à la fin du camps
% Désormais plus besoin, d'aller à la salle
% Je me muscle en portant les jerricanes et les mâles
% Passer une journée, à creuser des feuillées
% Même si au final, tout le monde va pisser dans la forêt
% J'ai fait des gamelles, des thèques, des sioules
% J'ai chanté à tue tête, la scoutance et le matou
% Quand je repense, à tous les souvenirs que j'ai avec vous
% Je ne pense pas à moi, mais je pense à nous
% \endverse

% \beginchorus
% Je parle de souvenir, mais si t'es pas scout tu peux pas connaitre
% Je parle de souvenir, mais si t'es pas scout
% J'vais repartir en camp
% Revenir tous les ans
% Même quand je serai grand
% Promesse de scout
% J'vais repartir en camp
% Revenir tous les ans
% Même quand je serai grand
% Promesse de scout
% J'vais repartir en camp
% Promesse de scout
% \endchorus

% \beginverse
% Parfois j'ai pas de chance, pour la météo
% Concu sous la flotte, canicule en explo
% Je garde le sourire, même s'il fait pas chaud
% J'allume un p'tit feu, j'sors les chamallows (Hé hé)
% Je voudrais dormir, je suis épuisé
% Car depuis 6h du mat, les louveteaux sont réveillés
% Pour les imaginaires, je sais improviser
% Un drap et un sac poubelle, et me voilà déguisé
% J'connais le chant de la promesse par cœur
% Mes calendriers trouvent toujours preneur
% J'ai inventé un jeu, pour une réunion
% C'est un poule-renard-vipère, où j'ai juste changé le nom
% Ici on mange, mieux qu'au restau
% On mélange du éco+, avec des produits locaux
% On cuisine tout, au feu de bois
% Petit Bonus: j'ai d'la cendre dans tous mes plats
% Essayer de sortir sans se faire griller la nuit
% Sentir le feu de bois, pendant la moitié de ma vie
% Pas réussir à mettre tous dans mon sac à la fin du camp
% Être nostalgique de ces souvenir de ces moments
% J'pensais que pendant le camp, j'avais bronzé
% (Le monde m'appelle)
% Mais après deux douche, la crasse s'est enlevée
% (Le monde m'attend)
% \endverse

% \beginverse*
% Une chemise, un foulard
% Et mon sac sur le dos
% Ma promesse
% D'une mais qui t'aide à passer la rivière
% J'ai appris, j'ai grandi, j'ai découvert la vraie vie
% Ensemble on va plus loin, tu connais déjà le refrain
% \endverse*

% \beginchorus
% Je parle de souvenir, mais si t'es pas scout tu peux pas connaître
% Je parle de souvenir, mais si t'es pas scout tu peux pas connaître
% Je parle de souvenir, mais si t'es pas scout tu peux pas connaître
% Je parle de souvenir, mais si t'es pas scout
% J'vais repartir en camp
% Revenir tous les ans
% Même quand je serai grand
% Promesse de scout
% J'vais repartir en camp
% Revenir tous les ans
% Même quand je serai grand
% Promesse de scout
% \endchorus
% \endsong

%%%%%%%%%%%%%%%%%%%%%%%%%%
\beginsong{Red River Valley}[by={François Lebouteux},cr=1946]
\beginverse
Les pion\[G]niers sont pas\[C]sés avant \[G]le jour
Dans les rues du village acca\[A]blé \[D]
Et mon \[G]cœur a fré\[G7]mi à leur \[C]pas lourd
Sur les \[G]bords de la Red \[D]River \[G]Valley
\endverse

\beginchorus
O Seigneur la roue tourne entre tes mains
Où je vais aujourd’hui je ne sais
O Seigneur la roue tourne entre tes mains
Mais je veux retrouver les pionniers
\endchorus

\beginverse
Les pionniers ont peiné pour le village
A leurs mains la vallée s’est pliée
Et mes yeux ont vu naître un barrage
Sur les bords de la Red River Valley
\endverse

\beginverse
Les pionniers ont marqué dans la clairière
Que le pain se partage entre tous
Et ma main s’est ouverte à mes frères
Sur les bords de la Red River Valley
\endverse

\beginverse
Les pionniers ont chanté dans la nuit claire
Que la terre est à qui la voulait
Et ma voix s’est unie à leur chant fier,
Sur les bords de la Red river Valley
\endverse

\beginverse
Les pionniers ont promis de revenir
L’herbe pousse aujourd’hui à nos pieds
Et mon cœur s’est trouvé fait pour servir
Sur les bords de la Red River Valley
\endverse
\endsong



%%%%%%%%%%T'en fait pas la vie est belle
\beginsong{T'en fais pas la vie est belle}[by={Jamboree},cr=2021]

\beginverse
T'en fais p\[C]as la vie est b\[G]elle
Comme un v\[Am]ol d'hironde\[Em]lles
Qui s'en v\[F]a et qui rev\[C]ient
Au print\[Dm]emps, un beau mat\[G]in
\endverse

\endsong

%%%%%%%%%%%Viens grandir avec les Farfadets

\beginsong{Viens grandir avec les farfardets}[by={Romain Avice et Sylvain Bréant},cr=2023]

\beginverse
D\[C]ans la nature, je vis de be\[Am]lles aventures
J\[F]’éveille mes sens, exp\[C]ér\[G]ience
J\[C]e cours, je ris, avec mes br\[Am]as je construis
F\[F]ille ou garçon, cré\[A]ati\[G]on
Ran\[C]do fatigante, j’entends les o\[Am]iseaux qui chantent,
Que c’\[F]est beau la vie, je s\[C]ouri\[G]s !
Q\[C]uestions en tête, en faisa\[Am]nt des pirouettes,
P\[F]as d’Internet, c’est l\[C]a f\[G]ête
T\[Dm]om et Prune sont nos am\[G]is
E\[Dm]n vert pomme, réun\[G]is
\endverse

\beginchorus
V\[C]iens grandir avec les F\[G]arfadets
L’\[Am]épopée d’la ronde est e\[Em]n projet
L\[F]es parents plantent le déc\[G]or
T\[F]om et Prune sortent le tré\[C]sor, \[G]hey !
V\[C]iens grandir avec les F\[G]arfadets
N’\[Am]oublie pas d’apporter t\[Em]on carnet
R\[F]éunis lisons la lo\[G]i,
C\[F]hantons tous d'une même \[C]voix, \[G]tralalala
\endchorus

\beginverse
J'ouvre ma porte à ce que l'autre m'apporte
Les différences, c’est une chance
Ensemble on joue, et on s’amuse comme des fous
Bientôt on d’vient, des copains
Donne ton avis, et c’est comme ça qu’on choisit
Dans le respect, c’est la paix
Servir on assure, et on protège la nature,
Sortir d’sa case, c’est la BASE
Tom et Prune sont nos amis
En vert pomme, réunis
\endverse

\beginverse
Sens dans ton coeur, un arc en ciel de couleurs
Identifions, émotions
Chante à Jésus, tout ce que l’on a vécu
L’amour rayonne, je pardonne
Je suis comme je suis, dans ma ronde d’amis
Jésus merci, accueilli
Que c’est épatant, de découvrir ses talents
Viens dans la ronde, change le monde
Tom et Prune sont nos amis
En vert pomme, réunis
\endverse

\endsong


%%%%%%%%%%%%%%%Au coeur de la planète
\beginsong{Au coeur de la planète}[by={Hubert Bourel},cr=2016]

\beginverse
L\[Gm]a confiance est à l’acc\[Dm]ueil
L’\[Gm]orée des forêts t’app\[Dm]elle
L\[Bm]’amitié d\[F]onne des ailes
V\[Gm6]iens on v\[Gm7]a franc\[A]hir le seuil…
A\[Gm]uprès de la peup\[Dm]lade,
C\[Bm]e sont tes am\[F]is,
T\[Gm7]u vas réus\[A]sir ta v\[Dm]ie.
\endverse

\beginchorus
\[Dm]Un coeur \[C]bat \[A]au coeur de la plan\[Dm]ète
\[Dm]C’est ton \[C]coeur \[A7]louveteau jeann\[Dm]ette
\[Gm]Tu t’hab\[C7]illes aux cou\[Dm]leurs du \[Gm7]soleil le\[Dm]vant
\[B&]Du so\[C]leil le\[Dm]vant
\endchorus

\beginverse
Pour l’ambiance ensoleillée t’as des blagues plein les poches
Des idées plein la caboche, des histoires à raconter…
Auprès de la peuplade, ce sont tes amis,
Tu vas réussir ta vie.
\endverse

\beginverse
Tu vas devenir veilleur, tu grandis et tu t’engages
Tu signes au bas de la page, le monde devient meilleur
Auprès de la peuplade, ce sont tes amis,
Tu vas réussir ta vie.
\endverse
\endsong


%%%%%%%%Allez viens
% \songpos{0}
\beginsong{Allez viens !}[,cr=2021]

\beginchorus
Allez \[Em]viens, \[C]viens, \[G]viens dans la fo\[D]rêt (bis)
B\[Em]ienvenue dans \[C]la peuplade,
T\[G]on équipe dans l\[D]a cabane.
M\[Am]ets ta chemise, t\[Em]on foulard,
A\[C]llez rejoins-no\[D]us,
Allez vi\[Em]ens !
\endchorus

\beginverse
T\[Am7]u quittes pap\[Bm]a maman,
\[C]Au revoir le\[D]s parents.
S\[Am7]alut les cop\[Bm]ains :
En o\[C]range on est b\[D]ien.

Au cœur de la nature,
Tu vas apprendre en jouant,
Tu verras c'est pas si dur,
Tu viens vivre une aventure.
\endverse

\beginverse
Tu deviens louveteau-jeannette :
Jouons, dansons, c'est la fête.
Après s'être présenté,
On prend le goûter.

Tous ensemble pour faire des nœuds,
À plusieurs c'est toujours mieux.
Des étoiles plein les yeux
En chantant autour du feu.
\endverse
\endsong

%%%%%%%La vie en bleu (connecte!)
\beginsong{La vie en Bleu (Connecte!)}[{by=Blue Monkeys}, cr=2019]

\beginverse
\[G]L'esprit et les \[D]yeux ouverts, \[Em]attentif au monde qui \[C]m'entoure
Pas \[G]toujours facile \[D]d'y voir clair, \[Em]parmi ces images, \[C]ces discours
Je \[Em]me sens un peu \[C]pris au piège, entre \[D]la télé, les \[Em]infos
Ce \[Em]qui se raconte au \[G]collège, les \[Em]buzz sur les réseaux \[D]sociaux
\[G]Pas trop \[D]vite. \[Em]Ça \[C]cogite.Prendre le temps du discernement.
\[G]Entre deux \[D]clics. \[Em]Esprit \[C]critique. Fonder mon propre \[D]jugement.
\endverse

\beginchorus
Je \[G]vois la vie en \[C]bleu, chaque \[Em]jour ça m'rend \[D]heureux
Je \[C]veux me rendre \[G]utile, si \[Em]j'y crois c'est \[D]facile
Je choisis d'être actif, de penser collectif
Et artisan de paix, m'engager, être vrai
\[C]Avec mon \[Em]équipage, ensemble sur le même \[D]bateau
On \[C]chante ce Re\[Em]frain, qui résonne en éc\[D]ho, \[G]co... Connec\[D]te!
\endchorus

\beginverse
Le coeur et les yeux ouverts
Soucieux de ce que l'autre pense
Pas toujours facile d'être fier
Et d'assumer mon apparence
C'est bien tentant de se cacher
Derrière des filtres ou des montages
Pourquoi une photo retouchée
Quand j'peux montrer mon vrai visage?
Mon physique.
Mes mimiques.
Affirmer mon identité.
Être unique. Authentique.
M'accepter en vérité.
\endverse


\beginverse
La porte et les bras ouverts
Prêt à accueillir l'inconnu
Avec ou sans foulard, mon frère
Je fais de toi le bienvenu
Parfois on aurait plus envie
D'aller vers ce(ux) que l'on connaît
Moi je veux rel'ver le défi
De la rencontre et de la paix
Créer des liens.
Que j'entretiens.
Sans cesse me laisser surprendre.
Tendre la main. Citoyen.
L'Autre a tellement à m'apprendre.
\endverse

\endsong

%%%%%%%%%%%%%Oser en grand
\beginsong{Avec ma tribu}[by={Ambplitude},cr=2012]

\beginchorus
Un mo\[G]nde tout en bl\[D]eu, des pro\[G]jets plein les y\[D]eux
Des t\[G]erres à explore\[D]r avec \[C]ma \[D]tri\[G]bu!
Une aventure en vrai à portée d'équipage
Pour grandir chaque jour avec ma tribu!
\endchorus

\beginverse
Bi\[Em]envenue à toi qui vient d'ar\[Am]rive\[Em]r
Pour trouv\[C]er et découvr\[D]ir ce qui nous r\[G]end heu\[D7]reux!
Avec \[Em]ma tribu nous serons témo\[Am]in\[Em]s
Des pas q\[C]ue tu fera\[C]s d'aventure e\[G7]n aventur\[D7]e!
\endverse
\beginverse
Un jour tu voudras vivre notre loi
Avancer et t'engager avec ton équipage!
Devant ma tribu tu diras je crois
Promesse d'avancer d'aventure en aventure!
\endverse
\beginverse
Avec les copains rêver à demain
Explorer et inventer un défi pour chacun!
Avec ma tribu nous irons plus loin
Camper et décamper d'aventure en aventure!
\endverse
\beginverse
Après ces années tu relis ta vie
Éclaireur de la tribu un brevet pour grandir
Et de ma tribu alors tu iras
Vers d'autres horizons retrouver la caravane!
\endverse

\endsong


%%%%%%%%%%%%%%%%%%%%%%%%%%%%%%%%%%%%%%%%%%%%%%%%%%%%%%%%%%%%%%%%%%%%%

\beginsong{Je prends le relais}[by={Jérémie Dazy},cr=2014]

\beginverse
Mal\[Am]gré un monde qui se fissure, les exp\[F]ériences me rassurent
M’ou\[C]vrir au monde, passer les murs, ren\[G]contrer, vivre l’aventure.
Les an\[Am]nées passées à grandir m’on\[F]t amené prêt pour servir
Les y\[C]eux tournés vers l’avenir, un r\[G]egard vert plein de plaisir
\endverse

\beginchorus
Je prends le r\[Am]elais, je prends le r\[Am7]elais
Pour v\[F]ivre ma vie, éc\[G]rire des sourires
On prend le r\[Am]elais, on prends le r\[Am7]elais
Pour a\[F]gir ensemble, là \[G]où bon nous semble
Je \[Am]suis, tu e\[G]s, nous s\[F]ommes
Des \[G]citoyens du monde
Nous s\[Am]ommes, vous ê\[G]tes, \[F]ils sont
\[G] Des compagnons \[Am]

\endchorus

\beginverse
Notre histoire s’écrit à plusieurs pour partager toutes nos valeurs
Les rires, l’espoir et les projets font de nous tous, une unité
On a négocié des virages, on a avancé sans naufrage
Un seul mot d’ordre, fraternité, pour une terre à partager
\endverse

\endsong

%%%%%%%%%%%%% Cap O 100

\beginsong{Cap O 100}[by={Chant Scouts Marins},cr=2015]

\beginchorus
C\[Am]ap O 100 moussaillon, on lève l\[E]'ancre
C\[Am]ap O 100 et hisse et ha\[G]ut 
Vo\[Dm]iles au vent marin, b\[G]arre à Tr\[c]ibord
Na\[Dm]viguons jusqu'à \[E]bon por\[Am]t
Cap O 100 moussaillon, on lève l'ancre
Cap O 100 et hisse et haut
Voiles au vent marin, barre à Bâbord
Naviguons jusqu'à bon port
\endchorus

\beginverse
Une casquette, ou bien ton baschi
Une chemise, un rayé aussi
Scout marin avec ton foulard
100 ans de mer, 100 ans de gloire
\endverse

\beginverse
Sur un trois mâts ou une barque
Un équipage qui prend le large
Scout marin sur ton voilier
Chante plus fort qu'les canonniers
\endverse

\beginverse
En eaux calmes ou océan houleux
Sous un ciel bleu ou orageux
Scout marin tu sais très bien
Comme Ulysse voyager loin!
\endverse


\endsong


%%%%%%%%%%%%%%%%%%%%%%%%%%%%%%%%%%%%%%%%%%%
\beginsong{La caravane}[by={Amplitude}, cr={2012}]

\beginverse
Dans \[Am]ce voyage\[C] aux m\[Am]ille visages\[C]
Ren\[Am]contrer l'autre \[C]sur la \[Am]route\[C]
Nos dif\[Am]férences \[C] on \[Am]les partage \[C]
Par\[Am]tons ensemble \[C]sans \[Am]bagages \[C]
\endverse


\beginchorus
La \[F]caravane prend \[G]le départ
\[Am]Vers une desti\[F]nation bien con\[C]nue \[G]
\[Am]D'étapes en étapes, \[F]expériences vé\[C]cues \[G]
\[Am]Sans toi il n'y \[F]aura pas d'his\[C]toire \[G]
En rou\[D]te pour un nouveau voyage
\[F]Pour tracer en\[G]semble notre sill\[Am]age 
\endchorus

\beginverse
Dans ce voyage aux mille visages allons plus loin chaque seconde
Prenons le temps, des paysages, tendons nos mains vers le monde
\endverse

\beginverse
Dans ce voyage aux mille visages osons le doute, les questions
La vie avance à chaque virage, nous affirmons nos passions
\endverse

\beginverse
Dans ce voyage aux mille visages posons nos sacs pour l'étape
Vivons la fête, sous les étoiles demain commence un autre cap
\endverse
\endsong

%%%%%%%%%%%%%%%%%%%%%%%%%%%%%%%%% Unis pour l'expédition
\beginsong{Unis pour l'expédition}[by={Clémence Michoud}, cr={2023}]

\beginchorus
\[Am]Allez, allez, \[F]allez! \[C]Unis pour l'ex\[G]pédition
\[Dm]Dépassons-nous \[E7]dans l'action
\[Am]Allez, allez, \[F]allez! \[C]Chantons en pre\[G]nant la route,
\[Dm]Ensemble on est \[E7]fiers d'être \[Am]scouts.
\endchorus

\beginverse
\[Am] Audacieuse, \[F]Audacieux
\[C] Avec un grand \[G]{\textbf{A}}
\[Am] Ce foulard nous \[F]rend heureux
\[C] Les scouts c’est vrai\[G]ment extra
\[E] On chante, on danse, on \[Am]est joyeux
\[F] On s’amuse, on rit aux é\[C]clats
\[E] Chaque jour on fait de \[Am]notre mieux
\[F] Notre promesse guide nos \[G]pas
\endverse

\beginverse
Nous formons une Unité
Avec un grand \textbf{U}
Au sein de notre cordée
Chacun est le bienvenu
Pour vivre ensemble, se dépasser
Faire des rencontres inattendues
Coéquipières, coéquipiers
Nous accompagnent vers l’inconnu
\endverse

\beginverse
Nous aimons vivre Dehors
Avec un grand \textbf{D}
La nature est un trésor
Que nous voulons protéger
Faire un grand feu, nous on adore
Pour la cuisine ou la veillée
On vit en forêt, on explore
On campe sous la voie lactée
\endverse

\beginverse
Au long du sentier, j’Avance
Encore un grand \textbf{A}
Mon parcours est une chance
Je progresse pas à pas
Je construis des liens de confiance
Pour les autres, je suis toujours là
Avec énergie, espérance
Au sommet, rien ne m’arrêtera
\endverse

\beginverse
Aider ça nous tient à Cœur
Avec un grand \textbf{C}
Chacun donne le meilleur
Pour servir et s’entraider
Entre nous, c’est la bonne humeur,
La paix, la solidarité
Dans notre équipe, toutes ces valeurs
Sont la clé de belles amitiés
\endverse

\beginverse
Partons pour l’Expédition
Avec un grand \textbf{E}
Tous dans la même direction
Vers un projet fabuleux
Des aventures par millions
La montagne pour terrain de jeux
Fêtons ensemble ces émotions
Notre bonheur est contagieux
\endverse
\endsong

%%%%%%%%%%%%%%%%%%%%%%%% Chant Clameurs
\beginsong{Clameurs du Monde}[by={François Tardif, Johan Tardif, Maïwenn Tardif}, cr={2025}]
\beginverse
\[Bm] Malgré les peurs, les tem\[G]pêtes, les appels de la plan\[D]ète
Un \[A]cri pers\[Bm]iste
Sous les cendres des \[G]guerres, les droits du monde résis\[D]tent
Nos \[A]cœurs grand\[Bm]issent
Que nos clameurs \[G]franchissent les murs et les fron\[D]tières,
Que les voix des \[A]peuples s’unis\[Bm]sent
Clameurs de ceux qu’on \[G]oublie, de ceux qui es\[D]pèrent,
Et pour les démun\[A]is, rendons jus\[Bm]tice
\endverse

\beginchorus
En\[G]semble, on ré\[D]pond à l'ap\[A]pel du \[Bm]monde
En\[G]semble, on ré\[D]pond à l'ap\[A]pel
\[Bm]Ecoute, \[G]les voix \[D]rugir du \[A]fond de nos cœurs 
\[Bm]Ecoute, \[G]les voix \[D]rugir oh, \[A]oh !
Faisons monter nos \[Bm]clameurs, \[G]clameurs, \[D]clameurs oh, \[A]oh ! (x2)
\endchorus


\beginverse
Malgré les bouleversements, les changements de la terre
La joie apparaît
Sous le regard silencieux, toutes les nations s’affrontent
La paix renaît
Que nos clameurs dévoilent nos espoirs et nos rêves
Que toutes nos angoisses disparaissent
Clameurs des cœurs brisés, des âmes déchirées
Et pour l’avenir, une promesse
\endverse

\endsong

%%%%%%%%%%%%%%%%%%%%%%%%%%% Sonne les couleurs
\beginsong{Chevalier saluons nos couleurs}[by={}, cr={env. 1723}]

\beginchorus
\[D]Chevaliers, sal\[A]uons nos cou\[D]leurs
Sonne, \[D]sonne éclaireur, sonne les hon\[A]neurs!
\[D]Sonne bien, sonne-\[A]les de tout \[D]cœur
Sonne, \[D]sonne, éclaireur, son\[A]ne les hon\[D]neurs!
\endchorus

\beginverse*
Pour nous c’est \[D]fête quand sur nos têtes
Notre drapeau flotte bien \[A]haut.
Quand viendra \[D]l’ombre et la nuit sombre
Ses plis sa\[(Gmaj7)]crés se\[A]ront pl\[D]iés.
\endverse


\endsong




%%%%%%%%%%%%%%%%%%%%%%%%%%%%%%%%%%%% Souvenir qui passe
\beginsong{Souvenir qui passe}[by={François Lebouteux},cr=1914]
\transpose{-2}
\beginverse
\[E]Souve\[B7]nirs qui \[E]passent,
A\[B7]dieu l’école et l’ate\[E]lier.
Le camp \[B7]les rem\[E]place
Avec ses feux \[B7]à la veil\[E]lée.
\endverse

\beginchorus
\[E]Ne tourne \[A]pas la \[E]tête, \[E]un scout re\[B7]garde en a\[E]vant (bis)
\endchorus

\beginverse
Dans la pâle aurore,
Nous quittons la ville endormie.
Ils dorment encore,
Nos pas les réveillent à demi.
\endverse

\beginverse
Aux clartés brûlantes
La halte n’arrive jamais
Si mon copain chante,
Je chante avec lui pour l’aider
\endverse

\beginverse
Dans le soir qui baisse,
Je pense aux copains prisonniers
J’en fais à ma tête,
Ce soir je suis en liberté.
\endverse

\beginverse
Dans la nuit profonde
Je marche en rêvant au passé
Mon copain me montre
Par où les anciens son passés.
\endverse
\endsong

%%%%%%%%%%%%%%%%%%%%%%%%% LOVE
\beginsong{Love}[by={Jean-Claude Gianadda}, cr={2008}]
\beginchorus
\[Am]Love, c'était son \[Dm]nom, \[G7]Love
Un vaga\[C]bond qui vivait de so\[E7]leil, d'espace et de chan\[Am]sons
\endchorus

\beginverse
Il est venu chez \[Am]nous, guitare en bandou\[Dm]lière
Venait d'on ne sait \[G]où, il parcourait la \[C]terre
Et dans ses longs che\[E7]veux le vent semblait chan\[Am]ter
Tout au fond de ses \[E7]yeux dansait la liber\[Am]té
\endverse

\beginverse
Il écoutait le vent, les fleurs et les rivières
Jouait comme un enfant parlait à la lumière
Il partageait ses rires, ses rêves et ses projets
Et dans chaque sourire dansait la liberté
\endverse

\beginverse
Il est parti un jour, nul ne sait où il est
Au pays de l'amour tu peux le rencontrer
Mais dans notre maison il nous aura laissé
Avec cette chanson, un peu de liberté
\endverse

\endsong



%%%%%%%%%%%%%%%%%%%%%%%%%%%%%%%%% Changer le monde
\beginsong{Changer le monde}[by={Jamboree}, cr={2021}]

\beginverse
\[Dm]Même quand le \[Am]ciel vire au \[C]gris, derrière il \[G]cache un peu de bleu
Sous son armure mon cœur sourit, fais un effort découvre-le
Si on te dit que la vie n'est pas belle, répond leur que c'est là tout l'enjeu
De voir dans les nuages des merveilles quand il pleut
Si on te dit que le monde est cruel, réponds-leur qu'on te le révèle mieux
On a tous un petit bout de soleil dans nos yeux
\endverse

\beginchorus
Puisqu'on est \[Dm]deux qu'on se ras\[Am]semble
Puisqu'on le \[C]peut qu'on se res\[G]semble
Impossible on le laisse au vent
On est prêt, on est prêt on est prêt
Rien ne sera jamais comme avant
Demain est un nouveau présent
Qu'on redessine comme des enfants
On est prêt, on est prêt, on est prêt
À changer le monde, monde, oh, oh, oh, oh, oh \rep{2}
\endchorus

\beginverse
On a oublié en chemin que nos vies sont des châteaux de sables
Dont il faut savoir prendre soin, non ce ne sont pas que des fables
Le bonheur c'est comme un boomerang
Quand on le partage il nous revient
Si on a pas tous la même langue
Ça ne fait rien
C'est un point de vue, une question d'angle
De verre moitié vide ou moitié plein
On a toujours plus que c'qu'il nous semble dans nos mains
\endverse

\beginverse
Et si la vie nous joue des tours c'est que c'est la règle du jeu
Mais on se relèvera toujours, on fera de notre \[Dm]mieux
Et si la nuit remplace le \[Am]jour, si tous nos rêves prennent \[F]feu
Ce qui nous sauvera c'est l'a\[G]mour
\endverse
\endsong


%%%%%%%%%%%%%%%%%%%%% Ohé garcon
\beginsong{Ohé toi garçon (canon)}[by={Traditionnel}]

\beginverse*
\[Am]Ohé toi garçon, toi qui a pro\[Dm]mis
Un soir d'été \[E7]l'amour la fidélité
\[Am]Ohé vois garçon, vois tes compa\[Dm]gnons
Qui ouvrent leur cœur \[E7]à Jésus Sauv\[Am]eur !
\endverse
\textnote{Ossia : "Ohé toi pionnier", "Ohé caravelle"}
\endsong


%%%%%%%%%%%%%%%% Le monde m'appelle
\beginsong{Le monde m'appelle}[by={Jamboree}, cr={2021}]

\beginverse
Il faut que \[G]j'aille voir ai\[C]lleurs
Vers d'autres \[G]cieux et d'autres \[D]fleurs
Et s'il n'\[G]y a pas de che\[C]min
je saurai \[G]bien tracer le \[D]mien
Je pars, oui, \[Em]mais je n'oublie \[C]rien
De mon pa\[D]ssé ni de mes \[G]liens
Et j'ai gli\[Em]ssé dans mes ba\[C]gages
Bien \[D]plus que des i\[G]mages
\endverse

\beginchorus
Le monde m'a\[C]ppelle !
\[C] Le monde m'a\[D]ttend !
\[D] La vie est si \[C]belle
Je ne veux \[D]pas perdre mon \[G]temps
\[D] Le monde m'a\[C]ppelle !
\[C] Le monde m'a\[D]ttend !
\[G] La vie est si \[G]belle
Je ne veux \[D]pas perdre mon \[G]temps \[D]
\endchorus

\beginverse
Je ne peux pas me contenter
De la maison ou du lycée
Tous les autres ont tant à m'apprendre
Ils m'aideront à mieux m'comprendre
Je ne veux pas baisser les bras
Ailleurs on a besoin de moi
Même si parfois c'est difficile
Je sais que je peux être utile
\endverse

\beginverse
Non, je ne veux pas rejeter
Ce que mes parents m'ont donné
Mais si ils m'ont offert la vie
C'est pour la vivre comme j'ai envie
Et je construirai ma maison
Différente de leur maison
Je leur emprunterai des pierres
Mais la ferai à ma manière
\endverse
\endsong


%%%%%%%%%%%%%%%%%%%%%%%%%%%%%%%%%%%%%%%%
\beginsong{Ensemble on est mieux}[by={Théophile Renier},cr=2017]
\beginverse
J'm'en souviens, j'a\[Em]vais 6 ans,
J'comprenais \[Am]pas c'que voulaient mes parents
Quand ils m'ont \[C]dit "Mon enfant,
Va chez les \[B7]scouts, c'est important !"
Moi j'flippais grave au début
J'sortais d'chez moi, d'mes habitudes
Mais la ribambelle m'ouvre la porte
Et au final le rire m'emporte
\endverse

\beginchorus
\[Em]Ensemble on est mieux
On a du \[C]mal à s'dire adieu
Les scouts nous por\[G]tent, nous transportent
Nous font \[B7]danser comme le feu
\textnote{x2}
\endchorus

\beginverse
Puis j'me rappelle de mes 8 ans
Chez les Louv'teaux tout était différent
Je découvrais la vie au grand air
Les jeux dans l'bois, la vie sans manières
On court partout, on fait les fous
On rit, on chante, comme des loups
Avec la meute j'deviens plus grand
La force du loup, c'est le clan
\endverse

\beginverse
Arrivé chez les Éclaireurs
Plus personne ne me faisait peur
Quand j'suis en bande avec mes potes
J'me sens plus fort même pour la tot'
On vit ensemble et en patrouille
On devient les rois d'la débrouille
On construit même nos pilotis
À bout de bras et d'énergie
\endverse

\beginverse
Puis viennent les pi's et leurs envies
D'aventure et de fantaisie
Pour ça on rêve tout' l'année
On veut qu'une chose c'est tout changer
Durant le camp, on s'ouvre à tout
Une autre culture, un autr' chez nous
On vient, on aide, si on peut,
On échange même avec des vieux
\endverse

\beginverse
Enfin, tu es animateur
Du temps, du talent et du cœur
Et tu t'engages bénévolement
À transmettre tout ce que t'as dans l'sang
Parfois c'est vrai, c'est la galère
Mais c'est pas grave, en staff tu gères
Baden-Powell est fier de toi
Chante avec moi et lève 3 doigts
\endverse

\beginchorus
\textnote{Refrain x4}
\endchorus
\endsong


%%%%%%%%%%%%%%%%%%%%%%%%%%%%% Dans ce monde qui avance
\beginsong{Dans ce monde qui avance}[by={Hubert Bourel}, cr={2012}]
\beginchorus
\[E]Dans ce monde qui \[B]avance, \[A]je fais de mon mie\[B]ux
Pour \[E]faire vivre l'espé\[B]rance, \[A]et gran\[B]dir le \[E]feu
\endchorus

\beginverse
Je \[A]me connais, 5u \[E/G#]te connais.
Je \[F#m7]sais miser sur \[B]mes atouts.
Je \[A]le promets, tu \[E/G#]le promets.
En\[F#m7]semble on ira \[B]jusqu'au bout.
\[F#m]Quand l’un de nous \[C#m]tombera.
\[F#m]On pourra comp\[A]ter sur \[B]toi.
\endverse

\beginverse
Je me connais, tu te connais.
Tu sais qu'on sera toujours là.
Je le promets, tu le promets.
Même si j'oubliais la loi
Quand l'un de nous tombera
On pourra compter sur toi
\endverse

\beginverse
Je me connais, tu te connais.
Avec le secours du Seigneur
Je le promets, tu le promets.
Prenons la vie du coté cœur!
Quand l’un de nous tombera
On pourra compter sur toi
\endverse

\beginverse
Je me connais, tu te connais.
Je sais prendre confiance en moi
Je le promets, tu le promets
La vie ouvrira grand ses bras
Quand l’un de nous tombera
On pourra compter sur toi
\endverse

\beginchorus
\[F]Dans ce monde qui ava\[C]nce
\[B&]Je fais de mon \[C]mieux
Pour \[F]faire vivre l'espé\[C]rance
\[B&]Et gran\[C]dir le \[F]feu
\endchorus

\beginverse
Je \[B&]me connais, tu \[F/A]te connais
Je \[Gm]me sais déjà \[C]bâtisseur
Je \[Bb]le promets, tu \[F/A]le promets.
\[B&/G]Avec vous d'un \[C]monde meilleur
\[Gm]Quand l’un de nous \[Dm]tombera
\[Gm7]On pourra\[B&] compter sur \[C]toi
\endverse

\endsong

%%%%%%%%%%%%%%%%%%%%% La promesse
\beginsong{La promesse}[by={Grégoire, Jean-Jacques Goldman}, cr={2010}]

\beginverse
\[Em]On était quelques \[C]âmes
\[Am]Quelques hommes quelques \[D]femmes rêvant de liber\[Em]té
On n'était pas à \[C]vendre
\[Am] Mais on pouvait \[D]revendre des montagnes d'ami\[G]tié
Le cœur en bandou\[Em]lière
Et les bras grands \[Am]ouverts à tous les ét\[D]rangers
On n'avait \[B7]pas de \[Em]peur
\[C] On sentait la chaleur qu'on savait se don\[B7]ner
\endverse

\beginverse*
\[Am]Même au fin fond du \[Em]désert
\[D]On aidait les plus \[G]faibles à ne \[E/G#]jamais tom\[Am]ber
Même au milieu des \[Em]chimères
\[C]On y croyait plus fort quand le courage man\[B7]quait
\endverse

\beginchorus
\[G]Oh vous mes compagnons mes amis de jeu\[D]nesse
\[Em]Quelles que soient vos histoires non n'oubliez \[B7]jamais
\[G]Qu'un beau jour nous avions fait ensemble une pr\[D]omesse
\[C]S'il n'en reste qu'un nous seront ce \[B7]dernier
\endchorus

\beginverse
On était plein d'ardeur
Et l'on sortait vainqueur de nos pauvres blessures
Quand les cœurs étaient lourds
On se trouvait toujours une voix qui nous rassure
On avait tant d'envies
Qu'on voyait notre vie comme une belle aventure
On n'avait pas de maître
La seule à nous soumettre était la mère nature
\endverse

\beginverse*
\[Am]Même au fin fond du \[Em]désert
\[D]On aidait les plus \[G]faibles et \[E/G#]quitte à y res\[Am]ter
Même au milieu des \[Em]chimères
\[C]On y croyait plus fort quand le courage man\[B7]quait
\endverse

\beginchorus
Oh vous mes compagnons mes amis de jeunesse
Quelles que soient vos histoires, ne m'oubliez jamais
Et si un jour je tombe faites moi cette promesse
S'il n'en reste qu'un vous serez ce dernier
Ce dernier, e dernier
\endchorus

\endsong


%%%%%%%%%%%%%%%%%%%%%%%%%%%%%%%%% Train d'enfer
\beginsong{Train d'enfer}[by={Raymond Cazanave}, cr={env. 1940?}]

\beginchorus
Embarquez \[Em]moi, partons en cara\[D]velle
Je suis la \[G]paix et je pars avec \[D]vous
Je suis la \[C]vie, je déplierai mes \[Bm]ailes,
Et nous ir\[C]ons \[Dm]vivre deb\[Em]out.
\endchorus

\beginverse
\[C]Nous sommes sorties des yeux de \[G]glace,
\[Am]Pour laisser fleurir nos id\[Em]ées  
\[F]Pour que le silence laisse \[E]place
Aux mots de \[F#7]liber\[B7]té.
\endverse

\beginverse
L'inconnu n'est plus à nos têtes,
L'aventure de plus en plus rare.
Le monde marche à la baguette,
Mais il n'est pas trop tard.
\endverse

\beginverse
Vivre est un mot un peu trop vague
Et je veux apprendre à aimer
A écouter le chant des vagues,
Avant de les briser.
\endverse

\beginverse
J'avais envie d'une autre histoire
Que celle qui sort de mes cahiers
En embarquant vers le hasard
Aujourd'hui, je suis née.
\endverse

\beginverse
Qui de vous deux est le plus riche,
Toi l'inventeur, toi l'inventé
Pour mettre un monde en musique,
Il faut savoir chanter.
\endverse

\beginverse
Madame, arrêtez de mentir
Devant la télé tous les soirs
Sors de tes murs, mets-toi à vivre,
Je t'offre ma guitare
\endverse

\endsong

%%%%%%%%%%%%%%%%%
\beginsong{Hé, garçon, prends la barre}[by={Jean Kerloch}, cr={1985}]
\beginverse
\[G]Hé, garçon, prends \[D7]la bar\[G]re, 
Vire au \[Em]vent et \[A7]largue les \[D7]ris.
Le \[Am]vent te ra\[C]conte l'hist\[D7]oi\[G]re
\[(D7)]Des mar\[G]ins couv\[D7]erts de gloir\[G]e.
Il t'ap\[Em]pelle et tu \[D]le \[G]suis.
\endverse

\beginverse
Vers les rives lointaines
Que tu rêves tant d'explorer
Et qui sont déjà ton domaine,
Va tout droit, sans fuir la peine
Et soit fier de naviguer.
\endverse

\beginverse
Sur la mer et sur terre,
Au pays comme à l'étranger,
Marin (ou routier), sois fidèle à tes frères
Car tu as promis naguère
De servir et protéger.
\endverse

\endsong


%%%%%%%%%%%%%%%%%%%%%%%%%%%%%%%%% La légende du feu
\beginsong{La légende du feu}[by={Jacques Sevin},cr=1924]
\beginverse
Les scouts \[C]ont mis la \[F]flamme aux bois \[G7]rési\[C]neux
Ecoutez chanter \[F]l'âme qui pal\[G7]pite en \[C]eux
\endverse

\beginchorus
Monte \[G7]flamme lé\[C]gère
Feu de \[G7]camp si chaud, si bon
Dans la plaine ou la clair\[C]ière
\[C7]Monte encore et \[G7]monte donc
Monte \[F]encore et monte \[C]donc
Feu de \[F]camp si \[C]chaud, \[G7]si \[C]bon
\endchorus

\beginverse
Autrefois j'étais prince perfide et méchant
Dépeuplant sa province des petits enfants
\endverse

\beginverse
Me tendit ses embûches l'enchanteur Merlin
M'enferma dans les bûches du grand bois voisin
\endverse

\beginverse
Depuis lors je dévore tout autour de moi
De me voir près d'éclore on tremble d'effroi
\endverse

\beginverse
C'est moi qui vous éclaire dans les longues nuits
Qui vous rend plus légère la peur ou l'ennui
\endverse

\beginverse
Les gerbes d'étincelles que je sème au vent
Emportent sur leurs ailes vos rêves d'enfant
\endverse
\endsong


%%%%%%%%%%%%%%%%%%%%%%%%%%%%%%%%%%%%%%%%%%%%%%%%%%%%%%%
\beginsong{Ce n'est qu'un au revoir}[by={Jacques Sevin},cr=1920]
\beginverse
Faut-\[F]il nous quitter \[C7]sans espoir,
Sans \[Dm]espoir \[F]de re\[B&]tour,
Faut-\[F]il nous quitter \[C]sans espoir
De \[Dm]nous re\[B&]voir \[C]un \[F]jour ?
\endverse

\beginchorus
Ce n'\[F]est qu'un au-re\[C]voir, mes frères
Ce n'\[F]est qu'un au-re\[B&]voir
Oui, \[F]nous nous rever\[C]rons, mes frères,
Ce n'\[Dm]est qu'un \[B&]au-\[C]re\[F]voir
\endchorus

\beginverse
Formons de nos mains qui s'enlacent
Au déclin de ce jour,
Formons de nos mains qui s'enlacent
Une chaîne d'amour.
\endverse

\beginverse
Unis par cette douce chaîne
Tous, en ce même lieu,
Unis par cette douce chaîne
Ne faisons point d'adieu.
\endverse

\beginverse
Car Dieu qui nous voit tous ensemble
Et qui va nous bénir,
Car Dieu qui nous voit tous ensemble
Saura nous réunir.
\endverse
\endsong

%%%%%%%%%%%%%%%%%%%%%%%%%%%%%%%%%%%%% Chant de Rouffach
\beginsong{La Rose du Mundat}[by={Groupe de Rouffach}, cr={2023}]
\beginchorus
\[Em]On est les scouts de \[G]Rouff\[A]ach
\[Em]Une petite ville en \[G]Als\[A]ace
\[Am]La Rose du Mundat \[C]c’est \[D]nous
\[Am]Notre nom de groupe est \[C]trop \[D]chou
\[Em]Rouge, jaune et vert c’est \[G]notre \[A]foulard
\[Em]On est vraiment des \[G]vein\[A]ards !
\endchorus

\beginverse
\[Em]La rose du mundat sans les \[A]scouts c’est pas \[Bm7]ça
\[Em]Car depuis l’debut ils sont \[A]toujours \[Bm7]là
\[Em]Sans eux la vie ne serait \[A]qu’un or\[Bm7]age
\[Em]On est les scouts guides et on \[A]n'prend pas \[Bm7]d’âge \[Bm7]
\endverse

\beginverse
Nous les farfadets avec Tom et Prune
On fait plein de jeux et on chante un peu
On apprend des choses avec les parents
Ça nous fait grandir et on est content
\endverse

\beginverse
Nous c’est les oranges rien ne nous dérange
On s’amuse comm’ des fous et on court partout
On fait des cabanes et autour du feu
On rêve, on rit, on chant’ Et tout nous enchante
\endverse

\beginverse
La rose du mundat sans les piok c’est pas ça
Même dans les tempêt’ on se re-lè-vera
Avec nous les rouges toujours des projets fous
Et pour tout nos caps on va jusqu’au bout
\endverse
\endsong


%%%%%%%%%%%%%%%%%%%%%%%%%%% Le chant de la 5e
% \songpos{1}
\beginsong{5e Mulhouse}[by={Jean-Clément Ringenbach}, cr={2025}]
\beginverse
\[Dm]Notre histoire com\[B&]mence en l'année \[F]193\[C]5
Par \[Gm]un groupe de Mul\[Dm]house formé \[Em7&5]à l'aube de la \[A7&9]guerre,
\[Dm]Rempli de Scouts \[B&]et Guides rassem\[F]blés par Charles Ven\[G]ner
\[Gm]Aux couleurs \[Dm]du gris et \[Em7&5]rouge ! \[A7]
\endverse

\beginchorus
\[D]Ve Mul\[G]house, on \[A7]t'a dans le \[D]coeur,
\[D]Même sous la \[G]pluie, on \[Em]garde la bonne \[A9]humeur 
\[D]Ve Mul\[G]house, on \[A7]est toujours \[Bm]là
\[C]On est fiers \[G]d'être scouts et \[Ab5/E&]on conti\[A7]nue comme \[D]ça \[D4] \[G(m)] \[D]
\endchorus

\beginverse
Qu'on soit arrivés aux farfas ou un peu plus tard,
Les souvenirs des camps sont gravés dans nos mémoires,
Le feu, le vent, les noeuds, les jeux et les raviolis :
On a appris pour la vie!
\endverse
\endsong