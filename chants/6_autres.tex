
% % \beginsong{Ah tu sortiras biquette}[by=Traditionnel\ français]
% % \beginchorus
% % \[G]Ah ! Tu sortir\[D]as, Biquette, Bi\[G]quette,
% % Ah ! Tu sortiras de ce chou-là
% % Ah ! Tu sortiras, Biquette, Biquette,
% % Ah ! Tu sortiras de ce chou-là
% % \endchorus

% % \beginverse
% % \[D]On envoie chercher le chien, \[G](bis)
% % \[D]Afin de mordre Biquette. \[G](bis)
% % \[D7]Le chien ne veut pas mordre Biquette.
% % Biquette ne veut pas sortir du chou.
% % \endverse

% % \beginverse
% % On envoie chercher le loup, (bis)
% % Afin de manger le chien. (bis)
% % Le loup ne veut pas manger le chien.
% % Le chien ne veut pas mordre Biquette.
% % Biquette ne veut pas sortir du chou.
% % \endverse

% % \beginverse
% % On envoie chercher l’bâton, (bis)
% % Afin d’assommer le loup. (bis)
% % Le bâton n’veut pas assommer le loup.
% % Le loup ne veut pas manger le chien.
% % Le chien ne veut pas mordre Biquette.
% % Biquette ne veut pas sortir du chou.
% % \endverse

% % \beginverse
% % On envoie chercher le feu, (bis)
% % Afin de brûler l’bâton. (bis)
% % Le feu ne veut pas brûler le bâton.
% % Le bâton n’veut pas assommer le loup.
% % Le loup ne veut pas manger le chien.
% % Le chien ne veut pas mordre Biquette.
% % Biquette ne veut pas sortir du chou.
% % \endverse

% % \beginverse
% % On envoie chercher de l’eau, (bis)
% % Afin d’éteindre le feu. (bis)
% % L’eau ne veut pas éteindre le feu.
% % Le feu ne veut pas brûler le bâton.
% % Le bâton n’veut pas assommer le loup.
% % Le loup ne veut pas manger le chien.
% % Le chien ne veut pas mordre Biquette.
% % Biquette ne veut pas sortir du chou.
% % \endverse

% % \beginverse
% % On envoie chercher le veau, (bis)
% % Pour lui faire boire l’eau. (bis)
% % Le veau ne veut pas boire de l’eau.
% % L’eau ne veut pas éteindre le feu.
% % Le feu ne veut pas brûler le bâton.
% % Le bâton n’veut pas assommer le loup.
% % Le loup ne veut pas manger le chien.
% % Le chien ne veut pas mordre Biquette.
% % Biquette ne veut pas sortir du chou.
% % \endverse

% % \beginverse
% % On envoie chercher l’boucher, (bis)
% % Afin de tuer le veau. (bis)
% % Le boucher n’veut pas tuer le veau.
% % Le veau ne veut pas boire de l’eau.
% % L’eau ne veut pas éteindre le feu.
% % Le feu ne veut pas brûler le bâton.
% % Le bâton n’veut pas assommer le loup.
% % Le loup ne veut pas manger le chien.
% % Le chien ne veut pas mordre Biquette.
% % Biquette ne veut pas sortir du chou.
% % \endverse

% % \beginverse
% % On envoie chercher le diable, (bis)
% % Pour qu’il emporte le boucher. (bis)
% % Le diable veut bien prendre l’boucher.
% % Le boucher veut bien tuer le veau.
% % Le veau veut bien boire l’eau.
% % L’eau veut bien éteindre le feu.
% % Le feu veut bien brûler le bâton.
% % Le bâton veut bien assommer le loup.
% % Le loup veut bien manger le chien.
% % Le chien veut bien mordre Biquette.
% % Biquette veut bien sortir du chou !
% % \endverse

% \endsong
% %%%%%%%%%%%%%%%% Le saucisson
% \beginsong{Je veux du saucisson}[by={Elias Campredon}, cr=2021]

% \beginverse
% Oui, j'veux du bon \[F]saucis\[C]son, non pas \[G7]d'la contre-fa\[Am7]çon
% De la très \[F]bonne fac\[C]ture: Oui, de \[G7]la matière \[C]sûre.
% \endverse

% \beginchorus
% J'veux du \[F]sauci\[C]sson, j'veux du \[G7]saucis\[Am]son.
% J'veux du \[F]saucis\[C]son, je veux \[G7]du sauci\[C]sson!
% \endchorus

% \beginverse
% Emballage en plastique, ou bien en papier carton.
% Pas b'soin d'faire l'arithmétique: oui, je veux du saucisson.
% \endverse

% \beginverse
% Avec des herbes ou du poivre, ou plus classique de la peau,
% J'adore en manger le soir, c'est inné, c'est dans ma peau!
% \endverse
% \beginverse*
% Avec plein ou peu de gras, peu importe tout me va,
% Tant qu'il y a du saucisson: Je suis complètement à fond!
% \endverse

% \beginverse
% Le seul et unique saucisson, c'est vraiment beaucoup trop bon,
% A chaque bout j'me sens mieux parce que c'est trop délicieux!
% \endverse
% \beginverse*
% Alors je vous remercie, même si vous êtes aigri
% Vous êtes vraiment trop sympa d'aider un gars comme moi!
% \endverse
% \endsong
